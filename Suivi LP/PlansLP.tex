\documentclass[11pt]{report}

%Packages--------------------------------------------

\usepackage[twoside, top=2cm,bottom=1.5cm,  hmargin=1.5cm, bindingoffset=0.2cm]{geometry}

\usepackage[utf8]{inputenc}
\usepackage[T1]{fontenc}
\usepackage[french]{babel}

\usepackage[most]{tcolorbox}

%----------------------------------------
% Pour trier

\usepackage{datatool}% http://ctan.org/pkg/datatool
\newcommand{\sortitem}[1]{%
  \DTLnewrow{list}% Create a new entry
  \DTLnewdbentry{list}{description}{#1}% Add entry as description
}
\newenvironment{sortedlist}{%
  \DTLifdbexists{list}{\DTLcleardb{list}}{\DTLnewdb{list}}% Create new/discard old list
}{%
  \DTLsort{description}{list}% Sort list
  \begin{itemize}%
    \DTLforeach*{list}{\theDesc=description}{%
      \item \theDesc}% Print each item
  \end{itemize}%
}
%-------------------------------------------

\usepackage{titlesec}
\titleformat{\chapter}[display]{\normalfont\bfseries}{\Huge\thechapter}{0pt}{\Huge}

\usepackage{siunitx}

\usepackage{amsmath}
\usepackage{cite}
\usepackage{stmaryrd}
\usepackage{mathtools}
\usepackage{amsfonts}
\usepackage{amssymb}
\usepackage{array}
\usepackage{bbold}
\usepackage{dsfont}
\usepackage{mathrsfs}  
\usepackage{calc}
\usepackage{xcolor}
\usepackage{framed}
\usepackage{bm}
\usepackage{mdframed}
\usepackage{braket}
\usepackage{ulem}
\usepackage{comment}
\usepackage[toc,page]{appendix} 
\usepackage{csquotes}
\MakeOuterQuote{"}
\usepackage{dirtytalk}
\usepackage{epigraph}
\usepackage{subcaption}
\usepackage{tabularx}
\usepackage{booktabs}



\usepackage{lipsum}% pour les faux textes


%%% Encadré %%%%%%%%%%%%%%%%%%%%%%%%%%%%%%%%%%%%
\usepackage{tcolorbox}
\tcbuselibrary{theorems}

\newtcbtheorem[number within=section]{mytheo}{Activité pédagogique}%
{colback=myblue!5,colframe=blue!45!black,fonttitle=\bfseries}{th}


\titleformat*{\section}{\color{mycolor}\bfseries\Large}
\titleformat*{\subsection}{\color{mycolor2}\bfseries\large}
\titleformat*{\subsubsection}{\color{mycolor5}\bfseries}


%%%%%%%%%%%%%%%%%%%%%%%%%%%%%%%%%%%%%%%%%%%%%%%%%%%%%

\newtheorem{property}{Property}
\newtheorem{corollary}{Corollary}
\newtheorem{definition}{Definition}
\newtheorem{conjecture}{Conjecture}

\definecolor{Prune}{RGB}{99,0,60} % l14-33: 
\definecolor{mycolor}{RGB}{186, 11, 122} % l14-33: 
\definecolor{mycolor2}{RGB}{4, 87, 87} %{4,139,154} 

\definecolor{mycolor3}{RGB}{23, 101, 125}
\definecolor{mycolor4}{RGB}{252, 93, 93}
\definecolor{mycolor5}{RGB}{149, 0 , 193}


\usepackage[pagebackref, colorlinks=true, citecolor=blue, linkcolor=blue,urlcolor= blue]{hyperref}

\usepackage[capitalise]{cleveref}
\usepackage{caption}
\usepackage{float}

\usepackage{longtable}


\usepackage{graphicx}
\usepackage{pdfpages}


\usepackage{wasysym}

\usepackage{caption}
\usepackage{float}

\usepackage[utf8]{inputenc}
\usepackage[T1]{fontenc}


\usepackage{graphicx}
\usepackage{pdfpages}

\numberwithin{figure}{section}
\numberwithin{equation}{section}
\numberwithin{table}{section}

\usepackage[]{tcolorbox}
\usepackage[]{enumitem}
\usepackage[]{lipsum}
\usepackage[]{multicol}

%%%%%%%%%%%%%%%%%%%%%%%%%%%%%%%%%%%%%%%%%%%%%%%%%%
% Raccourcis 
%%%%%%%%%%%%%%%%%%%%%%%%%%%%%%%%%%%%%%%%%%%%%%%%%%
\newcommand{\ud}{\mathrm{d}}
\newcommand{\e}{\mathrm{e}}

\newcommand{\R}{\mathds{R}}
\newcommand{\rf}{\mathcal{R}}
\newcommand{\E}{\mathds{E}}

\newcommand{\bE}{\bm E}
\newcommand{\bB}{\bm B}
\newcommand{\bl}{\bm l}
\newcommand{\bv}{\bm v}
\newcommand{\br}{\bm r}

\newcommand{\f}{\varphi}
\newcommand{\g}{\boldsymbol{\gamma}}
\newcommand{\ut}{\tau}
\newcommand{\s}{\sigma}
\newcommand{\arcsinh}{\mathrm{arcsinh}}
\newcommand{\va}{A}
\renewcommand{\t}{\tau}
\newcommand{\w}{\boldsymbol{w}}
\newcommand{\ur}{{\mathrm{r}}}

\newcommand{\sm}{\mathrm{s}}
\newcommand{\um}{\mathrm{u}}
\newcommand{\Sr}{S_{\mathrm{r}}}


\renewcommand{\L}{\mathscr{L}}
\renewcommand{\H}{\mathscr{H}}


\newcommand{\A}{\boldsymbol{A}}
\newcommand{\C}{\boldsymbol{C}^n}
\newcommand{\Cd}{C_n}
\newcommand{\Ct}{\boldsymbol{C}^\tt}
\newcommand{\Ctt}{C^\tt}
\newcommand{\V}{\boldsymbol{V}^\tt}
\newcommand{\Vt}{V^\tt}
\newcommand{\bpi}{\boldsymbol{\pi}}
\renewcommand{\a}{\boldsymbol{a}}
\newcommand{\p}{\boldsymbol{p}}
\renewcommand{\P}{\boldsymbol{P}}
\renewcommand{\S}{\boldsymbol{F}}
\newcommand{\bmu}{\boldsymbol{\mu}}
\newcommand{\om}{\boldsymbol{\omega}}
\newcommand{\kk}{\hat{k}}
\newcommand{\bkappa}{\boldsymbol{\kappa}}
\newcommand{\K}{\hat{K}}
\newcommand{\uu}{\boldsymbol{u}}
\renewcommand{\r}{\mathsf{r}}
\newcommand{\er}{\epsilon \r}
\renewcommand{\ll}{\boldsymbol{l}}
\renewcommand{\f}{\boldsymbol{f}}
\newcommand{\ffp}{\boldsymbol{\f'}}
\newcommand{\h}{\boldsymbol{h}}
\newcommand{\hp}{\boldsymbol{h'}}
\newcommand{\M}{\boldsymbol{M}}
\newcommand{\N}{\boldsymbol{N}}
\newcommand{\x}{\boldsymbol{x}}
\newcommand{\y}{\boldsymbol{y}}
\newcommand{\bpsi}{\boldsymbol{\psi}}
\newcommand{\X}{\boldsymbol{X}}
\newcommand{\Y}{\boldsymbol{Y}}
\newcommand{\1}{\boldsymbol{1}}
\newcommand{\un}{\mathcal{N}}
\newcommand{\bg}{\boldsymbol{g}}
\renewcommand{\u}{\boldsymbol{u}}
\renewcommand{\bf}{\boldsymbol{\phi}}
\newcommand{\z}{\boldsymbol{z}}
\newcommand{\Z}{\boldsymbol{Z}}
\newcommand{\brho}{\boldsymbol{\rho}}
\newcommand{\n}{\boldsymbol{n}}
\newcommand{\m}{\boldsymbol{m}}
\renewcommand{\j}{\boldsymbol{j}}
\newcommand{\bDelta}{\bm{\Delta}}
\renewcommand{\d}{\bm{\mathfrak{D}}}
\newcommand{\df}{\mathfrak{D}}
\newcommand{\U}{\bm{U}}
\newcommand{\B}{\bm{B}}
\newcommand{\km}{W}
\newcommand{\kmb}{\mathcal{W}}
\newcommand{\D}{\mathcal{D}}
\newcommand{\tg}{\tilde{G}}
\newcommand{\bA}{\bar{\A}}
\newcommand{\btg}{\bm{\tilde{G}}}
\newcommand{\tips}{\mathrm{TiPS}}
\newcommand{\vp}{\chi}
\renewcommand{\tt}{\mathcal{T}}

\renewcommand{\thesection}{\arabic{section}} 



%----------------------------------------------------------


\selectlanguage{french}


\title{\textbf{Proposition de plans des leçons de physique}}
\date{Concours externe spécial de l'agrégation de physique-chimie, option physique -- Session 2023}
\author{\color{mycolor2}\bfseries \textit{Lydia Chabane}}

%\setcounter{chapter}{27}


%1erepage------------------------------------------

\begin{document}
\maketitle

%Sommaire-------------------------------------------

\begingroup
\hypersetup{linkcolor=black}
\tableofcontents
\endgroup

%Corps-----------------------------------------------

\chapter{Gravitation.}


\paragraph*{Niveau:} CPGE (2ème année)
\paragraph*{Prérequis:} 
\begin{itemize}
\item Mécanique du point
\item \'Electrostatique, théorème de Gauss
\item Ellipses.
\end{itemize}

\paragraph*{Bibliographie:}
\begin{itemize}
\item Mécanique: fondements et applications. Pérez.
\item Mécanique, Tome 1. Bertin-Faroux-Renault
\item Toute la mécanique, MPSI-PCSI, MP-PC-PSI. Bocquet, Faroux \& Renault. Dunod
\item Physique PCSI/MPSI, Tout-en-un Dunod, Nouveau programme
\item P. Brasselet, Mécanique. PCSI-MPSI.
\item  Mécanique H-prépa 1ère année MPSI-PCSI-PTSI
\item Physique Tout-en-un. Dunod PC-PC* 
\item Mécanique H-prépa 1ère année MPSI-PCSI-PTSI
\item Mécanique 1ère année. Gié \& Sarmant. Tec\&Doc.
\item \url{https://phet.colorado.edu/sims/html/gravity-and-orbits/latest/gravity-and-orbits_fr.html}
\item \url{https://www.geogebra.org/m/mmyeu93d}
\end{itemize}

  
\textcolor{mycolor4}{\textbf{N.B.:} Remarque Karim : vaut mieux bien faire une partie (soit sur l'interaction gravitationnel soit sur un problème à deux corps) en faisant des démonstrations.}


\section*{Introduction}
  \addcontentsline{toc}{section}{Introduction}

Slide: les 4 interactions fondamentales. Décrire brièvement chacune d'entre elles.

\section{Interaction gravitationnelle}


\subsection{Définition}

\subsection{Force gravitationnelle}

\subsection{Analogie avec l'électrostatique}

\subsubsection{Analogie et limites}

Présenter un tableau de l'analogie et discuter les limites.

[JF]
\begin{itemize}
\item Comme la charge est positive ou négative, l'interaction Coulombienne peut être attractive ou répulsive, là où l'interaction gravitationnelle est systématiquement attractive du fait de $m > 0$. D'ailleurs, deux charges de même signe se repoussent!
\item Différence importante d'échelle : l'électrostatique régit l'échelle atomique, alors que la gravitation régit le
mouvement des corps célestes.
\item $E$ dépendant du temps engendre $B$ , mais pas d'équivalent pour le champ de gravitation. D'ailleurs, l'électromagnétisme est à l'électrostatique ce que la relativité générale est à la gravitation : l'élargissement de la théorie aux particules en mouvement.
\end{itemize}

\subsubsection{Champ et potentiel de gravitation}

\section{Propriétés du champ gravitationnel}

[Pérez]

\subsection{Relation entre le champ et le potentiel}

\subsection{Calcul du potentiel et du champ pour une sphère uniforme}

Théorème de Gauss. Distinguer cas $r<R$ et $r>R$.

\section{Le poids}

[Sanz PC, p. 209]

\subsection{Au lycée}

\subsection{Redéfinition à partir du champ gravitationnel}

Bien définir $R_T$ et $h$ la hauteur. \\
Tracer l'évolution de g avec la distance.

\subsection{Mesure de $g$ (manip')}


\section{Applications}

\subsection{Vitesse de libération}

\subsection{Effets de marées}

Variation du champ de gravitation dû aux autres astres à la surface de la Terre.

\section*{Conclusion}
  \addcontentsline{toc}{section}{Conclusion}

La gravitation est à l'origine de
\begin{itemize}
\item La pesanteur.
\item Le mouvement des planètes.
\item Phénomènes de marées.
\end{itemize}


\section*{Description de l'expérience}
  \addcontentsline{toc}{section}{Description de l'expérience}



\begin{tcolorbox}[breakable, enhanced, colback=red!2!white,colframe=mycolor!85!black,title=\textbf{\textbf{Expérience}}]
\paragraph*{Matériel}
\begin{itemize}
\item Pendule pesant
\item Centrale d'acquisition Sysam-sp5
\item Câbles adaptés
\item Mètre ruban
\item Chronomètre
\end{itemize}

\paragraph*{Protocole } 

\begin{itemize}[label=$\triangleright$]
		\item Utiliser Latis Pro pour tracer $\theta(t)$. Lisser la courbe. Modéliser par un sinus. Lecture de $f$.
\end{itemize}

\paragraph*{Aspect quantitatif :} Mesure de $g$.
\begin{equation} \nonumber
f = \frac{1}{2\pi} \sqrt{\frac{g}{L}}.
\end{equation}


\end{tcolorbox}



\newpage

%------------------------------------------



\chapter{Lois de conservation en dynamique}


\paragraph*{Niveau:} CPGE
\paragraph*{Prérequis:} 
\begin{itemize}
\item Théorèmes généraux de mécanique newtonnienne
\end{itemize}

\paragraph*{Bibliographie:}
\begin{itemize}
\item Tec\&Doc MP. Gié, Sarmant et al. p. 303.
\item Toute la mécanique MPSI-PCSI-MP-PC-PSI. Bocquet, Faroux, Renault. J'intègre Dunod.
\item Mécanique PCSI-MPSI. Pascal Brasselet. puf
\item Mécanique: fondements et applications. Pérez
\item Dictionnaire de Physique. Taillet.
\end{itemize}

\paragraph{Notes agrégat}
\begin{itemize}
\item 2017 : Des exemples concrets d’utilisation des lois de conservation sont attendus.
\item 2016 : Lors de l’entretien avec le jury, la discussion peut aborder d’autres domaines que
celui de la mécanique classique.
\item 2015 : Cette leçon peut être traitée à des niveaux très divers. L’intérêt fondamental des lois
de conservation et leur origine doivent être connus et la leçon ne doit pas se limiter à
une succession d’applications au cours desquelles les lois de conservation se résument à une propriété anecdotique du problème considéré.
\end{itemize}

\section*{Introduction}
\addcontentsline{toc}{section}{Introduction}


[Taillet. p. 149] On dit qu'une grandeur est conservée si elle reste constante au cours d'un processus.

Trouver une loi de conservation, c’est trouver une quantité qui ne varie pas au cours du temps.

Nous verrons ici les lois de conservations de la mécanique newtonienne.

\section{Lois de conservation en dynamique}

\paragraph{Cadre de la leçon} Dire dès le début que les référentiels considérés lors de la leçon sont supposés galiléen.

[Tec\&Doc] En mécanique, on caractérise un système par deux grandeurs vectorielles: sa quantité de mouvement $p$ et son moment cinétique $L$. Et par une grandeur scalaire : l'énergie. Chacune de ces grandeurs obéit à une loi d'évolution qui* permet de faire le lien entre la variation d’une quantité et les causes de cette variation (principe fondamental de la dynamique,
théorème de l’énergie mécanique, théorème du moment cinétique,...). \\
(*) [JF] \url{https://www.lpens.ens.psl.eu/wp-content/uploads/2021/04/LeconsPhysique.pdf}.

\subsection{Lois d'évolution}

\begin{align*}
\mathrm{PFD}~~ \frac{\ud \bm p}{\ud t} &= \bm F  \\
\mathrm{TMC}~~\frac{\ud \bm L}{\ud t} &= \bm M_O \\
\mathrm{TEM}~~\frac{\ud E}{\ud t} &= \mathcal{P}_{nc}
\end{align*}

\paragraph{Conservation} [JF*] Dans les cas particuliers où le terme de cause
est nul, on met en évidence une grandeur dont la valeur (norme, direction et sens) est constante dans le temps. On
parle alors de loi de conservation.

JF = Jules Fillette.

\subsection{Conservation de la quantité de mouvement}

[Tec\&Doc, Brasselet]

Illustrer avec l'expérience du rebond de la balle [Brasselet, p. 270] ou pendule de Newton [en direct ou \url{https://fr.wikipedia.org/wiki/Pendule_de_Newton#:~:text=5.2%20Liens%20externes-,Description,plan%20m%C3%A9dian%20des%20deux%20barres.}] 
Faire le petit calcul pour l'un ou l'autre.

\subsection{Conservation du moment cinétique}

[Tec\&Doc, Brasselet]

[JF] La remarque fin de page de [Brasselet, p.40] sur l’apport du TMC par rapport au pfd est fort intéressante et peut être discutée ici. \\

Montrer vidéo tabouret. Faire la danseuse qui déplie et replie les bras [Pérez, p. 324].


\subsection{Conservation de l'énergie}

[Tec\&Doc, Brasselet]

\subsubsection{Expérience} [Brasselet, p. 260 et p.270] Chute d'une balle.

\subsubsection{Lien avec la thermodynamique} 

[Tec\&Doc]

\section{Application au mouvement de force centrale}

[BFR, p.134]

\section{Conservation et invariance}

[Tec\&Doc, p. 314. Brasselet, p. 36]

\subsection{Invariance par translation spatiale}

\subsection{Invariance par rotation}

\subsection{Invariance par translation temporelle}

\subsection{Théorème de Noether}




\section*{Conclusion}
\addcontentsline{toc}{section}{Conclusion}

Conclure sur 
\begin{itemize}
\item L'intérêt des lois de conservation dans la résolution de problème
\item Universalité de $p, L, E$ dans toute la physique [Tec\&Doc, p.311]. Donner quelques exemples sur slide.
\end{itemize}

\paragraph{N.B} Savoir comment dériver Noether (\url{https://nc.agregation-physique.org/index.php/s/ZEnsAc39tMbzH2W?dir=undefined&path=%2FLyon%2FM%C3%A9canique&openfile=48029} ou \href{https://fr.wikipedia.org/wiki/Th%C3%A9or%C3%A8me_de_Noether_(physique)#D%C3%A9monstrations}{wik}). \\
\paragraph{N.B} Lire sur Runge-Lenz.

\section*{Description de l'expérience}
  \addcontentsline{toc}{section}{Description de l'expérience}

\textcolor{mycolor5}{TP Outils informatiques} [Duffait CAPES. p.241] 

\begin{tcolorbox}[breakable, enhanced, colback=red!2!white,colframe=mycolor!85!black,title=\textbf{\textbf{Expérience}}]
\paragraph*{Matériel}
\begin{itemize}
\item Balle de ping-pong
\item Scotch de couleur
\item Grande règle
\item Caméra rapide
\item Grand écran \textbf{qui touche le sol}.
\end{itemize}

\paragraph*{Protocole } 

\begin{itemize}[label=$\triangleright$]
\item Brancher la caméra rapide
\item Faire la mise au point sur le logiciel Caméra de Windows 10: il faut qu'on voit le sol pour voir les rebonds.
\item Démarrer l'acquisition vidéo sur VirtualDub \textbf{Attention: lâcher la balle sans vitesse initiale à un point vue par la caméra}:
	\begin{itemize}
	\item Ouvrir le logiciel, et dans File, cliquer sur Capture AVI
	\item Dans la nouvelle fenêtre, aller sur l'onglet Device et vérifier que HD USB Camera est bien sélectionnée.
	\item dans l’onglet Video, cliquer sur Capture Pin. On peut alors changer la résolution, qui adapte automatiquement la fréquence d’acquisition. En profiter pour mettre la qualité au maximum.
	\item dans l’onglet Capture, aller dans Timing. Vérifier que toutes les cases sont décochées. Cela permet d’assurer la précision du traitement a posteriori.
	\item Choisir un nom dans File et Set capture file. Puis lancer l’acquisition dans Capture ou avec F5, pour l’arrêter, appuyer sur Esc.
	\end{itemize}
\item Traiter la vidéo sur Tracker:
	\begin{itemize}
	\item Ouvrir Tracker, puis ouvrir la vidéo en question. Tracker avertit l’utilisateur si les images n’ont pas des durées identiques sur l’enregistrement (ce qui n’arrive pas avec VirtualDub, mais est fréquent sinon).
	\item Délimiter la partie utile : placer sur le curseur sur le début souhaité, puis avec un clic droit définir l’image de départ. De même pour la fin.
	\item cliquer sur l’icône Ruban et créer un bâton de calibration. Définir le début et la fin en maintenant la touche Shift, et en cliquant sur les endroits choisis. Modifier la valeur de cette échelle.
	\item Créer une nouvelle trajectoire avec l’icône, et choisissez Masse ponctuelle. Se placer sur la première image de la séquence, cliquer sur Masse A, puis Repérage automatique. En maintenant Ctrl et Shift, cliquer sur l’objet à suivre. Puis cliquer sur Chercher. Le logiciel va rechercher cette forme sur toutes les images. Il peut bloquer sur certaines (souvent aux rebonds), dans ce cas déplacer à la main le gabarit sur l’objet, et cliquer sur Chercher ceci, et ainsi de suite.
	\end{itemize}
\item Sélectionner les bonnes variables ($y$ et $v_y$) et exporter les données vers QtiPlot.
\item Tracer $v_y(t)$, en déduire l'accélération donnée par la pente.
\item Tracer $E_m(t) = E_c(t) + E_p(t)$. Voir les plateaux. \textbf{N.B.} Il faut définir $E_p$ avec l'accélération mesurée et non pas $g$. On trouve $a = 11m/s^2$ et non pas $9.81ms^2$...
\end{itemize}

\paragraph*{Aspect quantitatif :} Mesure de  $E_m$. Évaluer le coefficient de restitution (rapport entre l'énergie de l'objet après et avant le choc) en vérifiant qu'il est identique lors de chaque rebond. On peut également vérifier la trajectoire de chute libre entre deux rebonds.

\end{tcolorbox}



\newpage

%------------------------------------------



\chapter{Notion de viscosité d'un fluide. Écoulements visqueux}


\paragraph*{Niveau:} 2ème année CPGE
\paragraph*{Prérequis:} 
\begin{itemize}
\item 
\end{itemize}

\paragraph*{Bibliographie:}
\begin{itemize}
\item Physique Spé. PC-PC* ou PSI-PSI*. Tec\&Doc. Olivier, More, Gié.
\item Mécanique des fluides. PC-PSI. H-prépa. Brébec, Desmarais.
\item Tout-en-Un Physique PC. Sanz et al.
\item Cours Moisy: \url{http://www.fast.u-psud.fr/~moisy/teaching/cours_mecaflu_2017.pdf}
\item Bonus: Hydrodynamique physique. \'Etienne Guyon, Jean-Pierre Hulin et Luc Petit.  CNRS ÉDITIONS (3ème édition) 
\item Bonus manip' : \url{http://materiel-physique.ens-lyon.fr/Logiciels/CD%20N%C2%B0%203%20BUP%20DOC%20V%204.0/Disk%201/TEXTES/1999/08140799.PDF}
\item Bonus manip': \url{http://materiel-physique.ens-lyon.fr/Logiciels/CD%20N%C2%B0%203%20BUP%20DOC%20V%204.0/Disk%201/TEXTES/1998/08010329.PDF}
\end{itemize}



\section*{Introduction}
  \addcontentsline{toc}{section}{Introduction}

\paragraph{Manip' qualitative} Faire couler différents fluides -> notion de viscosité

\section{Viscosité}

3'50: \url{https://www.youtube.com/watch?v=pqWwHxn6LNo&t=57s}

\subsection{Définition}

Action de contact exercées par le fluide situé en dessous de $z$ : $\ud F_t = \eta \frac{\partial v_x}{\partial z} \ud S u_x$.

Donner l'unité. Donner quelques ordres de grandeurs. Dépend de $T, P$.

\subsection{Interprétation microscopique}

Qualitativement.

\subsection{Force volumique de viscosité}

Hyp.: écoulement incompressible.


\subsection{Équation de Navier Stokes}



\section{Nombre de Reynolds}

\subsection{Définition}

Compétition entre convectif et terme diffusif: $R_e = \frac{v L}{\nu}$. Interprétation.

\subsection{Régimes d'écoulement}

[H-prépa]

Cas grand $R_e$ vs petit $R_e$ (limite à $\sim 2000$). Montrer des images écoulement laminaire vs turbulent. Donner quelques valeurs (cours M. Rabaud).

\subsubsection{Régime linéaire permanent}

\subsubsection{Non-linéarité, turbulences et chaos}

\section{Écoulement autour d'un obstacle}

Obstacle = sphère lisse.

\subsection{Force de traînée}

Force de traînée, coeff de traînée, dépendance en $R_e$ : deux régimes.

\subsection{Régime visqueux: équation de Stokes}

$F = 6 \pi \eta r v$.

\subsection{Application : mesure de $\eta$}

\paragraph*{Expérience} Tracer $v_{lim}(R)$. En déduire $\eta$. Au préalable: estimer à la caméra la valeur à quelle distance le régime stationnaire est atteint puis faire les mesures au chrono.
\\

\textcolor{mycolor5}{Transition: autre manière de mesurer $\eta$}



\section{Écoulements parallèle}

Choisir entre Poiseuille (plan ou cylindrique) et Couette (plan ou circulaire) selon la maîtrise et/ou le temps.

Montrer comment on remonte à $\eta$.

\paragraph*{Rq} Pour être cohérent avec la logique, idéalement faire le Viscosimètre de Couette (exo 7 p.153 de H-prépa ).

Au pire si vraiment pas la temps, le mettre en sous-section sur slide avec le profil de vitesse et l'équation qui permet de remonter à $\eta$ expérimentalement.

\section*{Conclusion}
  \addcontentsline{toc}{section}{Conclusion}
  
\begin{itemize}
\item Conclure sur couche limite
\item Quelques mots sur écoulements turbulents (H-prépa: balle de golf?)
\item Autres types de fluides (non-newtonien, rhéofluidifiant, rhéoépaississant).
\end{itemize}


\section*{Description de l'expérience}
  \addcontentsline{toc}{section}{Description de l'expérience}



\begin{tcolorbox}[breakable, enhanced, colback=red!2!white,colframe=mycolor!85!black,title=\textbf{\textbf{Expérience}}]
\paragraph*{Matériel}
\begin{itemize}
\item Récipient rempli de silicone
\item Chronomètre
\item Billes de différents rayons
\item Pied à coulisse
\item Règle ou mètre
\item Balance
\item Caméra rapide
\item Scotch coloré
\item Post-it
\item Écran blanc
\end{itemize}

\paragraph*{Protocole } 

\begin{itemize}[label=$\triangleright$]
\item Mesurer rayon et masse de la bille.
\item Au préalable: estimer à la caméra la valeur à quelle distance le régime stationnaire est atteint pour la plus grosse bille puis faire les mesures au chrono (enregistrer la chute puis traiter avec Tracker).
\item Calculer le nombre de Reynolds pour chaque bille: attention à la validité du modèle $R_e<1$.
\item Tracer $v_{lim}(R)$. En déduire $\eta$. Attention: si la valeur de Reynolds n'est pas valide pour des billes plus grandes et si pas assez de points alors faire une unique mesure. A priori: la plus grosse bille est pour $2$mm.
\end{itemize}

\paragraph*{Aspect quantitatif :} Mesure de $v_{lim}(R)$. En déduire $\eta$. 


\end{tcolorbox}


\newpage

%------------------------------------------



\chapter{Modèle de l'écoulement parfait d'un fluide}


\paragraph*{Niveau:} 2ème année CPGE
\paragraph*{Prérequis:} 
\begin{itemize}
\item 
\end{itemize}

\paragraph*{Bibliographie:}
\begin{itemize}
\item Physique Spé. PC-PC* ou PSI-PSI*. Tec\&Doc. Olivier, More, Gié.
\item Mécanique des fluides. PC-PSI. H-prépa. Brébec, Desmarais.
\item Tout-en-Un Physique PC. Sanz et al.
\item Cours Moisy: \url{http://www.fast.u-psud.fr/~moisy/teaching/cours_mecaflu_2017.pdf}
\item Bonus: Hydrodynamique physique. \'Etienne Guyon, Jean-Pierre Hulin et Luc Petit.  CNRS ÉDITIONS (3ème édition) 
\item Bonus manip' : \url{http://www.fast.u-psud.fr/~bertho/TP_L3.pdf}
\item Bonus manip' : \url{http://materiel-physique.ens-lyon.fr/Logiciels/CD%20N%C2%B0%203%20BUP%20DOC%20V%204.0/Disk%201/TEXTES/1999/08140799.PDF}
\item Bonus manip': \url{http://materiel-physique.ens-lyon.fr/Logiciels/CD%20N%C2%B0%203%20BUP%20DOC%20V%204.0/Disk%201/TEXTES/1998/08010329.PDF}
\end{itemize}

\section*{Introduction}
  \addcontentsline{toc}{section}{Introduction}
  
[JF] Dans le cas général de l'étude d'un fluide l'équation de Navier-Stokes est une équation compliquée à au moins
deux titres : la non linéarité, et la diffusion (laplacien de la vitesse). On étudie ici le cas ou la diffusion est négligeable,
on constate que cela simplifie grandement l'étude mais on expose aussi une limite du modèle.


[qqun] Dans une leçon précédente, nous avons étudié la dynamique des fluides visqueux, nous avons pu constater que le formalisme était lourd, bien que capable de décrire complètement les phénomènes dans le fluide. Dans cette leçon nous allons, au prix de quelques approximations, alléger l'équation de Navier-Stokes. Ceci est nécessaire, car résoudre les grands systèmes complexes avec l'équation originale est pratiquement impossible du fait de sa non-linéarité. Ceci
nous permettra aussi de traiter le cas des écoulements parfaits compressibles.

\section{Notion d'écoulement parfait}

[Tec\&Doc, fin de chapitre viscosité]

\subsection{Définition}

Définition. Intérêt: simple. \\

\textcolor{mycolor5}{Oui, mais dans la vraie vie n'existe pas.}

\subsection{Couche limite}

A l'exclusion d'une couche limite au voisinage de l'obstacle, le modèle de l'écoulement parfait est valable.

\textcolor{mycolor5}{Maintenant qu'on a justifier l'intérêt de ce modèle, on va introduire les équations qui permette d'étudier cet écoulement.}


\subsection{Équation d'Euler}

Obtention. Propriété.

Cette équation exprime le PFD pour une particule fluide d'un fluide parfait. Cette équation est valide pour un fluide compressible ou non.

Notons que l'équation d'Euler est une équation non-linéaire. Cela signifie que, si l'on connaît une solution $u$, alors $2u$ ne sera pas solution de cette équation en général. Cet aspect non-linéaire rend la résolution de l'équation difficile, et est une des raisons de la difficulté de la mécanique des fluides.

Éventuellement parler des CLs cinématiques (vitesse) et dynamique (pression).

\section{Relation de Bernoulli}

[H-prépa, Tec\&Doc]

Le théorème de Bernoulli est un résultat très utile aux nombreuses conséquences pratiques,
qui traduit la conservation de l'énergie mécanique pour un fluide parfait. Ce théorème découle de l'équation d'Euler, et existe sous deux variantes : stationnaire, ou instationnaire-mais-irrotationelle.

\subsection{Écoulement parfait stationnaire incompressible et homogène}

Dérivation le long d'une ligne de courant.

\subsubsection{Hypothèse supplémentaire: irrotationnel}

Simplification du théorème.

\subsection{Interprétation énergétique}

Interpréter les différents termes. Possible car pas de dissipation (conservation).

\section{Application}

\subsection{Tube de Pitot}

A quoi ça sert ? Monté sur avion, il permet de mesurer la vitesse de l'air "loin de l'avion" par rapport à l'avion.

\subsubsection{Modèle}

\subsubsection{Mise en équation}


\subsubsection{Expérience} Montrer que la vitesse varie bien en $\sqrt{\Delta P}$.

\subsection{Effet Venturi}

[Tec\&Doc, H-prépa]

\paragraph*{Manip' qualitative:} souffler sur des feuilles de papier.

\subsubsection{Principe}

\subsubsection{Limites du modèle de l'écoulement parfait}

\subsubsection{Application}

Vidéo de pleins d'expérience : \url{https://www.youtube.com/watch?v=E32YHDTDy-4}

Trompe à eau, débitmètre de Venturi, portance.


\section*{Conclusion}
  \addcontentsline{toc}{section}{Conclusion}
  
  
[JF] Le modèle de l'écoulement parfait est un outil efficace pour traiter de l'écoulement des fluides loin des obstacles et pour des écoulements laminaires. En revanche, l'analyse rigoureuse de comportement des fluides au voisinage d'obstacle doit être traitée avec l'outil général de Navier-Stokes, ou a minima en s'intéressant en détail au comportement de la couche limite!


\section*{Description de l'expérience}
  \addcontentsline{toc}{section}{Description de l'expérience}



\begin{tcolorbox}[breakable, enhanced, colback=red!2!white,colframe=mycolor!85!black,title=\textbf{\textbf{Expérience}}]
\paragraph*{Matériel}
\begin{itemize}
		\item Tube de Pitot
		\item Anémomètre à fil chaud (attention au sens !)
		\item Soufflerie
		\item Manomètre différentiel 
		\item Alimentation continue pour manomètre
		\item Multimètre (voltmètre)
		\item Fils
		\item Potence, noix, pinces, truc pour accrocher le manomètre
		\item Support élévateur
		\item Thermomètre
		\item Chronomètre 
\end{itemize}

\paragraph*{Protocole } 

\begin{itemize}[label=$\triangleright$]
\item Attention: bien faire la conversion pression/tension $\Delta P(V)$ et prendre en compte l'offset.
\item Attention: prendre la température de la pièce.
\item Attention à bien placer l'anémomètre dans le bon sens (flèche dans le sens de l'écoulement) ! Faire une mesure pour vérifier.
\item Mesurer plusieurs points $\Delta P(v)$ et faire une régression linéaire. En déduire $\rho$
\end{itemize}

\paragraph*{Aspect quantitatif :} Mesure de $\Delta P(v) = \frac{\rho v^2}{2}$. Faire une régression linéaire sur $v^2(\Delta P)$ car les incertitudes sont plus grandes sur $v$. En déduire $\rho_{\text{air}} = 1,204$ kg/m$^3$ à $20$°C au niveau de la mer.


\end{tcolorbox}

\newpage

%------------------------------------------



\chapter{Phénomènes interfaciaux impliquant des fluides}


\paragraph*{Niveau:} L3
\paragraph*{Prérequis:} 
\begin{itemize}
\item Forces, travaux de force, énergie.
\item Principe de travaux virtuels.
\item Equation de l'hydrostatique.
\end{itemize}

\paragraph*{Bibliographie:}
\begin{itemize}
\item  Gouttes, bulles, perles et ondes. David Quéré, Françoise Brochard-Wyart et Pierre-Gilles de Gennes. \'Editions Belin (2002)
\item Hydrodynamique physique. \'Etienne Guyon, Jean-Pierre Hulin et Luc Petit.  CNRS ÉDITIONS (3ème édition) 
\item Capillarité (cours). P. Lidon. \url{https://cel.hal.science/cel-01332274}
\item Notes de cours sur les fluides (2019-2020). Marc Rabaud. \url{http://ressources.agreg.phys.ens.fr/static/Cours-TD/Rabaud/NotesCours_Agreg2019.pdf}
\item Why is surface tension a force parallel to the interface? Marchand, A., Weijs, J. H., Snoeijer, J. H., \& Andreotti, B. American Journal of Physics (2011)
\item TP LLG \url{http://supernovae.in2p3.fr/~llg/Enseignements/Agregation/TP/Fluides-Capillarite/fluides-capillarite.pdf}
\end{itemize}

 \section*{Introduction (3min)}
 \addcontentsline{toc}{section}{Introduction}

L'étude des phénomènes interfaciaux permet de répondre à un certain nombre de questions comme : pourquoi est-ce que insectes marchent sur l'eau, pourquoi les gouttes et les bulles ont la formes qu'elles ont, ou encore pourquoi en TP de chimie, lorsqu'on met un liquide dans un tube, on voit la formation d'un ménisque ?
  \section{Tension de surface}
  \subsection{Définition}
  \textcolor{purple}{Expérience qualitative films de savon sur des surfaces (cubes) : il faut que le système minimise son énergie de surface.}\\
  Si on augmente la surface d'une interface \textit{dA}, le coût en énergie associé vaut : 
  \begin{equation}
      \delta W = \gamma dA
  \end{equation}
  avec $\gamma$ le coefficient de tension de surface (J.m$^{-2}$). C'est l'énergie qu'il faut fournir pour augmenter la surface d'une interface d'une unité. Ex : pour l'interface eau-air, $\gamma = 72.8$~mJ/m$^2$ à $20°$C. 
  
  \subsection{Origine microscopique}
  Une molécule en volume subit des interactions de cohésion de la part de ses voisines qui la stabilisent. Une molécule à l'interface n'a plus de voisines au dessous, cette configuration augmente son énergie. Le fluide ajuste donc sa forme pour exposer le minimum de surface afin de minimiser son énergie.

  La forme minimale pour une goutte ou une bulle en l'absence d'interaction est une sphère (vidéo bulle d'eau dans l'espace : \url{https://www.youtube.com/watch?v=bKk_7NIKY3Y}). 
  
  \subsection{Force capillaire (7min)}
  Le coefficient de tension de surface peut également être vu comme une force par unité de longueur. \\
  Vidéo force capillaire \url{https://www.youtube.com/watch?v=g4c_tj1CccE}.
  Permet de répondre à pourquoi le gerris peut marcher sur l'eau: les pattes hydrophobes du gerris déforment la surface de l'eau qui va chercher à retrouver sa forme en appliquant une force capillaire sur les pattes. Comme l'insecte est suffisamment léger, la force arrive à compenser le poids. 
  
 \subsection{Pression et tension de surface}
 Question : La pression est-t-elle plus grande dans les petites bulles ou dans les grandes bulles ? \\
 \textcolor{purple}{Expérience avec générateur de bulles : la petite bulle est "mangée" par la grosse.}

CCL : la pression est donc plus grande dans les petites bulles : on peut le démontrer mathématiquement via la \textbf{Loi de Laplace.}
 Pour cela, on applique le principe des travaux virtuels. On imagine que l'on déforme une bulle de $dR$ : \\
 $\delta W = -P_1dV_1-P_2dV_2+\gamma dA =0$ à l'équilibre.\\
 $\ud V_1 = - \ud V_2 = 4 \pi R^2 \ud R$\\
 $dA = d(4\pi R^2) = 8\pi RdR$.\\
 Finalement : 
 \textcolor{red}{Loi de Laplace : }
 \begin{equation}
     (P_1-P_2)=\frac{2\gamma}{R}
 \end{equation}
 \begin{itemize}
     \item Plus $R$ est petit, plus la pression à l'intérieur est grande. 
     \item Lorsque $R \rightarrow \infty$ (surface plane), on a continuité de la pression.
     \item Cette loi nous permet de mesurer $\gamma$ expérimentalement.
 \end{itemize}
 \subsection{Mesure de $\gamma$ (17min)}
 \textcolor{purple}{Expérience :}  Pour une bulle de savon, deux interfaces donc $\Delta P =\frac{4\gamma}{R}$. On génère une bulle, on mesure son rayon et la pression à l'intérieur grâce à un manomètre différentiel de mesurer la différence de pression entre l'intérieur de la bulle et l'extérieur à travers une mesure de tension (piézo). J'ai pris plusieurs points en préparation, je vais en prendre un devant vous. Une regression linéaire $\delta P = \frac{A}{R}$ permet d'obtenir $\gamma$. \\
 \textbf{19min30}\\

La pente de $\Delta P = 4 \gamma\times 1/R$ donne $\gamma=28.8 \pm 2.2$~mN/m$^2$ qui est plus faible que celle de l'eau dû à la présence d'un tensioactif (savon = tensioactif). Par ailleurs, on retrouve le bon ordre de grandeur pour une eau savonneuse (internet donne $25$~mN/m$^2$). \\


  \section{Contact à 3 phases : mouillage (22min)}
  \paragraph*{Mouillage :} étude de l'étalement d'une liquide sur un substrat (solide ou liquide). Utile dans l'industrie (peinture, encre, traitement des pneus, crème, maquillage, etc.).\\
  $\theta =$ angle de contact, permet de savoir si un liquide mouille plus ou moins bien un substrat. Résulte d'une compétition entre les tensions de surface intervenant dans les trois interfaces (L/G, L/S, S/G). \\
  \textbf{26min}\\
  Le système est \{\textbf{dl}$\in$ ligne triple\}. 
  $\textbf{dF}=0$ à l'équilibre. La projection sur l'axe du solide donne la \textcolor{red}{Loi de Young-Dupré} :
  \begin{equation}
      \gamma_{LG}\cos(\theta) = \gamma_{SG}-\gamma_{SL}
  \end{equation}
  
  \section{Influence de la gravité}
  On voit que plus la taille de la goutte est importante, plus la goutte est applatie : il y a un effet de gravité qui n'est plus négligeable à partir d'une certaine taille : laquelle ?
  \subsection{Nombre de Bond (30min)}
  Compétition entre gravité et tension de surface : $B_0=\frac{\rho R^2g}{\gamma}$. La longueur capillaire $l_c$ est telle que: $B_0=\frac{\rho l_c^2g}{\gamma}=1$ (frontière entre les deux régime), soit $l_c = \sqrt{\frac{\gamma}{\rho g}}$.
  Pour l'eau, $l_c=3$mm.\\
  Deux régimes : 
  \begin{itemize}
      \item $R>>l_c$, la gravité domine : goutte plate.
      \item $R<<l_c$, la tension de surface domine, la bulle est sphérique.
  \end{itemize}
  Photo ménisque : résulte de cette compétition entre tension de surface responsable de sa formation et gravité qui s'y oppose. Menisque concave pour un fluide mouillant (ex: eau dans tube en verre) et convexe pour un fluide peu mouillant (ex: mercure dans tube en verre)

  \subsection{Ascension dans un tube (34min)}
  Vidéo de l'ascension d'un liquide dans différents tubes capillaires \url{http://supernovae.in2p3.fr/~llg/Enseignements/Agregation/TP/Fluides-Capillarite/}. Plus le tube est fin, plus le liquide monte haute.
  Schéma liquide dans un tube de rayon $r$, rayon de courbure du ménisque $R$, angle de contact $\theta$. On applique la loi de Laplace : $\Delta P = \frac{2\gamma}{R}=\frac{2\gamma\cos(\theta)}{r}$. En appliquant l'équation de l'hydrostatique, on obtient \textcolor{red}{la Loi de Jurin} :
  \begin{equation}
      h = \frac{2\gamma\cos(\theta)}{\rho g r}
  \end{equation}
  En effet, plus $r$ est petit, plus $h$ est grand. Cette loi permet de mesurer $\gamma$.
  
  \section*{Conclusion (40min)}
  \addcontentsline{toc}{section}{Conclusion}

  Expérience de conclusion : Pince à nourrice dans eau flotte, puis coule avec tensio-actif (savon).


\section*{Description de l'expérience}
  \addcontentsline{toc}{section}{Description de l'expérience}



\begin{tcolorbox}[breakable, enhanced, colback=red!2!white,colframe=mycolor!85!black,title=\textbf{\textbf{Expérience}}]
\paragraph*{Matériel}
\begin{itemize}
\item Générateur de bulle
\item Pied à coulisse
\item Manomètre différentiel + alimentation continue 12V + tuyau
\item Eau savonneuse spéciale (glycérol, savon, eau, etc.).
\item Multimètre (voltmètre) + câble BNC-double banane
\item Chronomètre
\end{itemize}

\paragraph*{Matériel pour manip' qualitatives}
\begin{itemize}
\item Eau savonneuse spéciale (glycérine, eau, teepol)
\item Cadres cubique
\item Pince à nourrice
\item Verre d'eau
\item savon
\end{itemize}

\paragraph*{Protocole } 

\begin{itemize}[label=$\triangleright$]
		\item Connecter le dispositif au manomètre différentiel (on met l'autre voie du manomètre à la pression atmosphérique)
		\item Convertir $U$ en $ex: \Delta P$: $\Delta P = 125*(U - 2.27)$.
		\item  Mesurer la tension en sortie du manomètre et la convertir en $\Delta P$ sur Qtiplot
		\item Mesurer le diamètre avec un pied à coulisse et le convertir en \textbf{rayon} sur Qtiplot
		\item Refaire avec des bulles de taille différentes.
\end{itemize}

\paragraph*{Aspect quantitatif :} Mesure de la tension superficielle d'une eau savonneuse.
\begin{itemize}
\item Tracer $\Delta P$ en fonction de $\frac{1}{R}$
\item Faire un fit (Loi de Laplace). En déduire $\gamma$.
\end{itemize}

\end{tcolorbox}



\newpage


%------------------------------------------



\chapter{Premier principe de la thermodynamique}


\paragraph*{Niveau:} 1ère année CPGE
\paragraph*{Prérequis:} 
\begin{itemize}
\item Système thermodynamique fermé \item Transformation thermodynamique quasi-statique
\item Fonctions d'état
\item Équilibre thermodynamique
\item Énergies cinétique et potentielles, travail d'une force 
\end{itemize}

\paragraph*{Bibliographie:}
\begin{itemize}
\item Physique Tout en 1 MPSI PTSI. Bernard Salamito et al. Dunod.
\item Thermodynamique 1ère année MPSI-PCSI-PTSI. Jean-Marie Brébec. H Prépa (Hachette Supérieur).
\item Bellier 4ème édition: transferts thermiques. p.448  
\item Manip': Quanranta. Dictionnaire de physique expérimentale: Tome II - Thermodynamique et applications. p. 43
\end{itemize}

  \section*{Introduction (3min)}
  \addcontentsline{toc}{section}{Introduction}

 La thermodynamique classique est une branche de la physique qui s'intéresse aux propriétés macroscopiques d'un système, et qui étudie les transformations de la matière et les échanges d'énergie sous différentes formes entre ce système et son environnement sans chercher à comprendre ce qui se passe au niveau microscopique. C'est une théorie axiomatique basée sur principalement sur deux principes.\\
  Animation : \url{https://phet.colorado.edu/sims/html/energy-forms-and-changes/latest/energy-forms-and-changes\_en.html}. Pour un système isolé, l'énergie ne se créé pas et ne disparaît pas, mais elle se transforme d'une forme à une autre : Ex animation : conversion énergie mécanique-électrique, électrique-thermique. \\
  Si le système est isolé, il y a conservation de l'énergie. 
  
  
  \section{Premier principe de la thermodynamique}
  Système ($\Sigma$) fermé.
  
  \subsection{Enoncé} Il existe une fonction d'état extensive \textbf{U} appelée énergie interne, telle que : 
  \begin{equation}
      \Delta E = \Delta E_c + \Delta E_p + \Delta U = W + Q
  \end{equation}
  \textcolor{red}{Interprétation :}
  \begin{itemize}
  \item U : $\sum_{i} E_{c,micro} + E_{p,micro}$
  \item $E_c$ : énergie cinétique macroscopique
  \item $E_p$ : énergie potentielle macroscopique (ex : $E_p=mgz$)
  \item $W$ : travail des forces macroscopiques extérieures non conservatives
  \item $Q$ : transfert thermique (chaleur) dû à l'agitation thermique aléatoire des particules
  \end{itemize}
  $Q$ et $W$ deux modes de transferts d'énergie. Ils sont algébriques (comptés positivement si reçus par le système. \\
  
  \textcolor{red}{Conséquences : } Si système isolé : $\Delta E = 0$ (conservation de l'énergie). \\
  
  \textcolor{red}{Version infinitésimale : } $dE_c + dE_p + dU = \delta W + \delta Q$.
  
  \subsection{Travaux des forces de pression (11min30)}
  Schéma fluide contenu dans une paroi. Pression extérieure $P_e$ uniforme et constante. Système = {fluide + paroi}. Variation du volume total de $dV$ entre entre $t$ et $t+dt$. En $M$ déplacement de $\bm{dM}$. \\
  Force apppliquée au système en un point $M$ : $\mathbf{dF}=-P_e\mathbf{dS}$. Travail asssocié $\delta^2 W = \mathbf{dS}\cdot\mathbf{dM} = -P_e \mathbf{dS}\cdot\mathbf{dM} = - P_e d^2 V$. Travail totale $\delta W$ s'obtient en sommant les travaux : $\delta^2 W$ $\delta W = -P_e dV$\\
  \underline{Si transformation quasi-statique et équilibre mécanique avec l'extérieur $P=P_e$} $\delta W = - p \ud V$ et $W = - \int_A^B p \ud V$ entre deux états $A$ et $B$.
  
 \subsection{Exemples de bilan d'énergie}
 \subsubsection{Trasformation isochore}
 $dV=0$ d'où $W=0$ donc $\Delta U = Q$.
 
 \subsubsection{Transformation monobare et enthalpie}
 $P_e=cste$ donc $W=-P_e(V_f-V_i)$ donc $\Delta U = Q + W$ donc $Q = \Delta U - W = [U_f +P_f V_f]-[U_i+P_iV_i] = H_f - H_i$.\\
 On définit l'enthalpie $H=U+PV$.\\
 
 \textcolor{red}{Premier Principe (transormation monobare):} $\Delta H = Q + W_{autre}$.
 
  \section{Applications du premier principe (20min)}
  \subsection{Définitions préliminaires}
  \textcolor{red}{Capacité thermique à volume constant : } $C_V = (\frac{\partial{U}}{\partial{T}})_V \rightarrow c_V = \frac{C_V}{m}$. \\
  \textcolor{red}{Capacité thermique à pression constante : } $C_P = (\frac{\partial{H}}{\partial{T}})_V \rightarrow c_P = \frac{C_P}{m}$. \\
  
\subsection{Calorimétrie (24min)}
  \textcolor{red}{Expérience} Vase Dewar avec agitateur permettant d'homogénéiser le contenu. On peut remonter à la capacité calorifique.  La pression extérieure est fixée : transformation monobare. On suppose la transformation adiabatique : $\Delta H = Q = 0$.\\
  Pour le système \{eau+calorimètre+fer\} : $\Delta H = \Delta H_{eau} + \Delta H_{calo} + \Delta H_{fer} = c_{eau}\times m_{eau}\times \Delta T_{eau} + c_{eau}\times \mu \times \Delta T_{cal} + c_{fer}\times m_{fer}\times \Delta T_{fer}$.\\
  $\mu =29(3)$ déterminé en préparation. Mesure des masses à la balance : $m_{eau}$ et $m_{fer}$ ; et mesure des températures au thermocouple : $T_{eau+cal}$ et $T_{fer}$ à l'état initial et  $T_f$ à l'état final.\\
  On en déduit : $c_{fer}=\frac{c_{eau}(m_{eau}+\mu)(T_{eau+cal}-T_f)}{m_{fer}(T_f-T_i)}=1016\pm 184$~J.kg$^{-1}$.K$^{-1}$ à comparer à $c_{fer}=449$~J.kg$^{-1}$.K$^{-1}$.\\
  
  \subsection{Détente de Joule - Gay Lussac (38min)}
  Deux enceintes séparées par un robinet. Une enceinte est remplie par un gaz, l'autre par un fluide. Ces enceintes sont calorifugées et avec des parois rigides. On ouvre le robinet. \\
  Système = {gaz+vide+enceintes}. On a $\Delta U=0$. Si $U(T)$ (première loi de Joule) : $\Delta U = C_{V}\Delta T = 0$ donc transformation isotherme. \\
  Cette expérience permet de vérifier si un gaz vérifie la première loi de Joule en mesurant la variation de température.
  
  \section*{Conclusion (40min)}
  \addcontentsline{toc}{section}{Conclusion}

  Dans cette leçon, on a parlé du premier principe qui est un principe de conservation de conservation. Ce principe est complété par le second principe, qui lui, est plutôt un principe d'évolution et qui porte sur le caractère réversible ou irréversible d'une transformation. Pour finir, ces principes et la thermodynamique classique en général a été formalisée plus tard par la mécanique statistique qui permet d'expliquer les résultats de la thermodynamiques en faisant le lien entre l'échelle microscopique et macroscopique.

\section*{Description de l'expérience}
  \addcontentsline{toc}{section}{Description de l'expérience}

[Bellier 4ème édition: transferts thermiques. p.448] ou mieux [Dictionnaire de physique expérimentale: Tome II - Thermodynamique et applications. p. 43]

\begin{tcolorbox}[breakable, enhanced, colback=red!2!white,colframe=mycolor!85!black,title=\textbf{\textbf{Expérience}}]
\paragraph*{Matériel}
\begin{itemize}
\item Vase Dewar
\item Morceau de fer ou autre
\item Thermomètre
\item Bouilloire
\item Balance
\item Chronomètre
\item Bonus: Étuve, potence + fil
\end{itemize}

\paragraph*{Protocole} \textbf{\textcolor{red}{Attention: prendre plutôt le protocole de Quaranta (en bas)}} 

\begin{itemize}[label=$\triangleright$]
		\item \textbf{Prendre $\mu$ sur la notice}. SINON: En préparation: déterminer la masse en eau du calorimètre: méthode 1 Bellier: on mesure $T_{cal,i} = T_{amb}$ et on chauffe de l'eau $T_{eau,i}$. On met l'eau dans le calorimètre et on mesure la température d'équilibre $T_f$. Puis: $m c_{eau} (T_f - T_{eau,i}) + \mu c_{eau} (T_f - T_{cal,i}) = 0$. On en déduit $\mu$. 
		\item Mesurer $m_{fer}$ et $T_{fer,i} = T_{amb}$. \textbf{Mettre le calorimètre sur la balance et tarer}.
		\item Mettre de l'eau à chauffer. Mettre l'eau dans le calorimètre et mesurer $T_{eau+cal,i}$. 
		\item  Mettre dans le fer dans le calorimètre + eau. Fermer vite. Agiter.
\item Mesurer la température finale d'équilibre.
\item Faire le calcul pour en déduire $c_{fer}$.
\end{itemize}

\paragraph{Alternative} [Bellier, Quaranta]
\begin{itemize}
\item Mettre le calorimètre sur la balance. Tarer.
\item Mettre de l'eau à température ambiante dans le vase. \textbf{Mesurer $T_{i,eau+cal}$ et mesurer $m_{eau}$.}
\item Mettre rapidement le fer dans de l'eau bouillante/étuve. Laisser thermaliser. Mesurer $T_{fer,i} = T_{eau/etuve}$. 
\item Puis mettre le fer chaud dans l'eau et suivre la température: elle va monter puis descendre: prendre la température max $T_f$ (cf. courbe p.44 Quaranta). En préparation: Tracer $T(t)$. En leçon, montrer la courbe et prendre deux mesures sur place $T_i$ et $T_f$. 
\end{itemize}

\paragraph*{Aspect quantitatif :} Mesure de  $c_{fer} = \frac{c_{eau} (m_{eau}+\mu) (T_f - T_{i,eau+cal})}{m_{fer} (T_{fer,i}-T_f)}$. 

$\Delta H = \Delta H_{eau} + \Delta H_{cal} + \Delta H_{fer} = 0 = c_{eau}\times (m_{eau} + \mu) \Delta T_{eau+cal} + c_{fer}\times m_{fer}\times \Delta T_{fer}$.


\end{tcolorbox}


\newpage

%------------------------------------------

\newpage

\chapter{Transitions de phase}

\paragraph*{Niveau:} L3

\paragraph*{Prérequis:}
\begin{itemize}
\item Potentiels thermodynamiques
\item Éléments de thermochimie (variance, potentiel chimique, condition d'équilibre, etc.)
\item Variance
\item Milieux magnétiques (aimantation, paramagnétisme et ferromagnétisme).
\end{itemize}

\paragraph*{Bibliographie:}
\begin{itemize}
\item Thermodynamique: fondements et applications. Pérez.
\item Thermodynamique. Diu, Guthmann, Lederer, Roulet (DGLR)
\item Thermodynamics and an introduction to thermostatistics. Callens. J. Wiley \& sons. 1985
\item TD Jules Fillette. Énonce: \url{https://www.lpens.ens.psl.eu/wp-content/uploads/2022/10/TDThermo23_Enonce_TD3.pdf}. Corrigé: \url{https://www.lpens.ens.psl.eu/wp-content/uploads/2022/10/TDThermo23_Correction_TD3.pdf}
\item LP Sylvio Rossetti : \url{https://perso.ens-lyon.fr/sylvio.rossetti/AGREG/LP/LP15_Transition%20de%20phase/Transition_de_phase.pdf}
\item Cours LKB: \url{https://www.lkb.upmc.fr/boseeinsteincondensates/wp-content/uploads/sites/10/2017/10/CoursThermo2017.pdf}
\item Dictionnaire de physique. Taillet.
\item Notice N131 site de Montrouge - Matériel.
\item Bonus: cours changement d'état \url{http://www.phys.ens.fr/~ebrunet/Thermo.pdf}
\item Bonus: Physique statistique: Des processus élémentaires aux phénomènes collectifs. Christophe Texier, Guillaume Roux.
\item Bonus: Physique statistique et thermodynamique: 2e cycle. Claude Coulon, Stéphanie Moreau.
\item Bonus: Physique des transitions de phase -- Concepts et applications. Pierre Papon, Jacques Leblond, Paul Meijer
\item Bonus: Physique statistique: Cours et exercies corrigés. Livre de Christian Ngô et Hélène Ngô. SCIENCES SUP
\end{itemize}

\paragraph{Notes agrégat}
\begin{itemize}
\item 2015 : Il est dommage de réduire cette leçon aux seuls changements d’états solide-liquide-vapeur. La discussion
de la transition liquide-vapeur peut être l’occasion de discuter du point critique et de faire des analogies avec la
transition ferromagnétique-paramagnétique. La notion d’universalité est rarement connue ou comprise.
\item 2014 : Il n’y a pas lieu de limiter cette leçon au cas des changements d’état solide-liquide-vapeur. D’autres
transitions de phase peuvent être discutées.
\end{itemize}

\section*{Introduction}
\addcontentsline{toc}{section}{Introduction}

Montrer sur slide les phases de l'eau. 

[Pérez] Corps pur = système constitué d'une seule espèce. Exemple: eau. \\
Il peut exister sous plusieurs phase = partie homogène caractérisée par les mêmes propriétés physico-chimiques. Exemple: liquide, solide, gaz. \\

\noindent Le passage d'une phase vers une autre est un changement d'état, ou \textbf{transition de phase.} 

\paragraph{Définition} [Taillet] Transformation macroscopique de la structure d'un milieu sous l'effet d'une instabilité, le faisant passer d'une phase à une autre. Cette transformation est identifiée à l'aide d'une grandeur appelée \textbf{paramètre d'ordre}. \\

\textbf{On ne vas s’intéresser ici que de corps pures (pas de diagrammes binaires).}


On va introduire ce sujet en étudiant une transition de phase qu'on rencontre dans notre vie de tous les jours: les changements d'états.

\section{Changement d'état d'un corps pur}

[Tec\&Doc, Sujet 1993]

On va partir d'un système qu'on connaît bien: l'eau. Il a trois états physique: liquide, vapeur, solide.

\subsection{Description qualitative}

\subsubsection{Diagramme d'équilibre (P,T)}

Partir de la variance $v = N + 2 - \varphi = 3-\varphi$. Pour un corps pur en équilibre sous deux phases, il y a une seule variable d'état intensive indépendante. \\ 

Dessiner et décrire le diagramme (faire le cas de l'eau: pente S/L négative (c'est aussi le cas pour le bismuth). ATTENTION: ce n'est pas le cas en général). Parler du point triple et du point critique. \\

\textbf{Application à la métrologie} Jusqu’au 20 mai 2019, le kelvin était défini comme la fraction 1/273,16 de la température thermodynamique du point triple de l'eau (H2O), une variation de température d'1 K étant équivalente à une variation d'1 °C.

\paragraph{Vidéo point critique} \url{https://www.youtube.com/watch?v=-AXJISFdC2E}

\paragraph{Transition} [Diu, p. 300] En regardant juste ce qu’il se passe lorsqu’on augmente la tempréature on aurait envie de dire que
l’eau passe subitement du liquide à la vapeur... ce qui n’arrive pas en réalité : les proportions de chaque phase varie
continument! On a besoin du diagramme de Clapeyron (P,V ).


\subsubsection{Diagramme de Clapeyron (P,V)}

[LP Sylvio Rosseti] Dessiner et discuter Diagramme de Clapeyron.

\paragraph{Expérience} Isothermes du SF6. Si vraiment motivé tracer $P(V)$ pour différentes températures en préparation. Sinon montrer la vidéo: \url{https://www.youtube.com/watch?v=jMfDBOg8ibY}. \\

Monter le dispositif expérimental (s'il y est). Sinon, montrer le schéma de l'expérience. Prendre la notice N131 sur le site de Montrouge \url{https://agreg.phys.ens.fr/collection.php?id=946}. Montrer sur slide le schéma des courbes obtenus expérimentalement (N131, dernière page).


\begin{itemize}
\item Si la température est suffisamment faible, on observe un palier dans le diagramme de Clapeyron : la transition
se fait à pression constante.
\item Quand on est sur le pallier on observe 2 phases (une liquide et une gazeuse)
\item Si la température est trop élevé on n’observe si palier, ni interface
\end{itemize}

\paragraph{La thermo va nous permettre de comprendre ces données empiriques.}


\subsection{Approche thermodynamique}


[Cette partie est fortement inspirée de \href{https://www.lkb.upmc.fr/boseeinsteincondensates/wp-content/uploads/sites/10/2017/10/CoursThermo2017.pdf}{LKB}] La description des transitions de phase peut se faire avec l’approche usuelle
de la thermodynamique. \\

\textbf{Rappeler ce qu'est un potentiel thermo} [Diu, p. 316]


Un potentiel thermodynamique est une fonction d'état particulière qui permet de prédire l'évolution et l'équilibre d'un système thermodynamique, et à partir de laquelle on peut déduire toutes les propriétés (comme les capacités thermiques, le coefficient de dilatation, le coefficient de compressibilité, etc.) du système à l'équilibre. Un potentiel thermodynamique du système thermodynamique décroît lors d'une transformation spontanée et atteint un minimum à l'équilibre. \\

On doit identifier le potentiel thermodynamique adapte à la situation et on en déduit l’état d’equilibre du système (qu’il corresponde à une phase ou l’autre ou à la coexistence des deux) qui correspond à un minimum de ce potentiel. La particularité de l’étude des transitions de phase est que l’on traite des cas où il peut y avoir plusieurs minima locaux différents du potentiel thermodynamique.

\subsubsection{Potentiel thermodynamique et stabilité de l'équilibre}

\paragraph{Interprétation de la courbe (P,T)}

On choisit comme système une petite (mais macroscopique) partie du corps.
Ce système a un nombre de particules fixe et sa pression et sa température sont
fixées par le système global. La fonction à minimiser est donc son enthalpie libre
$G(T,P,N)$. Ici, les variables $T, P, N$ sont les variables externes, fixées
dans ce problème ; alors que $U$ et $V$ sont les variables internes qui sont libres
de s’ajuster pour aboutir à l’état d’équilibre. En effet, l’énergie et le volume de ce système varient en fonction des échanges avec le reste du système. On a
vu que l’énergie va s’ajuster pour donner égalité des températures à l’équilibre.
On va raisonner dans cette partie à une température unique et on va étudier
la dépendance de l’enthalpie libre en fonction du volume $G(V)$ pour diverses
valeurs de la pression.

On trace les profils de l’enthalpie libre libre pour le cas à deux minima pour
différentes valeurs de la pression ($T$ fixée) de part et d'autre d'une transition (utiliser le diagramme P,T).

Lors d’une transition de phase, l’allure typique de la fonction G(V ) est
représentée sur la figure 8.1 [LKB]. Elle se présente sous la forme d’une fonction à
deux minima dont le détail de l’allure varie avec la pression. Chaque minimum
correspond à une phase différente ; une de faible volume et une autre de grand
volume. Ces deux minima sont séparés par un maximum intermédiaire. On a
donc une situation où la dérivée de l’enthalpie libre s’annule en trois points.
Le maximum correspond à un état dit instable. Les deux minima sont des
états d’équilibre stables mais seul le minimum absolu est l’état d’équilibre du
système. L’autre minimum est associé à un état dit métastable. Cela signifie
que le système est stable autour de ce minimum mais que si les fluctuations
thermiques sont suffisamment importantes, elles peuvent faire passer le système
dans l’état d’équilibre.

\textbf{Exemple:} surfusion de l'eau. Vidéo: \url{https://www.youtube.com/watch?v=8ExRopuown0}. 

\subsubsection{Coexistence de phases}

[LKB] Un dernier cas possible est que les deux minima aient la même valeur. Dans
ce cas, quel est l’état d’équilibre ? Il s’avère que cette situation correspond à la coexistence entre deux phases, la phase liquide et la phase vapeur par exemple.

Ceci arrive, à température fixée, pour une pression particulière. Dans le cas de l’équilibre liquide-vapeur, c’est ce que l’on appelle la pression de vapeur saturante.  Le même raisonnement à pression fixée donnerait une température particulière T(P) pour la condition de coexistence. On constate donc que les variables P et T ne sont plus indépendantes.

[Pérez] Sur une courbe (P,T), on a coexistence de deux phases : les potentiels chimiques du corps pur dans les deux phases sont égaux. En plus, la condition d'équilibre est indépendante de $x$: 2 phases coexistent en proportion arbitraire.  On dit que l’équilibre est indifférent. La fraction x peut prendre n’importe qu’elle
valeur fixée par l’extérieur. Ce résultat peut paraître surprenant ; quelle que
soit la quantité dans chaque phase, il peut y avoir équilibre.

\paragraph{Point triple}  

Nous avons traité jusqu’ici l’équilibre de deux phases. Il est possible que trois
phases soient en équilibre. De la même façon que pour l’équilibre à deux phases
on en conclut que les trois phases doivent avoir le même potentiel chimique.
L’égalité (8.1) devient donc un couple d’égalité et on en déduit donc que cet
équilibre arrive pour un seul couple pression/température. On appelle cette
valeur point triple d’un corps. Par exemple, le point triple de l’eau est à une température de 273.16 K et à une pression de 611 Pa. Il a une importance en
métrologie des températures puisque la température est fixée indépendamment
de la pression ce qui rend sa mesure plus robuste. Le degré (Celsius ou Kelvin)
est défini à partir du point triple de l’eau.



\textbf{Transition} On vient de voir la continuité de $G$. Mais ce n'est pas le cas de toutes les grandeurs thermodynamiques.

\subsection{Enthalpie de changement d'état}

Exemple transition solide/liquide. On trace $T(t)$: elle est constante pendant le changement d'état. Durant toute cette durée, on apporte de l'énergie quantifiée par une enthalpie de changement d'état.

\subsubsection{Définition} C'est l'énergie qu’il faut fournir par passer de façon réversible d’une phase à l’autre à la pression de coexistence de phase $P$:
\begin{equation}
L_{AB} = T (S^A - S^B) 
\end{equation}
\textbf{N.B.} $L_{AB} > 0$ car par convention on choisit toujours pour B la phase haute température qui est donc d’entropie la plus grande. On s'intéresse soit à l'enthalpie molaire ou massique. Unité: soit J/kg ou J/mol. Ordre de grandeur: La chaleur latente de fusion de l’eau à pression atmosphérique est d’environ 330 kJ/kg et la chaleur
latente de vaporisation est d’environ 2 200 kJ/kg.

$\delta Q = \ud m \times L$, $\ud m$ masse du corps changeant, $L$ enthalpie de changement d'état. Pour une transfo réversible isobare $\ud H = \delta Q_p = \ud m \times L$.

\subsubsection{Expérience} Mesure de $L_v^{eau}$ dans le cas du changement d'état liquide-vapeur.

\subsubsection{Relation de Clapeyron}

Si le temps, la démontrer en partant de $\mu_A(P,T) = \mu_B(P,T)$ et de la relation de Gibbs-Duhem . Sinon on la balance.
\begin{equation}
\frac{\ud p}{\ud T} = \frac{L_{AB}}{V_m^B - V_m^A}
\end{equation}
$V_m$ volumique molaire selon ce qu'on prend pour $L$. Attention à ces histoire de massique/molaire !

\paragraph{Interprétation du diagramme (P,T)}

$L_v = T (\bar{m}^v - \bar{v}^l) \frac{\ud P}{\ud t}\mid_{L-V}$. Avec $\bar{v}$ volume massique. Pente négative implique dans le cas solide-liquide $v_l < v_s$ donc la glace est moins dense que l'eau: elle flotte.

[Callens] Discuter les conséquences de l'étude de la pente.

Une enthalpie de changement d'état non nulle traduit la discontinuité du volume molaire et de l'entropie molaire lors d'une transition du premier ordre.

\paragraph{Transition} On s'est jusque là intéressé à une interprétation basée sur le potentiel thermodynamique. On peut faire une interprétation en termes des isothermes  (diagramme (P,V).

\subsection{Modèle de Van der Waals}

\textbf{Traiter cette partie semble illusoire.}

\subsubsection{Équation d'état}

[GDGLR, p.330 + TD JF + Callens p. 234] 

Forme intensive.
\begin{equation}
P = \frac{RT}{v - b} - \frac{a}{v^2}
\end{equation}

On trace (P,v) pour différentes températures.

Stabilité: $\frac{\partial P}{\partial v} < 0$. Condition violée sur certains intervalles $\rightarrow$ nécessite d'une transition de phase.

\subsubsection{Construction de Maxwell}

[Diu, p. 330 + TD]

On peut aussi tracer l'allure de $G(P)$ (ou $\mu(P)$).

Suivre le TD pour l'égalité des aires.



\paragraph{Transition} On vient de voir un type de transition de phase caractérisée par une coexistence de phase et par l'existence d'une enthalpie de changement d'état. Ce types de transitions sont dites d'ordre $1$. Est-ce qu'il y a d'autres types ? Oui: on va étudier un exemple.


\section{Introduction aux transitions d'ordre 2}

\subsection{Transition liquide-gaz au point critique}

C’est un point particulier au-delà duquel on ne peut plus faire la différence entre liquide et vapeur. Ceci est lié au fait qu’il n’y a plus de discontinuité du volume. 

[Péréz] Utiliser Clapyeron: $L=0$.


[LKB] On peut comprendre ce point facilement grâce à la figure 8.3. La courbe donnant l’enthalpie libre, qui présente deux minima, se modifie lorsque l’on augmente la température de
telle sorte que les deux minima se rapprochent et fusionnent au point critique, donnant alors une seule phase d’équilibre, appelée phase fluide. En ce point, le volume, la température et la pression sont fixés. Pour l’eau le point critique est à environ 374°C et 220 bars et donc peu facile à observer. Mais il est
aisément observable pour, par exemple, l’hexafluorure de soufre et donne lieu
au phénomène surprenant d’opalescence critique.

\subsection{Transition ferromagnétique-paramagnétique}

\paragraph*{\textcolor{red}{Vidéo introductive:}} \url{https://www.youtube.com/watch?v=03XDF5kzrEs}

\paragraph{Observation} A champ nul, un milieu ferromagnétique possède une
aimantation non nulle pour $T < T_C$ et nulle (en fait, proportionnelle au champ) pour $T>T_c$ . L’idée de cette partie est
de l’expliquer thermodynamiquement.

\subsubsection{Modèle de Landau}

[Bobroff]

Landau propose alors de développer le potentiel thermodynamique approprié en puissance du
paramètre d'ordre près de TC ce qui permet alors de déduire les différentes réponses quand on s’approche de Tc. C'est légitime car ce paramètre d'ordre s'annule à TC. Ce développement permet alors de calculer les comportements analytiques des fonctions de réponse près de $T_c$.
Prenons l'exemple de la transition vers un état ferromagnétique. On note M l'aimantation du corps qui apparaît sous $T_c$, paramètre d'ordre de la transition. On développe alors le potentiel
de Gibbs G selon les puissances de M. De plus, à l'équilibre, toutes les directions de l'aimantation sont possibles donc G est le même pour M et -M, donc une fonction paire de M, d'où on écrit
\begin{equation}
G(M) = G_0 + a(T) M^2 + b(T) M^4
\end{equation}
Donner les conditions de stabilité sur les signe de $a$ et de $b$ (cf. TD). On dérive pour voir les minima.

\textbf{N.B.} Selon le temps, on peut donner directement $a(T) = A \times (T-T_c)$ et $b(T) = \frac{B}{4}$ avec $A>0$ et $B>0$.

\subsubsection{Interprétation graphique}

[\href{http://hebergement.u-psud.fr/rmn/thermo/thermo_pdf/coursThermo2013.pdf}{Bobroff} et \href{https://perso.ens-lyon.fr/sylvio.rossetti/AGREG/LP/LP15_Transition%20de%20phase/Transition_de_phase.pdf}{Sylvio}] 

On trace $M(T)$ : On constate que l’aimantation est continue à la transition. \\
On trace $G(M)$ pour plusieurs températures. On passe de deux minima à un seul. Faire le lien avec le point critique.

Lire sur les limites de Landau [Bobroff].

\section*{Conclusion}
\addcontentsline{toc}{section}{Conclusion}

\paragraph{Classification}

Plusieurs classifications ont été proposées [ne pas rentrer dans les détails]
\begin{itemize}
\item Ehrenfest : L’ordre d’une transition de phase est défini comme suit : c’est l’ordre le plus bas de la dérivée du potentiel thermodynamique G présentant une discontinuité lors d’une transition. A été abandonnée car ne permet pas de rendre compte  des transitions pour laquelle la singularité du potentiel thermodynamique est plus complexe qu’un discontinuité de dérivée.
\item Landau : basée sur la brisure de symétrique. On passe d'une phase symétrique (ex: para) à une phase pas ou moins symétrique (ex: ferro). Landau utilise alors la notion de paramètre d’ordre, grandeur nulle dans la phase la plus symétrique, et différente de 0 sinon. 
\end{itemize}

Ici, on va présenté la \textbf{classification moderne}. On distingue deux classes de transitions de phase: \textcolor{mycolor3}{Faire petits schéma paramètre d'ordre vs paramètre pour chacune des deux classes de transitions (discontinuité vs pas de discontinuité).}


\begin{center}
   \begin{tabularx}{18cm}{| c | X | X |}
     \hline
      & Premier ordre & Second ordre  \\ \hline
     Caractéristiques & Ne possède pas de paramètre d'ordre ou qui possède un paramètre d'ordre discontinu à la transition de phase. Ces transitions s'accompagnent d'une énergie latente de transition et permettent la coexistence des phases. C'est le cas des changements d'état.  & Toutes les autres transitions sont du second ordre, elles possèdent un paramètre d'ordre continu à la transition de phase. Elles ne font pas intervenir d'énergie latente de transition et ne permettent pas la coexistence des phase. \\ \hline
     Exemples & Dans la transition eau-vapeur, en faisant varier la température en dessous ou au-dessus de la température critique, le système se trouve dans deux états différents (gaz et liquide). Le paramètre d'ordre est la \textbf{densité} ($\rho^{100 °C}_{\mathrm{liq}} = 0.958~\unit{kg.m^{-3}} > \rho^{100 °C}_{\mathrm{vap}} = 0.597~\unit{kg.m^{-3}}$). &  ferromagnétique/paramagnétique (aimantation).  La transition conducteur/supraconducteur (Amplitude de la paire d'électrons). Normal-superfluide (amplitude quantique de la fonction d'onde). Transitions
allotropiques entre deux types de cristaux d’un solide (graphite-diamant).  \\
     \hline
   \end{tabularx}
 \end{center}

Conclure sur d'autres exemples de transitions de phase [Slide] (et nematique-smectique dans les cristaux liquides?)


\textbf{N.B.} Lors d’une transition de phase du deuxième ordre, au voisinage du point critique, les systèmes physiques ont des comportements universels en lois de puissances caractérisés par des exposants, dits critiques. 

[Wiki] It is a remarkable fact that phase transitions arising in different systems often possess the same set of critical exponents. This phenomenon is known as universality. For example, the critical exponents at the liquid–gas critical point have been found to be independent of the chemical composition of the fluid. Universality is a prediction of the renormalization group theory of phase transitions, which states that the thermodynamic properties of a system near a phase transition depend only on a small number of features, such as dimensionality and symmetry, and are insensitive to the underlying microscopic properties of the system.

Lire Sylvio pour les questions.


\section*{Description de l'expérience}
  \addcontentsline{toc}{section}{Description de l'expérience}

[TP Transitions de phase]

\begin{tcolorbox}[breakable, enhanced, colback=red!2!white,colframe=mycolor!85!black,title=\textbf{\textbf{Expérience}}]

\paragraph*{Matériel} 
\begin{itemize}
\item Eventuellement: Appareil Isothermes du SF6
\item Vase Dewar, eau
\item Balance de précision, niveau à bulle
\item Résistance (thermoplongeur)
\item Voltmètre
\item Ampèremètre
\item Pince, potence (x2)
\item Alternostat 0-240V
\item Chronomètre
\item Thermomètre
\end{itemize}


\paragraph*{Protocole}
\begin{itemize}
\item Brancher avec des fils la prise du thermoplongeur vers l'alternostat.
\item Mettre l'ampèremètre en série et le voltmètre en parallèle.
\item Tracer $-m(t)$. 
\end{itemize} 

\paragraph*{Mesure}

$UI = - \frac{\ud m}{\ud t} L_v$. Hyp : $\frac{\ud m}{\ud t}\mid_{\text{sans chauffage}} = 0$. 

Ajustement linéaire: $L_v$ relié à la pente.

\end{tcolorbox}




\newpage

%---------------------------------------

\chapter{Phénomènes de transport}


\paragraph*{Niveau:} PC
\paragraph*{Prérequis:} 
\begin{itemize}
\item Thermodynamique à l'équilibre
\item Notion de flux
\item Conduction électrique
\end{itemize}

\paragraph*{Bibliographie:}
\begin{itemize}
\item Thermodynamique. DGLR. Hellman
\item J.P. Pérez et A.M. Romulus, Thermodynamique. Fondements et applications, Masson, Paris, 1993, page 52.
\item Tout en un PC, Sanz et al. DUNOD
\item Tec\&Doc Phyique MP. Gié et al.
\item Tec\&Doc Phyique PC.  Gié et al.
\item Physique PC. Pascal Olive.
\end{itemize}

\paragraph{Notes agrégat}
\begin{itemize}
\item 2017 : La leçon ne peut se limiter à la présentation d’un unique phénomène de transport.
\item 2016 : Les analogies et différences entre les phénomènes de transport doivent être soulignées
tout en évitant de dresser un simple catalogue.
\item 2015 : Les liens et les limites des analogies entre divers domaines doivent être connus.
\item 2013 : [À propos du nouveau titre] Le candidat développera sa leçon à partir d’un exemple
de son choix.
\end{itemize}


\section*{Introduction}
  \addcontentsline{toc}{section}{Introduction}

[JF] On a précédemment étudié la thermodynamique des transformation
entre état d’équilibre sans jamais se poser la question du chemin suivi (c’était d’ailleurs le grand intérêt des fonctions
d’état !). Dans cette leçon, on va lever l’hypothèse d’équilibre thermodynamique et s’intéresser aux mécanismes de
transport des grandeurs usuelles (matière, énergie, charge, etc.).



%\paragraph{Sur slide: projeter des images de transports par convection et rayonnement}

\textbf{N.B} L'entité transférée est transportée par les porteurs de charge. Selon la nature de ces porteurs et les conditions dans lesquelles s'effectue le transfert, on peut distinguer la diffusion (transfert par diffusion), la convection (transfert par convection) et le rayonnement (transfert par rayonnement). 


\section{Introduction aux processus irréversibles}

\subsection{Position du problème}

[DGLR] Slide: deux corps à des T différentes. La thermo nous donne des infos sur les états d'équilibre initiaux et finaux, sens de l'échange, mais rien entre les deux.

[DGLR] Définition d'un phénomène de transport (slide): Un phénomène de transfert (ou phénomène de transport) est un phénomène irréversible durant lequel une grandeur physique est transportée d'un endroit vers un autre sans être créées ni perdues en chemin.

Les entités transférées les plus connues sont l'énergie (transfert thermique), la matière (transfert de masse) et la quantité de mouvement (transfert de quantité de mouvement).


\paragraph{Problématique} la thermodynamique n'est pas en mesure d'analyser le transport de ces grandeurs (phénomène hors équilibre). Pire, les grandeurs ne sont pas définies hors équilibre. Comment allons-nous nous en sortir ? C'est ce qu'on va voir dans la suite.


\subsection{Équilibre thermodynamique local}

[JF] Définir les trois échelles et comment arriver à l'équilibre local par découpage en sous volumes mésoscopiques. Donner des exemples d'échelles. Limite de l'hypothèse: déséquilibre pas trop fort ne permettant plus l'approximation linéaire.


[Tec\&Doc, p. 660] Définition échelle mésoscopique, ODG, équilibre local. 

\textbf{N.B.} Pour les questions [DGLR, p. 463] deux échelles de temps: relaxation et évolution. Eq. local si $\tau_{rel} \ll \tau_{ev}$.

[DGLR, p. 465] Autre condition nécessaire à notre étude est l'approximation linéaire (proche équilibre).

\subsection{Grandeurs transportées}

\begin{itemize}
\item la viscosité (transport de quantité de mouvement) : transfert de quantité de mouvement dû à une inhomogénéité de vitesse ;
\item la diffusion moléculaire (transport de particules) : transfert du nombre de particule ou transfert de masse dû à une inhomogénéité de densité particulaire ;
\item la diffusion thermique (transport de matière) : transfert thermique (ou chaleur) si la température n'est pas uniforme ;
\item la conductivité électrique (transport de charges) : transfert de la charge électrique en cas de différence de potentiel.
\end{itemize}

[JF] Nous avons établi les conditions générales de l’étude des phénomènes de transport. Nous pouvons donc entamer l’étude successive de deux grands types de transport au programme de PC : le transport de matière et le transport d’énergie sous forme de transfert thermique.

\section{Diffusion thermique}

[Tec\&Doc, Sanz]

La diffusion est le 3ème mode de transfert thermique à côté de la convection et du rayonnement.

\subsection{Modes de transferts thermiques}

Définir rapidement chaque mode + images/vidéos. Ici, on va s'intéresser à uniquement à la diffusion.


\paragraph{Convection} \url{https://www.youtube.com/watch?v=v62ilJCaMFk}

\paragraph{Diffusion} \url{https://www.youtube.com/watch?v=6byqNP3Tif0} 

\subsection{Bilan d'énergie}

Vecteur densité de flux thermique (déf. et unité). Premier principe.

\subsection{Loi de Fourier}

Hypothèses. Loi. Sens physique. Limitations. Conductivité thermique. Interprétation microscopique de la loi de Fourier.

Revenir sur la vidéo.

\subsubsection{Expérience} Mesure de la conductivité thermique du cuivre. Hypothèses (pas de pertes, RP atteint, etc.). Faire les calculs avec l'équation locale + Fourier. 

\subsection{Équation de diffusion}

Établir. Dire que cette équation est générale à tous les phénomènes diffusifs. Commentaires physiques [DGLR, p. 479].

\subsubsection{Interprétation microscopique}

Agitation thermique \url{https://www.youtube.com/watch?v=qW59Y9lJso8}

\subsection{Régime stationnaire [facultatif]}

A faire ou pas selon le temps. Résistance thermique. Application: double vitrage.

\subsection{Analogie avec la conduction électrique}

Mettre le tableau de Tec\&Doc sur slide. Voir aussi [Olive, p. 588].


\section{Diffusion de particules}

[Sanz]

Intro : \url{https://www.youtube.com/watch?v=Mqi4KgR6PzE}

\subsection{Bilan de particules}


\subsection{Loi de Fick}

Loi. Signification physique. Coefficient de diffusion, unité et ordres
de grandeurs. Limites de validité.

\subsection{Équation de diffusion}

Commentaires physiques [DGLR, p. 492].


\subsection{Approche microscopique}

Marche aléatoire 1d. Interprétation microscopique de la loi de Fick [Olive, p. 561].



\section*{Conclusion}
  \addcontentsline{toc}{section}{Conclusion}

On a vu plusieurs exemple mais c'est le même phénomène. Même équation $\frac{\partial f(M,t)}{\partial t} = D \Delta f(M,t)$. 

$D$ en m$^2$ s$^{-1}$. Irréversibilité.

Le moteur microscopique commun à tous les phénomènes de diffusion est l'agitation
aléatoire : à une température fixée, chaque molécule possède une énergie cinétique et peut subir des chocs avec les molécules voisines. Plus l'agitation aléatoire est importante, plus les chocs sont nombreux et les transports diffusifs efficaces (diffusion de particules, de la chaleur, de la quantité de mouvement).

\textbf{N.B.} [Wiki] In physics, transport phenomena are all irreversible processes of statistical nature stemming from the random continuous motion of molecules, mostly observed in fluids. Diffusion is a stochastic process due to the inherent randomness of the diffusing entity and can be used to model many real-life stochastic scenarios. There are two ways to introduce the notion of diffusion: either a phenomenological approach starting with Fick's laws of diffusion and their mathematical consequences, or a physical and atomistic one, by considering the random walk of the diffusing particles.

Ouvrir sur la théorie d'Onsager [DGLR, p.508].


[JF] Possibilité d’ouvrir sur l’effet Seebeck qui fait le lien entre différents modes de transport dans les
métaux, ou la loi de Wiedmann-Franz. On peut aussi ouvrir sur le transport de quantité de mouvement et le nombre
de Reynolds qui permet de quantifier l’importance relative de la convection et de la diffusion.


\section*{Description de l'expérience}
  \addcontentsline{toc}{section}{Description de l'expérience}

TP Thermométrie.

\begin{tcolorbox}[breakable, enhanced, colback=red!2!white,colframe=mycolor!85!black,title=\textbf{\textbf{Expérience}}]
\paragraph*{Matériel}
\begin{itemize}
\item 2 thermocouples + 2 boitiers thermocouple de type K + 1 lecteur de thermocouple (+ les câbles thermocouples)
\item Alimentation continue
\item Montage contenant un barreau de cuivre chauffée à une extrémité par une résistance
\item 2 tuyaux pour l'eau (pour refroidir l'autre extrémité)
\item Multimètre (ohmmètre puis voltmètre)
\item Fils
\item Chronomètre
\end{itemize}

\paragraph*{Protocole } 

\begin{itemize}[label=$\triangleright$]
		\item On mesure au préalable la résistance
		\item On fait les branchements  et on attend l'établissement du régime permanent.
		\item On mesure le gradient de température $A = \frac{T(L) - T(0)}{L}$.
		\item On suppose que la puissance électrique fournie $P_e = \frac{U^2}{R}$  contribue entièrement au flux thermique à travers le barreau.
		\item $j = \lambda \times A = \frac{P_e}{S}$. D'où $\lambda = \frac{P_e}{S A}$. 
		\item Comparer à la valeur tabulée $\lambda = 401$ W m$^{-1}$ K$^{-1}$.
\end{itemize}

\paragraph*{Aspect quantitatif :} Mesure de la conductivité thermique du cuivre.


\end{tcolorbox}





\newpage


%------------------------------------------


\chapter{Conversion de puissance électromécanique}

\paragraph*{Niveau:} PSI
\paragraph*{Prérequis:} 
\begin{itemize}
\item Force de Laplace
\item Induction, champ électromoteur
\item Electrocinétique de base
\item Mécanique de base
\end{itemize}

\paragraph*{Bibliographie:}
\begin{itemize}
\item Physique PSI-PSI* Tout-en-un. Dunod.
\item Électronique II. H-Prépa 2ème année PSI-PSI*.
\item Physique SP\'E PSI*-PSI. Tec\&Doc. Olivier, More \& Gié.
\item Électronique -- Conversion de Puissance. PSI-PSI*. ellipses. Taupe-niveau. Meiler, Irlinger \& Kempf.
\item \url{https://www.electronique-mixte.fr/wp-content/uploads/2018/07/Formation-Machines-Electriques-cours-1.pdf}
\item Cours Jérémy : \url{https://jeremy.neveu.pages.in2p3.fr/Moteurs/induction.html}
\item Animations: \url{https://sitelec.org/animations2.htm}
\end{itemize}

\paragraph{Notes agrégat}
\begin{itemize}
\item 2017 : Une approche à l’aide des seules forces de Laplace est insuffisante. Les candidats doivent aussi s’interroger sur l’intérêt d’utiliser des matériaux ferromagnétiques dans les machines électriques.
\item 2016 : Afin de pouvoir aborder des machines électriques de forte puissance, le rôle essentiel du fer doit être considéré car les forces électromagnétiques ne se réduisent pas aux seules actions de Laplace s’exerçant sur les conducteurs traversés par des courants.
\item 2015 : Il est souhaitable de préciser le rôle de l’énergie magnétique lors de l’étude des convertisseurs
électromécaniques constitués de matériaux ferromagnétiques linéaires non saturés.
\item 2014 : Dans le cas des machines électriques, les candidats sont invités à réfléchir au rôle
du fer dans les actions électromagnétiques qui peuvent également être déterminées par
dérivation d’une grandeur énergétique par rapport à un paramètre de position.
\end{itemize}

\section*{Introduction}
\addcontentsline{toc}{section}{Introduction}

[Tec\&Doc, p. 585]

La conversion de puissance est nécessaire tout au long de la \textit{production} (conversion), la \textit{transmission} et l'\textit{utilisation} de l'énergie électrique. Chaque étape met en jeu respectivement
\begin{itemize}
\item la conversion électromécanique (ex: alternateur, moteurs).
\item la conversion électromagnétique (transformateur).
\item la conversion électronique (ex: hacheur, onduleur, redresseur).
\end{itemize}

Montrer des images de la vidéos de tous les jours.

[Sylvio] Cette leçon traite de la conversion d’énergie électrique en mécanique. Concrètement cela parle des moteurs. Pourquoi faire cette leçon ? Parce que les moteurs constitue un outil essentielle dans notre vie quotidienne (électroménager), mais également à plus grande échelle (trains, machines -outils). Rien que dans une voiture on entre 10 et 100 moteurs (ventilation, essuie-glace déplacement des vitres).

L'étude de l'électromagnétisme a permis de mettre en évidence un couplage entre la mécanique et l'électricité via les forces de Laplace d'une part, et l'induction d'autre part. On va traiter ici la conversion de puissance électromécanique.

\section{Principe de la conversion électromécanique}


Principe basé sur:
\begin{itemize}
\item La force de Laplace : énergie électrique $\rightarrow$ énergie mécanique.
\item L'induction : énergie mécanique $\rightarrow$ énergie électrique.
\end{itemize}

\subsection{Bilan de puissance}

[Tec\&Doc, p. 596]

On s'intéresse à des porteur de charges dans un élément de volume $\ud \tau$.

Bilan sur porteurs de charge : $P_e + P_L = 0$. Équivalence entre puissance électrique et puissance mécanique pour un transducteur parfait. 

\subsection{Réversibilité} Un convertisseur est réversible si

Puissance électrique  $\xrightleftharpoons[\mathrm{générateur}]{\mathrm{moteur}}$ Puissance mécanique. 

\subsection{Dans un transducteur réel}

\subsubsection{Utilisation de matériaux ferromagnétiques}

\paragraph{Jury 2016} Afin de pouvoir aborder des machines électriques de forte puissance, le rôle essentiel du fer doit être considéré, car les forces électromagnétiques ne se réduisent pas
aux seules actions de Laplace s’exerçant sur les conducteurs traversés par des courants.

En pratique, on utilise des matériaux ferromagnétiques afin de canaliser les lignes de champ magnétique. De l'énergie est alors emmagasinée dans le fer. Dans le cas du relais, lorsque la bobine est parcourue par un courant, le bloc subit une force attractive, quel que soit le sens du courant: \textbf{il faut la prendre en compte, en plus des forces de Laplace}.

\paragraph{Exemple:} Le contacteur [Olive, p. 859. Dunod PSI, p. 733].

Soit faire les calculs, soit en parler de manière descriptive, (soit ne pas en parler mais l'avoir en tête).

[JF] Présentation du fonctionnement. Un
noyau ferromagnétique \textbf{linéaire} fixe en forme de U est excité par une bobine de N spires parcourue par un courant d’intensité i (électro-aimant). Deux entrefers d’épaisseur x variable le séparent d’un bloc ferromagnétique mobile en translation
selon Ox. On note S la section constante du circuit magnétique et l sa longueur moyenne en l’absence de l’entrefer. La bobine est constituée de N spires et parcourue par un courant i(t).

[JF] Lorsque l’intensité i est nulle dans la bobine, la partie mobile ne subit
aucune force. En présence d’un courant, la partie fixe se comporte comme un électroaimant, elle attire la partie mobile
avec une certaine force qu’on cherche à déterminer. Établissement du champ B et de l’inductance propre par
théorème d’Ampère et conservation du flux de champ magnétique.

Donner l'expression de la force obtenue en dérivant l'énergie magnétique. Faire le calcul d'ordre de grandeur: $F_{max} = - 1.1 kN$: permet de soulever jusqu'à 115 kg !

\textbf{A savoir} [Olive] Le ferro subit des contraintes mécaniques (forces de Laplace dues aux courants de Foucault): le matériau se dilate et se contracte à 2 fois la fréquence de B (et des courants induits), générant une onde sonore. 


\subsubsection{Pertes}

\href{https://fr.wikipedia.org/wiki/Transformateur_%C3%A9lectrique#Les_pertes_de_puissance_d'un_transformateur}{wiki}

[Hprépa, p. 84] Jusque là, on a supposé qu'on avait pas de pertes, ce qui n'est évidemment pas le cas. On peut recenser:

\begin{itemize}
\item Pertes cuivre (par effet Joule).
\item Pertes mécaniques par frottement.
\item Pertes fer (puissances dissipées dans le matériau ferromagnétique dues
d'une part au phénomène d'Hysteresie magnétique, d'autre part à la circulation des courants de Foucault).
\item [Wiki] Fuite de flux: Le circuit magnétique est considéré dans le modèle du transformateur idéal comme sans perte, ce qui serait le cas si la résistance magnétique du fer était nulle. Or ce n'est pas le cas, le flux circule donc partiellement à l'extérieur du noyau, ce flux appelé « de fuite », par opposition au flux « principal », est modélisable par une inductance en série avec la résistance de chaque enroulement.
\end{itemize}

\textcolor{red}{Nous verrons deux exemples dans convertisseurs réversibles dans la suite.}


\section{Machine à courant continu}

\subsection{Définition}

[Hprépa, p. 76]

Convertisseur électromécanique permettant la conversion bidirectionnelle d'énergie entre une installation électrique parcourue par un \textbf{courant continu} et un dispositif mécanique.

\subsection{Structure}

Montrer un schéma \url{https://jeremy.neveu.pages.in2p3.fr/Moteurs/moteurs.html#structure-2}. 

[Hprépa, p. 76. Kempf, p. 324]


\paragraph*{Inducteur (stator)} Créé le champ magnétique. C'est soit un aimant permanent (faible puissance), soit un électroaimant (grande puissace, des bobines créent le champ, c'est ce qu'on va décrire ici). Constitué de

\begin{itemize}
\item Culasse en acier (ferromagnétique dur), supporte toutes les parties fixes et ferme le circuit magnétique.

\item Nombre paire de pôles principaux, excroissance de la culasse.

\item Bobines montées en série, placées sur les noyaux des pôles principaux alimentée en courant continu.

\end{itemize}
 

\paragraph*{Induit (rotor)} C'est à ce niveau que sont créés les courants induits. L'armature mobile est appelé rotor. Formé de:

\begin{itemize}
\item Pièce métallique, siège de courants de Foucault. Permet d'assurer la continuité des lignes de champ magnétique créé par les pôles principaux. Formée de tôles fines isolées les unes des autres et collés (pour limiter les courants de Foucault), et faites dans un ferro doux pour diminuer les pertes par hystérésis. 

\item Bobinage: deux fils conducteurs placés de part et d'autre du rotor forment une spire. Le rotor est alimenté en courant continu.
\end{itemize}

\textbf{N.B. Ce qu'il faut retenir:} Pour obtenir des champs forts, le stator est réalisé avec des ferro. L'entrefer entre induit et inducteur doit être étroit pour limiter les pertes de flux. Si deux pôles (nord et sud) machine bipolaire. On peut avoir machines multipolaires (2p, p>2, p pôles nords alternés avec p pôles sud). Les  machines de faible puissance ne possèdent pas de circuit inducteur. Le champ est créé par des aimants permanents.

\paragraph{A savoir} Un électro-aimant produit un champ magnétique lorsqu'il est alimenté par un courant électrique : il convertit de l’énergie électrique en énergie magnétique. Il est constitué d’un bobinage et d’une pièce polaire en matériau ferromagnétique doux appelé cœur magnétique qui canalise les lignes de champ magnétique.

\subsection{Champ magnétique}

[Kempf, p. 325] L'entrefer est la région où sont logés les conducteurs du rotor (induit). La forme des pièces est telle que  B est radial dans l'entrefer. Montrer les lignes de champ (pôle nord, traversent l'entrefer, rotor où elles sont canalisées, à nouveau l'entrefer, pôle sud, et retournent par le stator au pôle nord).

[Dunod PSI, p 808] B permanent, plan de symétrie, résolution numérique.

\subsection{Principe de fonctionnement}

Montrer la vidéo \url{https://www.youtube.com/watch?v=A3b3Km5KVXs&t=42s}

\begin{itemize}
\item Machine simplifiée (bipolaire).
\item Champ magnétique créé par l'inducteur $B(R,\theta) u_r$ (faire le schéma de [Tec\&Doc psi, p. 600].
\item Rotor = une spire.
\item Le rotor est le siège d'induction de Lorentz  (circuit mobile dans un champ stationnaire). 
\item Lorsque le rotor tourne, il apparaît une f.e.m. S'il est parcouru par un courant, le conducteur subit alors une force de Laplace: $I \ud l \wedge B$. On a une force qui fait tourner le rotor.
\item Couple moteur et fem $e = - \phi \omega$, $C = \phi i$.
\end{itemize}


\subsubsection{Rôle du collecteur}

[Kempf, p. 328. Dunod PSI, p.810]

Animation : \url{https://sitelec.org/applets/walter_fendt/electricmotor_f/electricmotor_fr.html}

On change le sens du courant dans le rotor de sorte que la force de Laplace soit toujours dans le même sens.

\paragraph{Transition} Mise en équation.

[Tec\&Doc PSI, p. 600] Schéma spire.

\subsubsection{Aspects mécaniques}

Forces de Laplace sur chaque portion de la spire. Vitesse de rotation $v = r \omega u_\theta$. $d$ Diamètre de la spire. Moment de ces forces: $\Gamma = \frac{d}{2} u_r \wedge F + (-\frac{d}{2}) u_r \wedge F = d i L B u_z$, soit sa valeur algébrique $C =  i d L B = M B$. Puissance mécanique $P_m 2F \cdot v = C \omega$.

\textbf{N.B.} Un couple est un ensemble de forces de résultante nulle, dont en revanche le moment total n'est généralement pas nul. Il est donné par $\Gamma = r \wedge F$, où $r$ est le bras de levier, c'est-à-dire le vecteur de la distance entre le point de rotation (ou l'axe) et la ligne d'action de la force.

\subsubsection{Aspects électriques}

\textbf{IMPORTANT: faire attention aux conventions !}

Champ électromoteur. fem $e = - d L B \omega$. Schéma électrique équivalent. Puissance électrique $P_e = -e i = P_m$.

\textbf{N.B.}
\begin{itemize}
\item En convention récepteur, $P = Ui$ doit être interprétée comme la puissance \textbf{consommée} par le dipôle. Une puissance positive est alors physiquement consommée, alors qu’une puissance négative est physiquement produite.
\item En convention générateur, $P = Ui$ doit être interprétée comme la puissance \textbf{produite} par le dipôle. Une puissance positive est alors physiquement produite, alors qu’une puissance négative est physiquement consommée.
\end{itemize}

\subsubsection{Machine réelle}

[Tec\&Doc, p. 602. H-prépa, p. 81] Plusieurs spires. $e \propto - \phi \omega$. Couple $C \propto \phi i$. 

\subsubsection{Modes de fonctionnement}

\paragraph{Fonctionnement en générateur} pour fournir de l’énergie électrique, de l’énergie mécanique doit être transformée en énergie électrique $ei > 0$ (convention générateur).

\paragraph{Fonctionnement en moteur} Le système consomme de la puissance électrique: $e i < 0$.

\subsection{Etude expérimentale en fonctionnement moteur}

\subsubsection{Relation entre la tension et la vitesse de rotation}

Un grand intérêt du moteur à courant continu est de pouvoir commander la vitesse de rotation via la tension à ses bornes $U$ car la dépendance entre les deux est affine: $u = R i + \phi \omega$. 

\paragraph{Manip':} [TP moteurs]

On propose de vérifier dans un premier temps cette relation tension-vitesse à masse fixée en faisant varier la tension d'alimentation dans un domaine raisonnable : on ne dépassera jamais 12 V, et on n'imposera pas non plus des tensions trop faibles (à juger selon la charge imposée, ne pas aller en dessous de 8 V pour les grandes charges). Vérifier également que le courant ne prend pas des valeurs trop importantes.

\textbf{On trace $U(\omega)$ avec $\omega = \frac{v}{r}$.}

\subsubsection{Rendement et bilan de puissance}

Cela semble déraisonnable de faire cette partie avec l'expérience complète associée. Mais si jamais, voir [Dunod psi, p. 818]. $\eta = \frac{P_u}{P_e} = \frac{C_u \omega}{ui}$. 

\paragraph{Manip'} Mesure du rendement (à la puissance nominale).

Faire la manip' avec seulement une unique mesure ($v$ et $u$) puis en déduire le rendement.

\paragraph{A savoir} le point de fonctionnement se trouve à l'intersection de son couple électromécanique $C(\omega)$ et de son couple mécanique $C_r(\omega)$.

\subsection{Applications}

Jouets, essuie-glaces, des ventilateurs, des machine-outils, dans certaines lignes de métro, RER et TGV.

\subsection{Avantages et inconvénients}

\noindent (+) : Capacité de variation de vitesse. \\
(-) : [Tec\&Doc, p. 608]

\paragraph{Transition} On a vu les MCC qui se basent sur l'induction de Lorentz. On va voir un autre type qui utilise l'induction de Von Neumann.

\section{Machines à courant alternatif (synchone)}

Ça va dépendre du temps, au pire la faire vite fait en conclusion...

\subsection{Champ magnétique tournant (triphasé)}

\url{https://sitelec.org/flash/champ_magnetique_tournant.htm}

En utilisant l'animation (appuyer à chaque fois sur (I)), on illustre les notions de [Tec\&Doc psi, p. 608].

La machine synchrone est basée sur ce principe
\begin{itemize}
\item Stator: créé le champ tournant.
\item Rotor: aimant ou électroaimant (alimenté par un courant continu).
\end{itemize}

\subsection{Fonctionnement en moteur}

[Hprépa, p.106. Kempf, p. 334, Tec\&Doc, p. 611] Calculer l'expression du couple $C(\theta)$. Discuter le cas moteur et résistif.  

\subsubsection{Démarrage} [Kempf, p. 334, Tec\&Doc, p. 613].

\subsection{Fonctionnement en générateur (alternateur)}

[Kempf, p. 334. Tec\&Doc, p. 610] Le rotor tourne à l'aide d'un dispositif annexe, induisant dans les bobines du stator une fem alternative.

\subsection{Fonctionnement en générateur}

\subsection{Avantages et inconvénients}

+ : facilement commandable en vitesse par la vitesse de rotation du champ statorique car les deux vitesses sont égales. \newline
 
-: son couple au démarrage est de valeur moyenne nulle (car le rotor n'a pas encore accroché le champ statorique et sans couple il ne peut commencer à vaincre le couple de charge).

\subsection{Applications}

Production d'énergie électrique dans les centrales de grande puissance.

Moteurs synchrones autopilotés: TGV Atlantique, propulsion de gros navires, malaxeurs de l'industrie chimique.

\section*{Conclusion}
\addcontentsline{toc}{section}{Conclusion}


Avantages/inconvénients MCC vs MCA. Lire [Hprépa, p. 111 et p. 113].

Ouverture sur le moteur asynchrone [Hprépa]: \\
+: n'a pas, a priori, de problème de démarrage. \\
On les trouve dans toutes les applications industrielles et domestiques correspondantes : machines outils, congélateurs, machines à laver, pompes diverses, etc. \\



Ouverture sur d'autres types de moteurs.



\section*{Description de l'expérience}
  \addcontentsline{toc}{section}{Description de l'expérience}



\begin{tcolorbox}[breakable, enhanced, colback=red!2!white,colframe=mycolor!85!black,title=\textbf{\textbf{Expérience}}]
\paragraph*{Matériel}
\begin{itemize}
\item Chronomètre
\item Masses
\item MCC
\item Alimentation variable ($12V, 0.5 A$)
\item Grande règle
\item Fils
\item Ampèremètre
\item Voltmètre
\item Interrupteur inverseur
\item Serre-joint
\item Mousse
\item Écran
\item Pied à coulisse
\end{itemize}

\paragraph*{Protocole } 

\begin{itemize}[label=$\triangleright$]
		\item On laisse suffisamment d'espace au sol pour que la masse atteigne son régime stationnaire
		\item On mesure le temps entre deux points de l'écran de distance connue (par exemple $30cm$) en montée
		\item Faire la manip' à la tension nominale $U = 12V$		
\end{itemize}

\paragraph*{Aspect quantitatif :} Tracer à $U$ fixée pour différentes masses $\eta(P_u)$ avec $\eta = \frac{P_u}{P_e}$ où $P_e = U I$ est la puissance électrique fournie et $P_u = \Gamma_u \Omega = m g v$ est la puissance utile.

Le point de fonctionnement nominal, point de fonctionnement "normal" prévu par le constructeur (dont les valeurs typiques des différents paramètres est indiqué dans la notice et directement sur l'appareil), correspond généralement au point de fonctionnement de rendement maximal. Vérifier si c'est le cas ici.

\end{tcolorbox}





\section*{Glossaire}
\addcontentsline{toc}{section}{Glossaire}
\begin{sortedlist}
  \sortitem{\textbf{Contacteur électromagnétique:} appareil électrotechnique destiné à établir ou interrompre le passage du courant, à partir d'une commande à distance, électrique ou pneumatique. Relais électromagnétique qui permet de faire la liaison entre un circuit de commande et un circuit de puissance. Il permet d'ouvrir ou fermer un circuit automatiquement pour piloter certains appareils électriques (moteurs, résistances, etc). Il fonctionne comme un interrupteur à l'intérieur d'un circuit en établissant ou en interrompant le passage du courant.}
  \sortitem{\textbf{Relais électromécanique :} organe électrique permettant de distribuer la puissance à partir d'un ordre émis par la partie commande. Ainsi, un relais permet l'ouverture et la fermeture d'un circuit électrique de puissance à partir d'une information logique.}
  \sortitem{\textbf{Actionneur :} objet qui transforme l'énergie qui lui est fournie en un phénomène physique qui fournit un travail, modifie le comportement ou l'état d'un système.}
\end{sortedlist}








\newpage

%------------------------------------------


\chapter{Induction électromagnétique}


\paragraph*{Niveau:} L3
\paragraph*{Prérequis:} 
\begin{itemize}
\item Equations de Maxwell
\item Forces de Lorentz, de Laplace
\item ARQS magnétique 
\item Potentiels scalaire et vecteur 
\item Électromagnétique
\end{itemize}

\paragraph*{Bibliographie:}
\begin{itemize}
\item Electromagnatisme 3 : magnétostatique, induction, équations de Maxwell et compléments électroniques. M. Bertin, J. P. Faroux, J. Renault. Dunod Université (1986)
\item Physique Spé. MP*, MP et PT*, PT. Hubert Gié, Jean-Pierre Sarmat, Stéphane Olivier, Christophe More. Editions Tec\&Doc (2000)
\item Physique Spé. PSI*, PSI. Stéphane Olivier, Christophe More, Hubert Gié. Editions Tec\&Doc (2000)
\item Électromagnétisme PC-PSI. Les nouveaux Précis. Bréal. Krempf.
\end{itemize}


\section*{Introduction}
  \addcontentsline{toc}{section}{Introduction}

\textbf{Poser dès le début le cadre générale de la leçon: ARQS magnétique}

[Slide] \textbf{Introduction historique:} Oersted (1820): courants électriques induisent $\bm B$. Faraday (1831): Variations de $\bm B$ qui induisent des courants électriques. 

\section{Approche expérimentale}

\subsection{Expérience introductive} 

Approche un aimant et éloigne un aimant droit d'une bobine fixe branchée à un oscilloscope: apparition d'une tension. Même observation avec déplacement de la bobine dans aimant fixe. Amplitude de l'intensité proportionnelle à la vitesse de variation de $\bm B$.

Bien exploiter la première expérience (aimant mobile dans une bobine) pour montrer que la fem dépend du flux du champ magnétique (retourner l'aimant, changer le nombre de spires, le bouger tangentiellement à la bobine plutôt que le long de la bobine) et de sa variation temporelle (aller plus ou moins vite).

\subsection*{Observations}

A l'oral:

\begin{itemize}
\item Apparition de la fem si l'aimant ou la bobine bouge
\item La fem dépend du flux du champ
\item La fem dépend de la variation temporelle du flux
\end{itemize}

\subsection{Définition}

Consiste en l'apparition d'une f.e.m et, s'ils peuvent s'écouler, de courants, dans un conducteur mobile placé d'un champ magnétique variable.

\subsection{Deux types d'induction}

\begin{itemize}
\item Induction de Neumann (circuit fixe, champ variable)
\item Induction de Lorentz (circuit mobile, champ stationnaire).
\end{itemize}

\subsection{Loi de Faraday}

\begin{itemize}
\item $e = - \frac{\ud \phi}{\ud t}$
\item Validité : circuits filiformes
\item Rappel définition du flux : $\phi = \iint \bm B \cdot \ud \bm S$
\item unités de $e$ et $\phi$, convention d'orientation de la surface par rapport au circuit (règle de la main droite)
\item Convention générateur de la f.e.m.
\end{itemize}

\subsection{Loi de Lenz}

Discussion du signe $-$ dans la loi de Faraday.

\paragraph*{Manip' qualitative} chute d'un aimant dans un tube conducteur.


\section{Théorie de l'induction}

\subsection{Définition formelle de la fem}

$e = \frac{1}{q} \oint \bm F(\bm r, t) \cdot \ud \bm l$.

Bien discuter la cohérence de cette définition en terme d'interprétation physique (cf. Tec\&Doc).

Ici : $\bm F$ force de Lorentz $\rightarrow e = \oint \bm E \cdot \ud \bm l + \oint (\bm B \land \bm v) \cdot \ud \bm l$.

\subsection{Induction de Neumann}

$\bm v \sslash
 \ud \bm l \rightarrow e = \oint \bm E \cdot \ud \bm l$. \\
 Équation de Maxwell-Faraday : $\nabla \land \bm E = - \frac{\partial \bm B}{\partial t}$ donne la loi de Faraday.

\subsubsection*{Notion de champ électromoteur}

 $\bm E = - \nabla V - \frac{\partial \bm A}{\partial t} \rightarrow e = \oint \bm{E_m} \cdot \ud \bm l $ où $E_m = - \frac{\partial \bm A}{\partial t}$ est le \textbf{champ électromoteur de Neumann}. 
 
 
\textcolor{mycolor5}{Et pour l'induction de Lorentz ?}
 
\subsection{Induction de Lorentz}
 
Non relativiste : $\bm v = \bm{v_r} + \bm{v_e}$, $\bm{v_r} \sslash \ud \bm l$ \\
$\bm E = - \nabla V$. \\
$e = \oint (\bm v_e \land \bm B) \cdot \ud \bm l$. Le terme $\bm v_e \land \bm B$ se substitue au champ électromoteur de Neumann. \\
Selon le temps: le faire proprement ou à l'oral : montrer qu'en utilisant l'équation de Maxwell-Flux, on retrouve la loi de Faraday.
\textbf{Éventuellement faire le calcul pour le rails de Laplace et admettre la généralisation.}

\section{Aspects pratiques}

Dessin spire avec ligne de champ. \\
    Flux propre : $\phi_p = L i$, $L$ inductance propre (H). \\
    f.e.m : $e = - L \frac{\ud i}{\ud t}$. \\
    Schéma équivalent en éléctrocinétique : convention générateur avec générateur, convention récepteur avec bobine. \\
    - \textbf{Expérience quantitative 1 : Mesure de L}: Circuit RL, mesure du temps caractéristique sur oscilloscope.
    
\section{Induction mutuelle}

Dessin spire 1 avec ligne de champ et spire 2 dans champ magnétique créé par spire 1. \\
    - Flux créé par spire 1 à travers spire 2: $\phi_{21} = M_{21} i_1$ ; \\
    - Flux créé par spire 2 à travers spire 1: $\phi_{12} = M_{12} i_2$ ; \\
    - $M_{12} = \oint \oint \frac{\mu_0 \ud \bm{l_1} \cdot \ud \bm{l_2}}{4 \pi r_{12}} = M_{21}$.

\section{Exemple d'application: le transformateur}

Modèle du transformateur parfait. Faire le schéma équivalent.

\subsubsection*{Manip' qualitative} Montrer la tension du secondaire.



\section*{Conclusion}
  \addcontentsline{toc}{section}{Conclusion}

\textbf{Conclusion} Applications diverses (on a vu bobines et transformateurs). \\
Autres applications (slides) : Plaques à induction, Freinage par induction.


\section*{Description de l'expérience}
  \addcontentsline{toc}{section}{Description de l'expérience}



\begin{tcolorbox}[breakable, enhanced, colback=red!2!white,colframe=mycolor!85!black,title=\textbf{\textbf{Expérience}}]
\paragraph*{Matériel}
\begin{itemize}
\item Chronomètre
\end{itemize}
\paragraph{Manip 1}
\begin{itemize}
\item Aimant droit
\item Bobine de Leybold 
\item Oscilloscope
\item Câbles et fils
\end{itemize}
\paragraph{Manip 2}
\begin{itemize}
\item Tube conducteur + aimant
\end{itemize}
\begin{itemize}
\item 2 Bobines de Leybold (250 et 500 spires)
\item GBF
\item Fils et câbles
\item RLC-mètre
\item Multimètre (ohmmètre)
\item T.
\item Amplificateur de puissance (pour pas que le signal du GBF soit déformé).
\end{itemize}

\paragraph*{Protocoles}

\paragraph*{Protocole 1}  

Mettre l'échelle adéquate pour l'oscilloscope (ex: 200mV, 500ms). Faire les mesures en Single.

On peut bien exploiter la première expérience (aimant mobile dans une bobine) pour montrer que la fem dépend du flux du champ magnétique (retourner l’aimant, changer le nombre de spires, le bouger tangentiellement à la bobine plutôt que
le long de la bobine) et de sa variation temporelle (aller plus ou moins vite).	

On peut savoir si c'est un tension négative en premier ou en deuxième en regardant le sens du bobinage (dessiné sur la bobine). A partir de là, le sens des lignes de champ de l’aimant étant connu, on peut connaître le signe du flux et de sa variation temporelle, et ainsi connaître le signe de la tension aux bornes de la bobine.


\paragraph*{Protocole 2} 

\begin{itemize}[label=$\triangleright$]
\item Mesurer au préalable au ohmmètre par ex $R = 10k\Omega$ et $r$ interne de la bobine et $R$ du GBF OU du secondaire de l'amplificateur de puissanccalorim. Mesurer au RLC-mètre L.
\item Faire le circuit GBF-L-R.
		\item Envoyer un créneau sur le circuit RL (ex: 5Vpp, 1kHz).
		\item Mesurer $U_R$ sur l'oscilloscope
		\item Avec les curseur mesurer $\tau = \frac{L}{R}$ via $U_R(\tau) = 0.63 U_{max}$.
\end{itemize}

\paragraph*{Aspect quantitatif :} Mesure de $L$ sur l'oscilloscope via $\tau$. Faire une autre mesure au RLC-mètre. Comparer. \\

\textbf{N.B.} Y aura des déformations due à la résistance interne du GBF: \textbf{utiliser un amplificateur de puissance}. En plus y a une histoire de pont diviseur de tension selon la valeur de $R$, il envoie Vpp ou Vpp/2. En plus y a des effets capacitifs qui nous font des oscillations.

\end{tcolorbox}




\newpage

%------------------------------------------


\chapter{Rétroaction et oscillations}

\paragraph*{Niveau:} PSI

\paragraph*{Prérequis:} 
\begin{itemize}
\item Amplificateur linéaire intégré.
\item Electrocinétique
\end{itemize}

\paragraph*{Bibliographie:}
\begin{itemize}
\item Physique PSI-PSI* Tout-en-un. Dunod.
\item Physique SP\'E PSI*-PSI. Tec\&Doc. Olivier, More \& Gié.
\item Électronique -- Conversion de Puissance. PSI-PSI*. ellipses. Taupe-niveau. Meiler, Irlinger \& Kempf.
\item H-prépa. Électronique I PSI (vert). Brébéc et al.
\item Physique PSI, Pascal Olive. ellipses.
\item Précis PSI \'Electronique. Brenders, Buffard, Douchet, Sauzeix et Tisserand. Bréal.
\item Dictionnaire de physique. Taillet
\item Cours Jérémy : \url{https://jeremy.neveu.pages.in2p3.fr/Electronique/asservissements.html}
\item MANIP Bellier, p332 oscillateur de Wien
\item Réussir les TPs de physique aux concours. Sallen et Meier. Dunod
\item \url{http://perso.ens-lyon.fr/tristan.jocteur-monrozier-fabre/LP22-R%C3%A9troaction%20et%20Oscillations/LP22_R_troaction_et_Oscillations.pdf}
\item Bonus: Electronique. Pérez.
\end{itemize}

\paragraph{Notes agrégat}
\begin{itemize}
\item 2015 : Dans le cas des oscillateurs auto-entretenus, les conditions d’apparition des oscillations
et la limitation de leur amplitude doivent être discutées. Le jury souhaiterait que
le terme de résonance soit dûment justifié sans oublier une discussion du facteur de
qualité. Il n’est pas indispensable de se restreindre à l’électronique.
\item Jusqu’en 2013, le titre était : Rétroaction et oscillations. Exemples en physique.
\item 2013 : Le jury n’attend pas une présentation générale et abstraite de la notion de système bouclé.  Il
estime indispensable de s’appuyer sur au moins un exemple concret et détaillé avec soin.
\item 2007 : La stabilité des systèmes bouclés est mal comprise. Le bouclage ne se limite pas uniquement à une
fonction d’asservissement. Le lien entre les réponses temporelle et fréquentielle est un aspect important.
\end{itemize}

\paragraph{Note JF} Un système en boule ouverte aussi simple soit-il n’est pas du tout satisfaisant pour une utilisation au quotidien. Pour atteindre un point de fonctionnement donné et y rester, il faut intégrer une rétroaction, on dit alors qu’on travaille en boucle fermée. Deux cas se présentent alors : soit le système est stable, le point de fonctionnement est atteint etmaintenu (c’est le cas usuellement recherché dans la vie de tous les jours) ; soit le système est instable, la réponse est alternativement supérieure et inférieure à la consigne : des oscillations apparaissent. Si c’est généralement un défaut, on peut aussi le concevoir comme un but à atteindre pour certains systèmes.

\section*{Introduction}
  \addcontentsline{toc}{section}{Introduction}
  
[Slide] Montrer des exemples de la vie sur la rétroaction: [Tec\&Doc psi, p. 65]

\begin{itemize}
\item En physiologie: régulation de la température du corps ($T$ normale $\rightarrow$ Système nerveux $\rightarrow$ glandes sudoripares $\rightarrow$ Température du corps $\rightarrow$ boucle de rétroaction vers système nerveux.
\item Régulation hydroélectrique de l'organisme.
\item En thermique: four.
\item En électronique: amplificateur non inverseur.
\end{itemize}

\textbf{Ou bien suivre avec l'exemple du chauffage central dans une habitation [Olive psi, p. 117]}

\paragraph{Mais c'est quoi la rétroaction} [Taillet, p. 654] Réintroduction d'une partie du signal de sortie d'un système à son entrée.

[Jules Fillette] L'idée de base de la rétroaction est de réinjecter tout ou une partie du signal de sortie vers l'entrée d'un opérateur. La notion de rétroaction est très générale et s'applique dans de nombreux domaines.


%\subsubsection*{Expérience qualitative moteur sans boucle de rétroaction}

%Commande direct : la consigne peut varier si des paramètres varient ($T$, tension du réseau, frottements apparaissant, etc.). On a alors plus ce qu'on voulait (par exemple vitesse de rotation d'un moteur) $\rightarrow$  nécessité d'une boucle de rétroaction.


\section{Systèmes bouclés linéaires}

Définition système linéaire [Tec\&Doc psi, p. 4]

\subsection{Schéma fonctionnel général}

[Tec\&doc psi, p. 67. Précis, p. 194]

cf. \url{https://jeremy.neveu.pages.in2p3.fr/Electronique/asservissements.html}

[Sylvio] Un système bouclé est constitué d’une chaîne directe qui donne la sortie "naive" en fonction de l’entrée. La sortie
est récupéré pour être comparer à la commande de sorte d’ajuster la réponse. la différence en sortie du comparateur
est appelé erreur. Si la sorti est trop grande alors cela va entraîner une rétroaction négative qui aura tendance à
diminuer la sortie.

[Précis, p. 1966] Illustrer avec \{conducteur + voiture\}. 

\textbf{Lire [H-prépa, p. 135] pour plus de détails et pour les questions.}

\begin{itemize}
\item une chaîne directe de fonction de transfert $A$ contenant un actionneur.
\item une chaîne de retour de fonction de transfert $\beta$ qui peut contenir un capteur ou être directe ($\beta = 1$).
\item un comparateur (souvent un soustracteur) qui fournit le signal de commande de la chaîne directe en comparant le signal d'entrée au signal de retour.
\end{itemize}

Calcul de la fonction de transfert.

\subsubsection{Application à l'amplificateur non inverseur}

[Jérémy, Tec\&Doc psi, p. 66. Olive, p. 120]

Un montage à amplificateur opérationnel se met sous la forme d'un système bouclé,
avec comparateur, chaîne directe et chaîne de retour. \newline

Calcul de la fonction de transfert en boucle fermée:
\begin{itemize}
\item $A = \mu_0$
\item $\epsilon = v_+ - v_-$
\item $u_r = \frac{R_1}{R_1 + R_2} s$ (pont diviseur de tension)
\item $\beta = \frac{R_1}{R_1 + R_2}$  
\item $H_{FTBF} =  \frac{\mu_0}{1 + \mu_0 \beta}$. 
\end{itemize}
On retrouve le résultat attendu.


\subsection{Améliorations apportées par la rétroaction}

H-prépa et Tec\&Doc. Continuer de raisonner sur l'ALI.

\subsubsection{Cas d'un gain élevé}

[Kempf, p. 150]

A grand gain, le système se comporte comme un système en chaîne ouverte dont la fonction de transfert ne dépend que des caractéristiques de sa chaine de rétroaction.  \textbf{Intérêt:} la chaîne de rétroaction peut être fixée avec une grande précision, contrairement à la chaine d'action (vieillissement, température, etc.).


\subsubsection{Moindre sensibilité aux variations}

[Kempf, p. 150. Tec\&Doc, p. 69] Faire le calcul $\frac{\ud H}{H} = \frac{\ud \mu}{\mu} \frac{1}{1+\beta \mu}$. Discuter. A.N. pour ALI.

\subsubsection{Immunité aux perturbations}

[Tec\&Doc psi, p. 69. Kempf, p. 152]

\subsubsection{Elargissement de la bande bassante}

[Tec\&Doc psi, p. 70. Kempf, p. 154]

Il faut aussi s'intéresser aux performances dynamiques. On a vu que la chaîne de retour ne devait pas trop varier $B(\omega) \simeq B$. On s'interesse à une chaîne d'action du premier ordre $A(\omega) = \frac{\mu_0}{1+j \frac{\omega}{\omega_0}}$ (ce qui est vrai pour les ALIs usuels). Fonction de transfert s'écrit sous la forme $H = H_0 A(\omega) = \frac{\mu_0}{1+j \frac{\omega}{\omega_c}}$, avec $H_0$ la fonction de transfert de la chaîne directe en statique, et on voit apparaître une nouvelle fréquence de coupure $\omega_c = \omega_0 (1+\mu_0 B)$. Avec $\mu_0 B >> 1$, on peut en déduire \textbf{La rétroaction permet une augmentation importante de la bande passante du système. En temporel, cela se traduit par un accroissement sensible de la rapidité.}

\subsubsection{Aspect temporel}


[Tec\&Doc, p. 73] Discuter le lien temporel. Parler d'un système du second ordre.

\paragraph{Transition} la présence d'une rétroaction peut dans certains cas engendrer une instabilité.

\subsection{Stabilité}

\subsubsection{Définition}

[Kempf, p. 156] Un système physique est dit stable s'il retourne spontanément vers son état d'équilibre lorsqu'il en est écarté.

[Précis, p. 203] Un système bouclé évoluant en régime libre ($e=0$) au voisinage de son équilibre ($s=0$) sera dit \textit{stable} si l'évolution de la sortie tend spontanément vers l'équilibre ($s \rightarrow 0$), \textit{instable} dans le cas contraire.

Un système bouclé peut être stable ou instable. Lorsqu'il est instable, il peut se mettre à osciller et cette oscillation peut être exploitée pour la réalisation d'oscillateurs (objet de la première partie).
Si le système bouclé est stable, il permet au contraire de réaliser une régulation, un asservissement (deuxième partie). Dans ce cas là, il convient d'éviter les oscillations du système.

\subsubsection{Condition}

[Précis, p. 203]

\textbf{Condition depuis l'équa. diff.} Un système linéaire permanent est stable si et seulement si la totalité des racines de l'équation caractéristique sont soit réelles négatives, soit complexes à partie réelle négative. Ceci est équivalent à:

\textbf{Condition sur $H$:} Un système asservi est stable si et seulement si tous les pôles (i.e. les zéros du dénominateur) de sa fonction de transfert en boucle fermée ont une partie réelle strictement négative. Ça se voit car les pôles sont les racines de l'équation caractéristiques en temporel. \newline

\textbf{N.B.} Je pense qu'il vaut mieux faire simple et faire un cas particulier comme [Tec\&Doc]. On part de l'équa diff de la partie "aspect temporel" et on en déduit le critère de Tec\&Doc. Puis on dit que ça se généralise à une condition sur la fonction de transfert en boucle ouverte en disant "stable si tous les zéros de $1 + A(p) B(p)$ ont une partie réelle strictement négative [Kempf, p. 159]".

\subsection{Application aux ALIs}

\paragraph{Pour le système bouclé} $H = \frac{A}{1+AB}$. Stable si $1+AB = 1+H_{BO}$ si ses racines sont à partie réelle négative.

[Tec\&Doc, p. 76. Cours Jérémy]

\textbf{N.B.} De manière générale, il y a une compétition entre rapidité et stabilité.


\paragraph{Transition} [JF] Si la condition de stabilité n'est pas respectée dans le système bouclé, des oscillations peuvent apparaitre: comment les rendre utiles ? Comment les maintenir ?

\section{Oscillateurs}

[Tec\&Doc, p. 77] Oscillations peuvent résulter d'une instabilité (Si toutes les parties réelles sont nulles, le système asservi est oscillant).

\subsection{Définition}

\textbf{(Ne pas forcément faire cette partie)} \\

Un oscillateur est un système bouclé auto-oscillant, c'est-à-dire qui oscille sans requérir de signal d'entrée.

On distingue deux types d'oscillateurs:
\begin{itemize}
\item les oscillateurs quasi-sinusoïdaux, tels que le signal de sortie soit presque sinusoïdal.

\item les oscillateurs à relaxation, délivrant un signal périodique non sinusoïdal.
\end{itemize}

Un oscillateur quasi-sinusoïdal est un oscillateur qui génère un signal comprenant un harmonique principal et des harmoniques secondaires à faible effet sur le signal sinusoïdal. La forme du signal est alors proche d'une sinusoïde (dite quasi sinusoïde).

\section{Oscillateur à boucle de rétroaction}

[Tec\&Doc, psi. Précis, p. 237. Hprépa I, p. 180]

Faire le schéma e - amplificateur - s - quadripôle.


\textbf{ex:} Laser.

\paragraph{On va tout illustrer sur l'oscillateur à pont de Wien.}

\subsection{Oscillateur à pont de Wien}

[H-prépa] ALI non inverseur + pont de Wien

On étudie ici les oscillateurs constitués d'un amplificateur et d'un passe-bande, bouclés car la sortie de l'un est l'entrée de l'autre, et vice versa.

\subsection{Condition d'oscillations} 

[Tec\&Doc, p. 79] Dériver l'équation différentielle. N.B. La deuxième équation sur $i$ s'obtient en faisant le schéma équivalent avec les impédances complexes : $e = Z i$ avec $Z = \frac{1}{1+j\omega RC}$. Solution sinusoïdale si $A = 3$, fréquence d'oscillation $\omega_ = \frac{1}{RC}$.

\paragraph{Expérience} vérifier le critère et mesurer $f$.
\textbf{Voir TP systèmes bouclés et [Hprépa, p. 158] pour d'autres manips'.}

\subsubsection{Critère de Barkhausen}

Le critère de stabilité de Barkhausen fixe une condition nécessaire pour qu'un circuit consistant en un amplificateur et une boucle de contre-réaction, se mette spontanément à osciller. Ce critère ne permet cependant ni d'affirmer que les oscillations seront durables, ni qu'elles seront d'amplitude constante. Ce critère marque les débuts de l'étude des circuits oscillants ; sa portée a depuis été précisée par d'autres critères, comme le critère de Nyquist.

\paragraph{N.B.} Le critère de stabilité de Nyquist est une règle graphique utilisée en automatique et en théorie de la stabilité, qui permet de déterminer si un système dynamique est stable. Cette construction, qui exploite le diagramme de Nyquist (Im H vs Re H en cartésien, ou G vs arg en polaire) des circuits à boucle ouverte, permet de se dispenser du calcul des pôles et des zéros des fonctions de transfert (bien qu'il faille connaître le nombre et le type des singularités du demi-plan réel). 

Critère du revers: Si un système linéaire est stable en boucle ouverte, une condition nécessaire et suffisante de stabilité asymptotique du système en boucle fermée est qu’en parcourant le lieu de Nyquist 
$H_F$ dans le sens des pulsations croissantes, on laisse le point critique 
-1 à gauche.

Critère de Nyquist: Un système asservi de fonction de transfert en boucle ouverte $H_O$ est asymptotiquement stable en boucle fermée à la condition nécessaire et suffisante que $H_F$ entoure le point critique -1 dans le sens trigonométrique un nombre T de fois égal au nombre de pôles P instables (à partie réelle positive) de $H_F$.

\subsubsection{Portrait de phase}

\textbf{Lire [Précis] à partir de la p. 245}

Si le temps, on peut faire ça. 	

On peut aussi parler des autres oscillateurs (résistance négative, etc.).

\section*{Conclusion}
  \addcontentsline{toc}{section}{Conclusion}

Oscillateur à relaxation ? Oscillateur à quartz [Jérémy] ?
Laser ?


\subsection*{A savoir [Agrégat]}
\begin{itemize}
\item Caractéristique principale des systèmes bouclés: Produit gain * bande passante = cste.
\item Qu’est-ce qui limite l’amplitude des oscillations dans l’oscillateur à pont
de Wien ? Effets non linéaires.
\item D’où viennent les non linéarités ? L’AO (transistors de l’AO).
\item Que se passe-t-il si on modifie la valeur de la résistance variable dans l’oscillateur à pont de
Wien ? Déformation du signal, on s’éloigne des oscillations quasi sinusoïdales.
\item Qu’est-ce qui
caractérise un oscillateur ? Son facteur de qualité.
\item Que vaut-il pour l’oscillateur à pont de Wien ? (1/3).
\item Un système linéaire continu et stationnaire n’est qu’une modélisation, les systèmes réels ne satisfont pas rigoureusement tous ces critères.
\end{itemize}


\section*{Description de l'expérience}
  \addcontentsline{toc}{section}{Description de l'expérience}


\textcolor{mycolor5}{Suivre le montage de H-prépa électronique I p.183-184.}

\begin{tcolorbox}[breakable, enhanced, colback=red!2!white,colframe=mycolor!85!black,title=\textbf{\textbf{Expérience}}]
\paragraph*{Matériel}
\begin{itemize}
\item 4 boîtes à décades R (Attention ! Il faut qu'au moins 2 soient réglables)
\item 2 boîtes à décade C
\item 1 ALI
\item Alimentation pour ALI 
\item Câbles/fils
\item Oscilloscope + câbles
\item Multimètre (ohmmètre)
\item RLC-mètre
\item Chronomètre
\end{itemize}

\paragraph*{Protocole } 

\begin{itemize}[label=$\triangleright$]
		\item Faire le montage.
		\item $R_2 = 10 k \Omega$, $R_1 = 4966 \Omega$ ? Prendre les valeurs de Hprépa ou du livre de TP. Les mesurer au préalable au ohmmètre. Mesurer $C$ au RLC-mètre.
\end{itemize}

\paragraph*{Aspect quantitatif :} Montrer que les oscillations commencent à $G=3$ (critère de Barkhaussen). Mesure de $f$. Vérifier que $f = \frac{1}{2 \pi RC}$.

\paragraph{Autres manip'}
On peut utiliser un circuit dérivateur (passe-haut RC) [Hprépa, p. 184] pour tracer le portrait de phase en mode XY.

[TP Systèmes bouclés + livre TP] On peut aussi étudier le filtre de Wien : mesure de Q et de la fréquence de résonance. A la résonance $\omega = \frac{1}{RC}$, $arg(H) = 0$: $s$ et $e$ sont en phase: on peut déterminer la fréquence de résonance en mode XY ($e$ en X et $s$ en Y). Pour déterminer $Q$, on cherche $f_1$ et $f_2$ de part et d'autre de $f_0$ telles que $s = \frac{v_{s,max,res}}{\sqrt{2}}$.

\end{tcolorbox}

\newpage

%------------------------------------------


\chapter{Traitement d'un signal. Étude spectrale}

\paragraph*{Niveau:} CPGE
\paragraph*{Prérequis:} 
\begin{itemize}
\item 
\end{itemize}

\paragraph*{Bibliographie:}
\begin{itemize}
\item Traitement des signaux et acquisition de données. Francis Cottet. Dunod.
\item Pascal Olive. Physique-chimie en PSI/PSI*. ellipses.
\item Sanz. Toute en un physique. \textbf{MP et PSI.}
\item Tec\&Doc. Physique MP. Gié et al.
\item H-prépa. Électronique 2ème année. PSI. Brébec et al.
\item Les nouveaux précis. Tout-en-un Physique PSI-PSI*. Tisserand et al. Bréal.
\item Expériences de physique : électricité, électromagnétisme, électronique, transferts thermiques. Bellier, Bouloy et Guéant. Je Prépare. 4ème édition
\end{itemize}


\section*{Introduction}
  \addcontentsline{toc}{section}{Introduction}

\subsection*{Signal: définition}

Un signal est une représentation physique d'une grandeur mesurable porteuse d'information. Exemple: pression, température, tension, intensité, etc.

\subsection*{Traitement du signal : définition}

[Sur slide] Discipline qui a pour objet l'élaboration ou l'interprétation ou l'analyse des signaux.  Parmi les types d'opérations possibles : filtrage, compression, transmission de données, réduction du bruit, etc.

\subsection*{Exemples d'opération du traitement du signal}

\begin{itemize}
\item \'Elaboration : modulation.
\item Interprétation : filtrage.
\item Analyse : transformée de Fourier.
\item Mesure : valeur moyenne.
\end{itemize}


\subsection*{Analogique vs numérique}


\section{Étude spectrale d'un signal}

[Tec\&Doc, p. 33]

\url{https://femto-physique.fr/omp/serie-de-fourier.php}


\subsection{Décomposition d’un signal en série de Fourier}

Décomposition d'un signal périodique. Définir spectre. [Hprépa].


\subsubsection{Principe de superposition}

Intérêt des sinus dans les systèmes linéaires (principe de superposition).

\paragraph{Transition} Nous verrons dans la suite l'action d'un filtre linéaire sur un signal périodique.


\subsection{Filtrage (linéaire)}

\url{https://femto-physique.fr/electrocinetique/filtrage-passif.php}

\url{https://femto-physique.fr/simulations/filtrage-passe-bas.php}

[Sanz MP, chapitre 4] Faire pour un sinus puis pour un signal périodique quelconque. 

\paragraph{Filtre passe-bas} Diagramme de Bode. Composition spectrale.

\subsubsection{Manip' quantitative} Filtre passe-base RC. On envoie un carré ou un triangle, on récupère un triangle ou un sinus. Tracer digramme de Bode sur Qtiplot. Faire le fit -2 dB/décade. Mesurer fréquence de coupure et comparer avec la théorie. 

\paragraph{Transition} [Sanz MP] On a vu qu’un filtre, système linéaire, n’enrichit pas le contenu spectral du signal. En revanche, si la relation entre le signal d’entrée e(t) et le signal de sortie s(t) d’un système est non linéaire, il est possible que le spectre du signal de sortie contienne des fréquences absentes du signal d’entrée.

\section{Modulation et démodulation d'amplitude (non linéaire)}

\url{https://jeremy.neveu.pages.in2p3.fr/Electronique/modulation.html#}

[Cottet, chapitre 3, p. 43 et p. 80. Sanz PSI, Chapitre 6]

\subsection{Principe de la modulation}

\subsection{Principe de la démodulation}


\subsubsection{Manip' qualitative} Modulation (multiplication du signal modulant et de la porteuse). FFT pour montrer le spectre du signal modulé. Puis démodulation par détection synchrone (en utilisant le filtre passe-bas de la section précédente).

\paragraph{Transition} On a traité jusque là que des signaux analogique.

\section{Conversion analogique-numérique}

[Sanz MP et PSI. Cottet] \url{https://jeremy.neveu.pages.in2p3.fr/Electronique/numerique.html#}

\subsection{Échantillonnage}

Définition. Spectre. Critère de Shannon et repliement du spectre.

\url{https://perso.ens-lyon.fr/sylvio.rossetti/AGREG/LP/LP23_Aspects%20analogique%20et%20num%e9rique%20du%20traitement%20du%20signal.%20Etude%20spectrale/LP23_Aspects_analogique_et_num_rique_du_traitement_du_signal__Etude_spectrale.pdf}

\subsection{Filtrage numérique}

[Sanz MP et PSI] Une fois le signal échantillonné, on peut effectuer sur lui des traitements en s’appuyant sur la
puissance de calcul du calculateur. Un filtre numérique est un dispositif qui effectue des opérations mathématiques sur un signal discret échantillonné.

\section{Avantages}

[Cottet, Sanz MP et PSI]

\section*{Conclusion}
  \addcontentsline{toc}{section}{Conclusion}
  
On peut parler de TF, application à la diffraction. TFD, FFT pour traiter les signaux numériques (équivalent discret de la transformation de Fourier (continue) utilisée pour traiter un signal analogique) ? \url{https://fr.wikipedia.org/wiki/Transformation_de_Fourier_discr%C3%A8te}

Donner d'autres applications hors électronique. Voir par exemple les approches documentaires de Dunod.

\paragraph{A connaître: notion de fenêtrage} [wiki] En traitement du signal, le fenêtrage est utilisé dès que l'on s'intéresse à un signal de longueur volontairement limitée. En effet, un signal réel ne peut qu'avoir une durée limitée dans le temps ; de plus, un calcul ne peut se faire que sur un nombre fini de points. Pour observer un signal sur une durée finie, on le multiplie par une fonction fenêtre d'observation. La plus simple est la fenêtre rectangulaire (ou porte). Lire Cours Jérémy.

\section*{Description de l'expérience}
  \addcontentsline{toc}{section}{Description de l'expérience}


\begin{tcolorbox}[breakable, enhanced, colback=red!2!white,colframe=mycolor!85!black,title=\textbf{\textbf{Expérience}}]
\paragraph*{Matériel}
\paragraph*{Pour le filtre PB}
\begin{itemize}
\item Câbles BNC/BNC (suffisamment $\times 10$ ?)
\item 2 multiplieurs analogiques
\item 2 alimentations pour multiplieurs ($\pm 15V$
\item Câbles d'alimentation
\item Câbles et fils (suffisamment)
\item 4 cavaliers + "ponts"
\item Plusieurs "T"
\item 2 GBF
\item 1 boîte $C$
\item 1 boîte $R$
\end{itemize}


\paragraph*{Filtre PB [Bellier et. al.} 

\begin{itemize}[label=$\triangleright$]
		\item  Circuit RC en série. $R = 10 k\Omega$, $C = 100 nF$.
		\item Pour différentes fréquences: mesurer la tension aux bornes du condensateur $V_c^{pp}$. On utilise "Meas" et on moyenne sur plusieurs périodes, surtout pour les hautes fréquences avec "acquire".
		\item Tracer sur Qtiplot $G(\omega) = 20 \log \frac{V_c^{pp}}{V_e^{pp}}$ (échelle log/log en abscisse).
		\item Si motivé, on peut aussi tracer $\varphi(\omega)$ avec Meas.
\end{itemize}

\paragraph*{Aspect quantitatif :} Mesure de $\omega_c$. Vérifier qu'elle est égale à $\frac{1}{RC}$. Faire le fit $-20$ dB/déc après la fréquence de coupure (attention: modéliser par $A \times \log(x) + B$ à cause de l'échelle log).

\paragraph*{Modulation/démodulation} 

\begin{itemize}[label=$\triangleright$]
		\item Suivre le schéma du poly.
		\item Pour la modulation : FFT.
		\item Pour la démodulation : par détection synchrone. Montrer qu'on récupère le bon signal (i.e. la bonne fréquence).
\end{itemize}

\paragraph*{Aspect quantitatif :} Mesurer les fréquences du signal modulé avec la FFT. Montrer que ça vaut bien $f_0 \pm f_p$.

\end{tcolorbox}


\newpage

%------------------------------------------



\chapter{Ondes progressives, ondes stationnaires}


\paragraph*{Niveau:} PC 
\paragraph*{Prérequis:} 
\begin{itemize}
\item Mécanique (première année)
\item Équation différentielle, développements limités, etc.
\item Notion de résonance, modes et pulsations propres
\end{itemize}

\paragraph*{Bibliographie:}
\begin{itemize}
\item Les nouveaux précis. Tout-en-un Physique PC-PC*. Tisserand et al. Bréal.
\item Physique Spé PC-PC*. Olivier, Gié, Sarmant. Tec \& Doc.
\item Tout-en-un PC, Sanz.
\item H-prépa Ondes 2ème année MP-PC-PSI. Brébec. Hachette Supérieur.
\item Ondes 2ème année PC-MP-PSI-PT. Classe Prépa. Nathan. Hulot et Venturi. 
\item Expériences de Physique. Optique, mécanique, fluides, acoustique. Béllier, Bouloy, Guéant. 4e édition.
\item Dictionnaire de physique. Taillet, Villain, Febvre.
\item Voir JF \url{https://www.lpens.ens.psl.eu/jfillette/}
\item Animation : \url{https://phyanim.sciences.univ-nantes.fr/Ondes/ondes_stationnaires/melde.php}
\item Animation : \url{https://phyanim.sciences.univ-nantes.fr/Ondes/ondes_stationnaires/stationnaires.php}
\end{itemize}

\paragraph*{N.B.} \textcolor{mycolor5}{Les deux parties doivent être équilibrées. L'équivalence entre les deux types d'onde doit être au coeur de la leçon. Idéalement, faire une manip' (qualitative) sur les ondes progressives.}

\paragraph*{N.B.} \textcolor{mycolor5}{Reprendre les définitions en utilisant le Taillet.}

\section*{Introduction}
  \addcontentsline{toc}{section}{Introduction}
  
Sur slide: images d'ondes (surface de l'eau, lumière, onde acoustique).  
  
\section{Généralités sur les ondes}

\subsection{Définition}

Définir une onde sur slide.

\subsection{Équation de propagation: établissement pour la corde vibrante}

\textcolor{mycolor5}{Il faut essayer d'être rapide sur cette partie, au pire mettre l'équation de d'Alembert en prérequis :}

\subsubsection{Hypothèses}
\begin{itemize}
\item Corde inextensible de masse linéique $\mu$.
\item On néglige le poids de la corde.
\end{itemize}

\subsubsection{Mise en équation}

PFD sur un élément de corde. $\alpha$ petit. Dérivation de l'équation de d'Alembert (1d).

\subsubsection{Propriétés}
\begin{itemize}
\item Linéaire
\item Réversible
\item Généralisation 3d.
\end{itemize}

\subsection{Solutions de l'équation de d'Alembert}

Deux familles de solutions : ondes planes progressives et ondes stationnaires. C'est ce qu'on va étudie dans la suite et voir qu'elles sont équivalentes.

\section{Solutions stationnaires}

\subsection{Définition}
Onde dont les dépendances spatiales et temporelles sont découplées.


\subsection{Exemple: la corde de Melde}

Selon le temps, résoudre plus ou moins: $y(x,t) = A cos(\omega t - \varphi) \cos(k x -\psi)$.

Donner les CL : $y(0,t) = y_0 \cos(\omega t)$ et $y(L,t) = 0$. Donne :

\begin{equation}
y(x,t) = y_0 \cos(\omega t) \frac{\sin(k(L-x))}{\sin(kL)}
\end{equation}

Pulsations de résonance: $\omega = \frac{n \pi c}{L}$. Fréquences de résonance: $f_n = \frac{n c}{2 L}$, avec $c = \sqrt{\frac{m g}{\mu}}$.

Justifier que ça ne diverge pas dans la vraie vie (dissipation).

\subsubsection{Expérience} Corde de Melde. Tracer $f(n)$. A l'extrémité de la corde, la tension vaut $T_0 = Mg$.

\url{https://nc.agregation-physique.org/index.php/s/AW9cDjFHrQMaRmW?dir=undefined&path=%2FLyon&openfile=48964}

Lire [Exo Dunod PCSI, exo 21.2 Chapitre 21: Poulie] La poulie est soumise à l’action de la liaison pivot d’axe (Oz), à la force exercée par l’opérateur, à la traction exercée par le fil et à son poids. La poulie est à l’équilibre lorsque la somme des
moments de toutes ces forces par rapport à l’axe de rotation (Oz) est nulle. \\

\url{https://phys.libretexts.org/Bookshelves/Classical_Mechanics/Classical_Mechanics_(Dourmashkin)/08%3A_Applications_of_Newtons_Second_Law/8.05%3A_Tension_in_a_Rope}

\paragraph{Transition} On a vu les ondes stationnaires. En vrai, somme de deux ondes progressives (Le dire sur la corde de Melde).

\section{Solutions progressives}

\subsection{Définition}

Onde ne dépendant que de la variable $x \pm ct$.

\subsection{Solution générale (théorème)}

$y(x,t) = f(x-ct) + g(x+ct)$. \\

Montrer que c'est bien une solution de l'équation de d'Alembert en la dérivant deux
fois par rapport à x et t et en l'injectant dans l'équation.

$f$ (resp.) $g$ pour propagation dans le sens des $x$ croissants (resp. décroissants).

La recherche des ces fonctions est compliquée, un moyen de simplifier la résolution est de décomposer en série de Fourier.

Elles sont utilisées pour décomposer des OPP (puisque l'équation de d'Alembert est linéaire).


\subsection{Onde plane progressive harmonique (OPPH)}

\subsubsection{Définition}
Onde plane progressive dont la dépendance en temps est sinusoïdale : $y(x,t) = A \cos(\omega t - kx + \varphi)$.

\subsubsection{Relation de dispersion}

La question qui se pose alors est de déterminer les caractéristiques  de chaque OPPH. Pour ça il faut établir le lien entre $k$ et $\omega$, appelé relation de dispersion. La déterminer pour d'Alembert en dérivant deux fois l'expression d'une OPPH par rapport à $x$ et $t$ et en injectant dans l'équation de d'Alembert. Définir vitesse de phase.



\section*{Conclusion}
  \addcontentsline{toc}{section}{Conclusion}

Équivalence des deux solutions: \textcolor{mycolor5}{Selon le temps, soit le dire à l'oral soit faire les démos.}

Une onde stationnaire est la superposition de deux OPPH (donc solution de d'Alembert). 

De même, les ondes progressives peuvent aussi s'écrire comme somme d'ondes stationnaires.

On choisit l'une ou l'autre selon le système étudié et les CLs. Ex: OPP pour corde infinie et OS pour corde fixée.
  
\section*{Description de l'expérience}
  \addcontentsline{toc}{section}{Description de l'expérience}


\begin{tcolorbox}[breakable, enhanced, colback=red!2!white,colframe=mycolor!85!black,title=\textbf{\textbf{Expérience}}]
\paragraph*{Matériel}
\begin{itemize}
\item GBF
\item Amplificateur de puissance
\item Fils et câbles
\item Vibreur
\item Corde
\item Poulie
\item Potence
\item Noix
\item Masses
\item Balance
\item Mètre
\item Serre-joint
\item Chronomètre
\end{itemize}

\paragraph*{Protocole } 

\begin{itemize}[label=$\triangleright$]
		\item Déterminer la masse volumique de la corde.
		\item Modifier la fréquence. Estimer l'incertitude. Trouver les modes propres.
		\item Tracer $f(n)$. Idéalement vs la longueur d'onde mesurée à la règle plutôt que $n$.
		\item Idéalement, utiliser un stroboscope.
\end{itemize}

\paragraph*{Aspect quantitatif :} Mesure de $c$ via deux méthodes: $f(n)$ et $\sqrt{\frac{m g}{\mu}}$.


\end{tcolorbox}
  

\newpage

%------------------------------------------


\chapter{Ondes acoustiques}


\paragraph*{Niveau:} 2ème année CPGE
\paragraph*{Prérequis:} 
\begin{itemize}
\item Équation de D'Alembert et bases de solution (ondes planes progressives/ondes stationnaires).
\end{itemize}

\paragraph*{Bibliographie:}
\begin{itemize}
\item Physique Spé. PC-PC* ou PSI-PSI*. Tec\&Doc. Olivier, More, Gié.
\item Ondes 2ème année PC-PSI. H-prépa. Brébec, Desmarais.
\item Tout-en-Un Physique PC. Sanz et al.
\item Dictionnaire de physique. Taillet. 
\item Physique expérimentale: électricité, électromagnétisme, électronique, acoustique. 3ème édition. Bellier, Bouloy et Guéant. DUNOD
\end{itemize}



\section*{Introduction}
  \addcontentsline{toc}{section}{Introduction}
  
Le son est un phénomène de la vie de tous les jours. Et si aujourd'hui, vous pouvez m'entendre, c'est parce que le son se propage. 

Qu'est-ce que le son ? Comment se propage-t-il ? C'est ce que nous allons voir.


Définir une onde acoustique (sur slide) : [H-prépa] Les ondes sonores sont des vibrations de faible amplitude du milieu matériel dans lequel elles se propagent à la vitesse $c_s$.


Exp de la cloche à vide -> le son a besoin d'un milieu pour se propager.

2'04 : \url{https://www.youtube.com/watch?v=BC9Pod4cnpk}  \\

\textcolor{red}{Passer plus du temps pour bien expliquer la cloche à vide}

\section{Propagation dans les fluides}

[tec\&Doc]

\subsection{Équation fondamentale des ondes sonores}

\textcolor{red}{Il faut s'entraîner à savoir bien refaire les calculs. Il faut aller plus vite sur cette partie pour avoir le temps de faire la partie sur l'énergie.}

\subsubsection{Hypothèses}

Bien poser toutes les hypothèses (cf. M. Rabaud).

\begin{itemize}
\item Écoulement parfait
\item On néglige la pesanteur
\item Au repos: $\mu_0$ et $p_0$ uniformes et constants. $v_0$ nul.
\end{itemize}

Onde sonore = perturbation de l'état de repos pour $v$, $P$ et $\mu$.

Approximation acoustique = à l'ordre 1. L'approximation acoustique est une approximation de grande longueur
d'onde.

$p_1$ = surpression. \\

\textcolor{red}{Avoir en tête l'équilibre thermodynamique local}

\subsection{Linéarisation des équations}

\begin{itemize}
\item Euler : $\mu_0 \frac{\partial v_1}{\partial t} = - \nabla p_1$
\item Conservation de la masse $\mu_0 \nabla \cdot v_1 + \frac{\partial \mu_1}{\partial t} = 0$
\item Évolution thermodynamique: isentropique. Coefficient de compressibilité isentropique $\chi_s = \frac{1}{\mu} \frac{\partial \mu}{\partial P} \mid_S$. Or: $\mu = \mu_0 + \frac{\partial \mu}{\partial t}$ donne au premier ordre $ \mu_1 = \mu_0 \chi_s  p_1$.
\end{itemize}

\subsection{Équations de propagation}

D'Alembert pour $v$ et $p$.

Célérité $c = \sqrt{\frac{1}{\mu_0 \chi_s}}$.

\subsection{Célérité des ondes sonores}

$c$ est d'autant plus grande qu'un milieu est moins dense et moins compressible.

\subsubsection{Mesure de la vitesse du son}

\subsubsection{Dans un gaz parfait}

Loi de Laplace : $P \mu^{-\gamma}$. On différentie la log : $\frac{\ud P}{P} = \gamma \frac{\ud \mu}{\mu}.$ Loi des GPs: $\frac{P}{\mu} = \frac{RT}{M}$. Donne $c = \sqrt{\frac{\gamma R T}{M}}$. $M = 29g/mol$ et $T = 300K$.

\paragraph{Expérience} Faire la mesure dans l'air.

Expliquer le principe [Bellier].

Comparer la valeur dans l'air avec la théorie. Comparer la valeur dans l'eau avec celle de l'air. Donner des odg.

\subsubsection{Interprétation microscopique} 

Expliquer avec les mains les facteurs influant sur la célérité: densité du milieu, température, pression.

\url{https://www.youtube.com/watch?v=Nkved7UcgqY}

\section{Ondes acoustiques planes progressives}

On utilise OPPH. \textcolor{red}{Bien revoir cette histoire de base.}

\subsection{Couplage entre $v$ et $p$}

En notation complexe. $\bm v \propto \bm k$ : Ondes longitudinales. $p_1 = \mu_0 c v_1$: les champs de pression et de vitesse proportionnels et vibrent en phase.


\subsection{Aspects énergétiques}

[Tec\&Doc]

. Moyenne du vecteur densité. ODG et justification de l'approximation acoustique. Intensité sonore. Échelle des décibel.

\subsection{Vecteur densité de flux}

$\Pi = p_1 \bm v_1$.

\subsection{Équation énergétique locale}

Densité d'énergie: $e = \frac{1}{2} \mu_0 v_1^2 + \frac{1}{2} \chi_s p_1^2$.

Pour une OPP. Discuter les odg et la validité de l'approximation acoustique.

\subparagraph{Intensité sonore}

Oreille = détecteur logarithmique. $I = 10 \log \frac{<\Pi>}{<\Pi_0>}$ (en dB). odg. Discuter les min et max pour I et P.

\section{Réflexion et transmission d'une OPP}

\textcolor{mycolor5}{Probablement ambitieux} \textcolor{red}{Je confirme: impossible}

\subsection{Impédance acoustique}

L'impédance acoustique est une quantité qui va permettre de caractériser la transmission et la réflexion des ondes acoustiques à travers une surface.

Adaptation d'impédance dans l'oreille. Échographie on met du gel.

\subsection{Conditions aux limites}

\subsection{Coefficients de transmission et réflexion}


\section*{Conclusion}
  \addcontentsline{toc}{section}{Conclusion}

\begin{itemize}
\item Validité des approximations (Lire Tec\&Doc à la fin).
\item Applications : Mesure des débits sanguins par effet Doppler, cavité sonore, isolation (cf. Épreuve 1998)
\item Dans les solides: Dans les solides, il n'existe pas seulement des ondes acoustiques longitudinales (dites ondes "P" en sismologie, pour Primaires car elles voyagent plus vite que les autres), mais aussi des ondes de cisaillement (S pour secondaires ou pour Shear).
\end{itemize}



\section*{Description de l'expérience}
  \addcontentsline{toc}{section}{Description de l'expérience}

\textcolor{mycolor5}{Bellier et al.}

\begin{tcolorbox}[breakable, enhanced, colback=red!2!white,colframe=mycolor!85!black,title=\textbf{\textbf{Expérience}}]
\paragraph*{Matériel}
\begin{itemize}
\item Diapason + truc pour frapper
\item Banc optique
\item Récepteur et émetteur + support sur le banc
\item Câbles + T
\item GBF
\item Oscilloscope
\item Éventuellement résistance $R=4k\Omega$ + multimètre
\item Chronomètre
\end{itemize}

\paragraph*{Protocole } 

\begin{itemize}[label=$\triangleright$]
		\item Faire le calcul (Bellier)
		\item Faire le montage: banc optique, émetteur sur GBF, GBF sur oscillo X, récepteur sur oscillo Y.
		\item En mode XY: voir quand les deux sont en phase, noter la première position. Déplacer de 10 périodes: noter $x_10$ quand il sont à nouveau en phase.
		\item $\Delta x = 10 \lambda$ et $c = \frac{\lambda}{T}$.
\end{itemize}

\paragraph*{Aspect quantitatif :} Mesure de la vitesse du son dans l'air.


\end{tcolorbox}


\newpage

%------------------------------------------


\chapter{Propagation guidée des ondes}


\paragraph*{Niveau:} L3
\paragraph*{Prérequis:} 
\begin{itemize}
\item Équations de Maxwell
\item Relations de continuité avec un conducteur parfait
\item Notation complexe
\end{itemize}

\paragraph*{Bibliographie:}
\begin{itemize}
\item Tec\&Doc MP-MP*, S. Olivier, Gié, Sarmant
\item Électromagnétisme 2, 2ème année MP-PC. Faroux et Renault. J'intège, DUNOD. 1998
\item Ondes 2ème année PC-MP-PSI-PT. Classe prépa, Nathan. Hulot et Venturi.
\item Ondes 2ème année MP-PC-PSI-PT. H-prépa, Hachette. Brébec.
\item Dictionnaire de physique. Taillet.
\item Exo sur mode TE$_{nm}$: Tec\&Doc Exercices et problèmes d'EM 2 (2ème année). Sarmant \textbf{OU} \url{https://fr.wikiversity.org/wiki/Ondes_%C3%A9lectromagn%C3%A9tiques_guid%C3%A9es/Guide_rectangulaire#%C3%89tude_des_modes_TEmn}
\item \url{https://www.etienne-thibierge.fr/agreg/ondes_poly_2015.pdf#page=69&zoom=100,60,668}
\end{itemize}



\section*{Introduction}
  \addcontentsline{toc}{section}{Introduction}
  
\paragraph*{Manip' qualitative} Lampe et fibre optique.


Inconvénient de la perte d'énergie pour une propagation libre, intérêt du guidage.

  
\subsection*{Définition}

\textcolor{mycolor5}{Éventuellement sur slide}
 
[Dico, FR] Déf., caractéristique. Type dépende de la gamme de fréquence.
  
\textcolor{mycolor5}{On va s'intéresser dans la suites aux ondes EM}
  

\section{Propagation dans un guide d'onde à section rectangulaire}

[Tec\&Doc, FR]


\subsection{Position du problème}

Schéma. Hypothèses. 

\subsection{Le champ électrique}


\subsubsection{Structure de l'onde}

[Nathan]

Progressive, monochromatique, transverse (TE). Attention: pas plane.

N.B : Dire qu'il y a TM. \\
N.B. 2 : Mode TEM pas possible dans guide creux $\Rightarrow$ câble coaxial.

\subsubsection{Équation de propagation}

MG. D'Alembert. Conditions limites.

Expliquer les étapes.

\subsubsection{Structures de champ admissibles}

Quantification des modes. Interprétation physique.

\subsubsection*{Remarque importante}

Discuter de la double quantification TE$_{nm}$. On se restreint aux modes TE$_{n0}$. Parler du mode dominant (fondamental).

N.B. : $B$ s'obtient avec MF.

\section{Étude des modes TE$_{n0}$}

\subsection{Relation de dispersion}

Passe-haute, fréquence de coupure.

\subsubsection{Vitesses de groupe et de phase}

$v_\phi > c$ mais ok. $v_g < c$. On verra son sens physique.

\subsection{Aspects énergétiques}

La vitesse de propagation de l'énergie le long du guide s'identifie à la vitesse de groupe.


\section{Câble coaxial}

\textcolor{mycolor5}{Facultatif: si vraiment il reste du temps}

[H-prépa Ondes]

\begin{itemize}
\item Impossibilité du mode TEM dans un seul conducteur
\item Nécessité de deux conducteurs: exemple du câble coaxial
\item Relation de dispersion
\item Comparaison avec une OPPM
\end{itemize}



\section*{Conclusion}
  \addcontentsline{toc}{section}{Conclusion}
  
\textcolor{mycolor5}{Le concept de guide d'onde ne se limite pas aux ondes EM}

\paragraph*{Manip' qualitative} Ondes acoustique avec et sans guidage.

  
\paragraph*{Applications}  Radars, récepteurs en radioastronomie, fibre optique, endoscopes

\section*{Description de l'expérience}
  \addcontentsline{toc}{section}{Description de l'expérience}


\textcolor{mycolor5}{TP Ondes II + notice N 311}


\begin{tcolorbox}[breakable, enhanced, colback=red!2!white,colframe=mycolor!85!black,title=\textbf{\textbf{Expérience}}]
\paragraph*{Matériel}
\begin{itemize}
\item Banc hyperfréquence (diode Gunn - isolateur - modulateur - fréquencemètre - atténuateur - ligne de mesure - court circuit variable) + les différents support, vis, etc.
\end{itemize}

\paragraph*{Protocole } 

\begin{itemize}[label=$\triangleright$]
\item Faire le montage de la notice p.8. + poly TP. \textbf{N.B.} l'atténuateur ne sert à rien.
\item Brancher la Gunn -12/0 V (\textbf{ne pas relier le +12V !}) et le modulateur (même si celui-ci n'est pas utilisé dans ce TP) au boîtier d'alimentation. 
\item Brancher le fréquencemètre vers l'oscilloscope (mettre l'échelle adéquate pour observer un signal de quelques dizaines de millivolts d'amplitude).
\item \textbf{Mesure de $f$:} On tourne la molette du fréquencemètre jusqu'à observer un max sur l'oscillo. \textbf{Important: les chiffres ne tournent pas bien, faut lire les chiffres d'en bas pour avoir la bonne valeur de $f$. Bref jouer dessus pour voir où on était avant et comment on avance.}
\item Dérégler le fréquencemètre pour maximiser la puissance transmise au guide d'onde.
\item \textbf{Mesure de la longueur d'onde:} Mode défilement sur l'oscillo. La distance entre deux minima consécutifs est $\frac{\lambda_g}{2}$.
\item La dimension intérieure du guide est donnée par $a = 22.860 \pm 0.046 mm$.
\end{itemize}

\paragraph*{Aspect quantitatif :} Mesure de  la relation de dispersion $f = c \sqrt{(\frac{1}{\lambda_g})^2 + (\frac{1}{2a})^2}$. Tracer $f^2(\frac{1}{\lambda_g^2})$. En déduire $c$ via le coefficient directeur et $a$ via l'ordonnée à l'origine.


\end{tcolorbox}



\newpage

%------------------------------------------


\chapter{Microscopies optiques.}


\paragraph*{Niveau:} 

\paragraph*{Prérequis:} 
\begin{itemize}
\item Optique géométrique
\item Théorie de la diffraction (critère de Rayleigh, transformée de Fourier, etc.)
\end{itemize}

\paragraph*{Bibliographie:}
\begin{itemize}
\item Optique: Une approche expérimentale et pratique. Houard.
\item La microscopie optique moderne. G. Wastiaux. Tec\&Doc
\item Optique: fondements et applications. Péréz.
\item Optique physique. Taillet.
\item Optique expérimentale. Sextant.
\item Dictionnaire de physique. Taillet, Villain, Febvre.
\item Images \url{https://www.nikonsmallworld.com/}
\item Vidéos : \url{https://toutestquantique.fr/microscopes/}
\item Epreuve agrégation standard 2015 : \url{https://nc.agregation-physique.org/index.php/s/XzZWHcEfQjWwD8f?path=%2F2015}
\item Bonus animation : \url{https://phyanim.sciences.univ-nantes.fr/optiqueGeo/instruments/microscope.php}
\end{itemize}

\paragraph{Notes agrégat}
\begin{itemize}
\item 2017 : L’intérêt des notions introduites doit être souligné.
\item 2016 : Une technique récente de microscopie optique à haute résolution doit être présentée.
\end{itemize}

\section*{Introduction}
  \addcontentsline{toc}{section}{Introduction}
  
Montrer des images de choses zoomés par le microscope qui sera traité à la fin. Dire que si on en est arrivés, là, c'est grâce aux développements en microscopie. Mais avant de présenter le microscope qui permet ça, commençons par les bases.
  
\section{Microscope de base}

[Houard] Historique (slide ou selon le temps juste pour les questions). Jansen père et fils (1590). Galilée (1610). Hooke (1665): microscope à 3 lentilles composémais d’une qualité optique inférieure à celui de Leeuwenhoek  (1632-1723): lentille simple mais grande résolution pour l'époque (de l'ordre du micron), l'utilisation d'une lentille simple limitant les aberrations, contrairement à Hooke. Puis Bancks (1830): microscope simple encore mais grand pouvoir de résolution (Darwin et Brown étaient ses clients !). Réputé jusqu'en 1848, puis comment à disparaitre au profit du microscope composé dont les performances deviennent comparables lorsque le problème des aberrations est enfin résolu.

\paragraph{Présentation d'un microscope optique élémentaire à champ large} [Epreuve 2015, Houard, p. 158] Montrer un vrai microscope. Nommer ses différents composants (oculaire, objectif, statif, source, etc.).


\subsection{Schéma optique simplifié}

Schéma optique avec deux lentilles. Constructions géométriques d'images et de rayons.
\begin{equation}
AB \xrightarrow[]{L_1} A_1 B_1 \xrightarrow[]{L_2} A'B'.
\end{equation}
Image intermédiaire $A_1 B_1$ agrandie et inversée. Pour obtenir un grandissement important et un encombrement réduit, on utilise $L_1$ de courte focale (qqs mm à qq cm). Pour que l'image intermédiaire soit réelle, il faut placer l'objet avant le foyer objet de l'objectif (s'en convaincre avec la loi de conjugaison de Descartes. Et juste très près juste avant pour optimiser le grandissement).\\
L'oculaire agit comme une loupe et donne de l'image intermédiaire réelle une image finale virtuelle. Lors d'une utilisation normale, $A_1B_1$ est dans le plan focal objet de l'oculaire pour une observation sans accommodation ($A'B'$ est rejetée à l'infini).

\paragraph{Expérience qualitative} 
[Houard, p. 166] Présenter le dispositif expérimental à deux lentilles. Voir en effet que l'image intermédiaire est renversée, et que l'image finale sur l'œil sera droite et agrandie (renverse l'image virtuelle renversée de l'oculaire). \\

\textbf{N.B.} C'est une microscopie à champ clair.

\subsection{Grossissement}

[Houard p. 156, TD Clément Optique géométrique]
Les performances du microscope sont caractérisés par deux grandeurs.
\begin{itemize}
\item Son grossissement
\item Sa puissance (ne pas en parler), cf. Houard.
\end{itemize}
Puisque le microscope forme d’un objet à distance finie une image à l’infini, on
cherche à calculer le grossissement commercial $G = \frac{\alpha'}{\alpha}$ avec $\alpha$ l’angle sous lequel l'œil voit $AB$ à $d_m = 25$ cm (punctum proximum) et $\alpha'$ l’angle sous lequel est vue l’image dans l’oculaire du microscope. Montrer que $G_c = \gamma_{ob} G_{c,oc}$ avec $\gamma_{ob} = \frac{A_1B_1}{AB}$ et $G_{c,oc} = \frac{\alpha'}{\alpha_{oc}} = \frac{A_1B_1}{f_2'} \frac{d_m}{A_1B_1} = \frac{d_m}{f'_2}$.

\textbf{N.B.} Lien avec la puissance intrinsèque: $G_c = P d_m = \frac{\Delta d_m}{f'_1 f'_2}$ avec $\Delta = F'_{ob} F_{oc}$ l'intervalle optique en général égal à 16cm.

\paragraph{Expérience} mesurer le grossissement commercial du microscope [TP + Sextant p.54].


\subsection{Aberrations liées aux lentilles}

[Houard, p. 125] Aberrations chromatiques et géométriques.

Corrections aux aux objectifs de microscope [Houard, p. 158] et aux oculaires [p. 161] (ne pas forcément en parler).

\paragraph{A ce stade, on ne voit a priori pas de limite à avoir de très grand grossissement. De plus, lors la
question de savoir si on est capable de voir nette toute l’épaisseur de l’objet se pose. On verra que ce n'est pas le cas, et qu'on aura une limite sur la résolution. Ce qui a nécéssité le développement de nouvelles microscopies optiques dont la confocale qui fera l'objet de la seconde partie.}


\section{Limite de résolution et microscopie confocale}

[Houard] L'objectif est la pièce maîtresse du microscope. Il doit fournir une image intermédiaire agrandie de très bonne qualité. Caractéristiques: grandissement, ouverture numérique, objectif à immersion ou non, degré de correction des aberrations chromatiques. La caractéristique la plus importante d'un objectif est elle qui conditionne le pouvoir de résolution du microscope: c'est l'ouverture numérique. Dans la suite nous allons introduire ces deux notions.


\subsection{Ouverture numérique} [Taillet] $\omega_0 = n \sin u$, où $u$ est l'angle que fait dans le milieu d'indice $n$ le rayon le plus incliné qui traverse l'objectif. 

\textbf{N.B.} Abbe montra que le pouvoir de résolution spatiale varie comme l'inverse de l'ouverture numérique. 

\subsection{Limite de résolution} c'est la distance minimale devant exister entre deux points $A$ et $B$ de la lamelle échantillon pour que leurs images à travers le microscope soient séparées.

Cette distance $AB_{\min}$ est imposée par le phénomène de diffraction. En effet, la figure de diffraction d'un trou circulaire donne la tache d'Airy. La tache centrale est limitée par le premier minimum nul d'angle vérifiant $\sin \theta \simeq \theta = 1.22 \frac{\lambda}{D_0}$ où $D_0 = 2R$ est le diamètre du trou. Le diamètre angulaire de la tache d'Airy vaut donc $\Delta \theta = 2 \theta = 1.22 \frac{\lambda}{R_0}$ et son rayon $R = \frac{\Delta \theta}{2} L = 0.61 \frac{\lambda L}{R_0}$ avec $L$ la distance d'observation (ou $f'$ si l'observation est faite dans le plan focal image d'une lentille).
 
\paragraph{Critère de Rayleigh} Deux objets ponctuels de même luminosité sont tout juste résolus si le maximum de la figure de diffraction correspond au premier minimum de la figure de diffraction de l'autre. 

\paragraph{Pour le microscope} [Houard, footnote p. 160] La tache d'airy intermédiaire en $A_1$ a un rayon $R_1 = \frac{0.61 \lambda L}{R_0} = \frac{0.61 \lambda}{u_1} \simeq \frac{0.61 \lambda_0}{n_1 \sin u_1}$. En appliquant la relation d'Abbe (cf. TD Optique géométrique): $n AB \sin u = n_1 A_1B_1 \sin u_1$, la condition 
$A_1B_{1~\min} = R_1$ donne $AB_{\min} = \frac{n_1 \sin u_1 A_1B_{1~\min}}{n \sin u} = \frac{n_1 \sin u_1 R_1}{n \sin u}$. Finalement
\begin{equation}
AB_{\min} = \frac{0.61 \lambda_0}{n \sin u} = \frac{0.61 \lambda_0}{\omega_0}.
\end{equation}

[Houard] Plus le demi-angle d'ouverture $u$ du rayon inciddent est grand, meilleure est la résolution. La limite de résolution de l'objectif conditionne celle du microscope. En effet, si l'image intermédiaire n'est pas résolue, l'image finale agrandie par l'oculaire ne le sera pas non plus.

ODG: $\lambda = 400$ nm, $n_{liq} = 1.5$ (objectif à immersion), $\omega_0 = 1.25$, on obtient $AB_{\min} = 0.2 \mu$m.

\textbf{N.B.} Le pouvoir de résolution de l'œil est d'environ une minute d'arc.

\paragraph{On vient qu'on est limité sur les détails d'un objet. Mais qu'en est-il de la résolution d'un objet le long de l'axe optique.}

\subsection{Profondeur de champ}

[Wiki] En microscopie optique à champ large, pour qu'une image soit nette, il faut que l'objet soit dans le plan focal du système optique. Lorsqu'un objet est épais, présente un relief important, ou bien lorsqu'il est incliné par rapport à l'objectif, seule une partie de l'objet est nette dans l’image.

[JF] Voir [44], p. 157 et [77], p. 70. On reste qualitatif en exprimant juste le lien entre profondeur de champ et ouverture
numérique.

[Péréz, p. 120] Lorsque l'ouverture numérique augmente, la profondeur de champ diminue. ODG : $1 \mu$m $\rightarrow$ le microscope permet de repérer avec une très grande précision la position de l'objet sur l'axe optique.


\paragraph{Transition} [\href{https://fr.wikipedia.org/wiki/Microscope_confocal}{Wiki}] Plus le grossissement est élevé, plus cette profondeur est faible, ce qui empêche d'avoir une image nette sur la totalité d'un objet un peu étendu. Ceci est particulièrement ennuyeux pour les objets allongés comme les nerfs ; ils sont donc flous sur une partie de leur trajet quelle que soit l'habileté du préparateur. C'est également un problème pour les surfaces rugueuses ou gauches, comme des faciès de rupture.
 
\subsection{Microscope confocal}

Le principe du microscope confocal a été décrit par Marvin Minsky en 19531, mais ce n’est que dans la fin des années 1980 que des modèles commerciaux sont apparus, rendant cette technique accessible à de nombreux laboratoires. La microscopie confocale est très utilisée aujourd'hui en biologie ainsi qu’en sciences des matériaux.

[Wastiaux p. 254] Il agit comme un "couteau optique" et permet d'examiner l’intérieur des structures épaisses et de "découper" l’échantillon dans plusieurs directions.


\subsubsection{Principe}

Vidéo: \url{https://fr.wikipedia.org/wiki/Microscope_confocal}

[JF] Reproduction point par point d’un fin diaphragme dans le plan de l’objet. Pour obtenir une image totale, le point lumineux balaie la surface de l’objet au moyen de miroirs. Seules les informations du plan focal parviennent au détecteur : microscope confocal.

[Wiki] Microscopie qui a la propriété de réaliser des images de très faible profondeur de champ (environ 400 nm) appelées « sections optiques ». En positionnant le plan focal de l’objectif à différents niveaux de profondeur dans l’échantillon, il est possible de réaliser des séries d’images à partir desquelles on peut obtenir une représentation tridimensionnelle de l’objet. L'objet n'est donc pas directement observé par l'utilisateur ; celui-ci voit une image recomposée par ordinateur. Le microscope confocal fonctionne en lumière réfléchie ou en fluorescence. La plupart du temps, on utilise un laser comme source de lumière.  

\subsubsection{Avantages}

[Wastiaux, p. 256]

[\href{https://fr.wikipedia.org/wiki/Microscope_%C3%A0_contraste_de_phase}{Wiki}] La résolution est légèrement meilleure, mais le point le plus important est qu'il permet de former une image de coupes transversales sans être perturbé par la lumière hors du plan focal. Il donne donc une image très nette des objets en trois dimensions. Le microscope confocal est souvent utilisé en conjonction avec la microscopie à fluorescence.

Images \url{https://www.nikonsmallworld.com/techniques/confocal} Confocal imaging involves scanning the specimen to create computer-generated optical sections down to 250 nm thickness using visible light. These optical sections may be stacked to provide a 3-D digital reconstruction of the specimen.

\subsubsection{Intérêt supplémentaire de la fluorescence}

Images \url{https://www.nikonsmallworld.com/techniques/fluorescence} Fluorescence imaging uses high intensity illumination to excite fluorescent molecules in the sample. When a molecule absorbs photons, electrons are excited to a higher energy level. As electrons ‘relax’ back to the ground-state, vibrational energy is lost and, as a result, the emission spectrum is shifted to longer wavelengths.

[JF] Ça marche super bien en fluorescence car la lumière diffusée est pratiquement éliminée. Vidéo de la mitose
d’une cellule : plus de destruction de l’échantillon, suivre l’évolution et en plus avec chaque couleur on sait qui est
qui. 


\paragraph{Transition} Et si ce qu'on veut regarder est transparent, sans effet sur l’intensité lumineuse ? Il faut alors travailler avec le seul élément optique modifié à la traversée de l’échantillon par l’onde lumineuse : la phase!


\section{Microscope à contraste de phase}

[Taillet, p. 153. Hecht, p. 634. \href{https://fr.wikipedia.org/wiki/Microscope_%C3%A0_contraste_de_phase}{wiki}]

Le contraste de phase est une technique largement utilisée qui permet de mettre en valeur les différences d'indices de réfraction comme différence de contraste. Elle a été développée par le physicien hollandais Frederik Zernike dans les années 1930 (il reçut pour cela le prix Nobel en 1953). Le noyau d'une cellule par exemple apparaîtra sombre dans le cytoplasme environnant. Le contraste est excellent, néanmoins cette technique ne peut être utilisée avec les objets épais.

\subsubsection{Principe mathématique}

Le microscope à contraste de phase est un microscope qui exploite les changements de phase d'une onde lumineuse traversant un échantillon.

Vidéo : \url{https://fr.wikipedia.org/wiki/Microscope_%C3%A0_contraste_de_phase}

\subsection{Intérêt}

Images \url{https://www.nikonsmallworld.com/techniques/phase-contrast}

[nikon] The technique is applicable to many transparent subjects, such as living cells in culture, micro-organisms, thin tissue slices, lithographic patterns, fibres, latex dispersions, glass fragments, and subcellular particles (including nuclei and other organelles) where the technique reveals structure that is not visible in brightfield imaging. An advantage of phase contrast microscopy is that living cells can be imaged in detail without the need for staining or use of fluorophores.

[TD] Les microscopes à contraste de phase sont utilisés dans les laboratoires de biologie car ils permettent d’étudier les objets vivants sans les colorer et donc sans les tuer.


\section*{Conclusion}
  \addcontentsline{toc}{section}{Conclusion}

\noindent
- Citer d'autres microscopes optiques:
\begin{itemize}
\item Microscopie en champ sombre
\item Microscope à contraste de phase
\item Microscopie à fluorescence (agreg 2015)
\item Microscopie confocale
\item Microscopies à champ proche (agreg 1998)
\end{itemize}
- Ouvrir sur d'autres types de microscopies (ex: effet tunnel, électronique. cf. agreg 2015).

[Houar] Dans un microscopique optique, il n'est pas possible de descendre en dessous de 0.2$\mu$m et le grossissement max est de l'ordre de $1000$. C'est suffisant pour observer des cellules ou des bactéries, mais insuffisant pour observer des virus. Cependant, le pouvoir de résolution étant proportionnel à $\lambda$, on peut espérer l'abaisser suffisamment en utilisant un flux de particules possédant une longueur d'onde nettement inférieure à celle des photons : c'est le principe du microscope électronique.



\section*{Description de l'expérience}
  \addcontentsline{toc}{section}{Description de l'expérience}

\textcolor{mycolor5}{TP Instruments d'optique}


\begin{tcolorbox}[breakable, enhanced, colback=red!2!white,colframe=mycolor!85!black,title=\textbf{\textbf{Expérience}}]
\paragraph*{Matériel}
\begin{itemize}
\item Microscope (ex: $\gamma_{obj} = 4$, $G_{c,oc} = 10$)
\item Support élévateur
\item Mire graduée en dixièmes de millimètre
\item Lampe quartz-iode + condenseur de courte focale (ex: 8cm)
\item Filtre antithermique (déjà installé au dessous de la platine porte-objet, sinon entre le condenseur et l'objet)
\item Lentille de grande focale (ex: 1 à 2 m, sinon improviser selon l'espace)
\item Écran dans le plan focal
\item Grande règle
\item Chronomètre
\item Pour qualitatif [Houard, p. 166]
\begin{itemize}
\item Lentilles ($f'_{ob} = 10cm, f'_{oc} = 20cm, f'_{proj} = 15cm$)
\item Objet : grille + F
\item Lampe QI
\item Grande règle
\item Écran
\item Support de lentille vide (pour le placer au plan de l'image intermédiaire issue de l'objectif)
\end{itemize} 
\end{itemize}

\paragraph*{Protocole } 

\begin{itemize}[label=$\triangleright$]
\item Placer la mire sur le microscope
		\item Faire le montage: lampe+condenseur - objet (mire) - microscope (objectif + oculaire) - lentille - écran.
		\item Mesurer $\alpha' = \frac{A'B'}{f'}$. En déduire $G_c = \frac{\alpha'}{\alpha}$ avec $\alpha = \frac{AB}{d_m}$ ($d_m$ = pronctum proximum). \textbf{N.B} Prendre plusieurs graduations pour diminuer les incertitudes.
\end{itemize}

\paragraph*{Aspect quantitatif :} Mesure du grossissment commercial du microscope $G_c =  \frac{A'B'}{f'} \frac{d_m}{AB}$ via la mesure de $A'B'$. Comparer à $G_c^{fab} = \gamma_{obj} G_{c,oc}$.


\end{tcolorbox}



%------------------------------------------

\chapter{Interférences à deux ondes en optique}


\paragraph*{Niveau:} 2ème année CPGE

\paragraph*{Prérequis:}
\begin{itemize}
\item Modèle scalaire de l'onde
\item Chemin optique, différence de marche
\item Intensité lumineuse
\item Formules trigonométriques
\end{itemize}

\paragraph*{Bibliographie:}
\begin{itemize}
\item Physique Spé MP-MP*. Olivier, Gié, Sarmant. Tec \& Doc
\item Optique expérimentale. Sextant.
\item Optique: Une approche expérimentale et pratique. Houard.
\item Tout-en-un, MP. Sanz. Dunod.
\item Eugene HECHT. Optique. Pearson, 2005
\end{itemize}

\section*{Introduction}
  \addcontentsline{toc}{section}{Introduction}


\subsubsection*{Manip' introductive :} si on superpose deux lasers, il ne se passe rien. Si on les fait passer à travers un dispositif qui élargit le faisceau + une fente source + une bifente : on voit une figure d'interférence.
  \section{Interférences à deux ondes}
  \textcolor{red}{Définition :} phénomène ondulatoire qui résulte d'une interaction entre deux ondes (lumineuses) qui produit une intensité totale qui diffère de la somme des intensités individuelles.
  \subsection{Superposition de deux ondes}
  On considère deux sources ponctuelles $S_1$ et $S_2$ et des amplitudes vibratoires $a_i(M,t)=A_i\cos\left(\omega_it-\phi_{S_i} - \frac{2\pi[S_iM]}{\lambda_0i}\right)$. L'amplitude totale est : $a(M,t)=a_1(M,t)+a_2(M,t)$. L'intensité est : $I(M,t) = <a^2(M,t)>$.\\

  En développant, on obtient :
  \begin{equation}
      I = I_1 + I_2 + I_{1,2}
  \end{equation}
  avec $I_{1,2} = 2A_1A_2\cos\left(\omega_1t-\phi_1(M)\right)\cos\left(\omega_2t-\phi_2(M)\right)$

  \subsection{Conditions d'interférence, notion de cohérence}
  \begin{itemize}
      \item $I_{1,2}\neq$, dans ce cas on dit que les ondes sont cohérentes,
      \item si $\omega_1\neq\omega_2$, $I_{1,2}=0$
  \end{itemize}

  \underline{\textcolor{red}{Condition 1 :}} deux ondes de pulsations différentes sont incohérentes.\\

  Présentation du modèle du train d'onde : paquet d'onde séparés par un temps $\tau$. Comme $\phi_{S1}$ et $\phi_{S2}$ varient aléatoirement, on obtient $I_{1,2}$ non nulles sur le détecteur si : \\
  
  \underline{\textcolor{red}{Condition 2 :}} Il faut que les deux ondes soient issus du même train d'onde. \\

  On obtient alors la formule de Fresnel :
  \begin{equation}
      I = I_1 + I_2 + 2\sqrt{I_1I_2}\cos{\Delta\phi(M)}  \end{equation}
      où $\Delta\phi(M) = \frac{2\pi([S_2M]-[S_1M])}{\lambda_0}$.

\section{Exemple d'interféromètre : les trous d'Young}
Cf photo.

\begin{equation}
    I = 2I_0\left[1+\cos\left(\frac{2\pi ax}{\lambda D}\right)\right]
\end{equation}
Succesion de franges brillantes et de franges sombres. La distance entre deux franges brillantes est appelée \textcolor{red}{interfrange} notée $i$ qui vaut ici : $i=\frac{\lambda_0D}{a}$.


\subsubsection*{Expérience} Mesure de l'interfrange de la figure d'interférences pour en déduire $a$. On mesure $a=0.16\pm0.04$mm à comparer avec la valeur $a_{fabricant}=0.2$mm.

\section{Notion de cohérence spatiale}
Effet de la largeur de la source en reprenant le problème avec deux sources séparées par une distance $b$. On obtient à l'aide des formules obtenues dans la partie précédente : 
\begin{equation}
    I_{tot} = 4I_0\left[1+\cos\left(\frac{\pi ab}{\lambda D}\right)\cos\left(\frac{2\pi a x}{\lambda D}+\frac{2\pi a b}{\lambda D}\right)\right]
\end{equation}

\paragraph{ODG}
\begin{itemize}
\item Lampe spectrale basse pression $l_c \simeq 1cm$.
\item Source de lumière blanche $l_c = c \tau_c = \frac{c}{\Delta \nu} \simeq 1 \mu m$.
\item Laser He-Ne du labo $30cm$. Peut atteindre le km ! 
\item Lire TD Clément.
\end{itemize}

\section*{Conclusion}
  \addcontentsline{toc}{section}{Conclusion}

Ouverture sur les dispositifs à division du front d'onde et division d'amplitude.




\section*{Description de l'expérience}
  \addcontentsline{toc}{section}{Description de l'expérience}



\begin{tcolorbox}[breakable, enhanced, colback=red!2!white,colframe=mycolor!85!black,title=\textbf{\textbf{Expérience}}]
\paragraph*{Matériel}
\begin{itemize}
\item Deux lasers.
\item \'Elargisseur (lentille de courte focale (ex : $f' = 5$cm + porte-lentille + support élévateur). 
\item Grande règle.
\item Fente source (facultatif?).
\item Bifente.
\item Lentille ($f' \sim 10 cm$) (facultatif?).
\item Pied à coulisse.
\end{itemize}

\paragraph*{Protocole } 

\begin{itemize}[label=$\triangleright$]
		\item $\oint$ 8.2.1 du Houard (p. 197)
\end{itemize}

\paragraph*{Aspect quantitatif :} Mesure de l'espacement entre les deux fentes à partir de la mesure de l'interfrange. 
\begin{equation} \nonumber
i = \frac{\lambda D}{a}.
\end{equation}


\end{tcolorbox}





%------------------------------------------


\chapter{Interférométrie à division d'amplitude}


\paragraph*{Niveau:} L2

\paragraph*{Prérequis:} 
\begin{itemize}
\item Optique Géométrique et trigonométrie
\item Interférences à deux ondes
\item Interférences à division du front d'onde
\item Cohérence temporelle et spatiale
\end{itemize}

\paragraph*{Bibliographie:}
\begin{itemize}
\item Optique ondulatoire, Pascal Legagneux-Piquemal, PC, MP, PSI, PT. Nathan
\item Physique Spé MP-MP*. Olivier, Gié, Sarmant, More. Tec \& Doc
\item Hprépa bleu, Optique ondulatoire 2e année: MP-PC-PSI-PT. Hachette (2004). p. 74. Brébec et al. \textbf{Attention il y a une erreur de signe
dans le théorème de localisation}
\item Houard. Optique: Une approche expérimentale et pratique.
\item Optique physique, Mauras, Presse Universitaire de France
\item Optique: fondements et applications. Pérez.
\item Étienne Thibierge \url{http://www.etienne-thibierge.fr/agreg/cplt_localisation.pdf}
\item Femto \url{https://femto-physique.fr/optique/coherence.php}
\item TD de C. Sayrin \url{http://www.lkb.upmc.fr/cqed/teaching/teachingsayrin/}.
\item Optique expérimentale. Sextant.
\item Tout-en-un, MP. Sanz. Dunod.
\end{itemize}

\paragraph{Notes agrégat}

\begin{itemize}
\item 2017 : Le candidat doit réfléchir aux conséquences du mode d'éclairage de l'interféromètre (source étendue, faisceau parallèle ou non...). Il est judicieux de ne pas se limiter à l'exemple de l'interféromètre de Michelson.
\item 2016 : La distinction entre divisions du front d'onde et d'amplitude doit être précise. Le jury
rappelle que l'utilisation d'une lame semi-réfléchissante ne conduit pas nécessairement
à une division d'amplitude.
\item 2015 : Les notions de cohérence doivent être présentées.
\item 2014 : Un interféromètre comportant une lame séparatrice n'est pas obligatoirement utilisé
en diviseur d'amplitude. La notion de cohérence et ses limites doivent être discutées.
\end{itemize}

\section*{Introduction}
  \addcontentsline{toc}{section}{Introduction}

[H-prépa, p. 74]  
  
[JF]
Les dispositifs de division du front d'onde présentent l'inconvénient d'être sensible à la perte de cohérence spatiale et contraint les utilisateurs à limiter la taille de la source, donc irrémédiablement l'intensité lumineuse envoyée dans le dispositif.


\section{Intérêt de la division d'amplitude}

\subsection{Retour sur les fentes d'Young}

\paragraph{Manip' introductive} Dans le dispositif des fentes d'Young, élargir la source et observer la perte de contraste due à la perte de cohérence spatiale.

[JF] Source ponctuelle implique interférences bien contrastée, partout dans l’espace, mais pour une source étendue,
la division par front d’onde n’est pas optimale : la division d’amplitude est nécessaire.

\subsubsection{Conséquences de l'élargissement de la source}

[H-prépa, p.77] L'éclairement augmente, le contraste diminue. Critère de visibilité : si la différence de marche varie peu à l'échelle de la longueur d'onde lorsque S balaie l'ensemble de la source étendue, le contraste de franges reste convenable.

\paragraph{Transition} On peut formaliser ce critère : c'est l'objet du théorème de localisation qui cherche à déterminer les points pour lesquels $\delta(S,M)$ dépendraient suffisamment peu de S.

\subsection{Théorème de localisation}

\url{http://www.etienne-thibierge.fr/agreg/cplt_localisation.pdf}

[Thibierge] Considérons un interféromètre tout à fait quelconque, éclairé par une source ponctuelle S, comme représenté figure 1. Les interférences sont observées en un point M de l’espace, et on note $\vec{u}_1$ et $\vec{u}_2$ les directions d’entrée dans l’interféromètre des deux rayons qui interfèrent en $M$. La différence de marche entre les deux voies s’écrit $\delta(S,M) = (SM)_2 - (SM)_1 = (SA_2) - (SA_1) - L_2 - L_1$ (stigmatisme).


\paragraph{Rappel}
\begin{itemize}
\item Si tout rayon issu d'un point A arrive, après avoir traversé le système optique, en un point A', on dit que A' est l'image de A, et le système optique est rigoureusement stigmatique vis à vis des points A et A'.
\item Condition de stigmatisme: Un système optique est stigmatique vis à vis du couple objet-image (A,A') si, et seulement si le chemin optique $\mathcal{L}(AA')$ est constant pour tous les rayons lumineux joignant A à A' à travers le système optique.
\end{itemize}

\subsubsection{Condition de non brouillage}

[Thibierge] On cherche à établir un critère de non-brouillage des interférences sous l’effet de l’élargissement de la source. [Mauras, p. 159] Non-brouillage si l'intensité en M due à S ne varie pas quand S se déplace en S', donc pas de variation de chemin optique.

\paragraph{Propriété} [TD I] De façon générale, si l’on considère un segment [AB], de vecteur directeur unitaire $\vec{u}$, sa variation de longueur engendrée par un déplacement du point B du vecteur $\vec{dB}$ et un déplacement du point A du vecteur $\vec{dB}$ est donnée par
\begin{equation}
\ud AB = \ud(\vec{u} \cdot \vec{AB}) = \vec{u} \cdot \ud \vec{AB} + \vec{AB} \cdot \ud \vec{u} = \vec{u} \cdot (\vec{\ud B} - \vec{\ud A}) + AB \vec{u} \cdot \ud\vec{u}
\end{equation}
Or $\vec{u} \cdot \ud\vec{u} = 0$:
\begin{equation}
\ud AB = \vec{u} \cdot (\vec{\ud B} - \vec{\ud A}).
\end{equation}

[\href{https://femto-physique.fr/optique/coherence.php}{femto}] Déplaçons légèrement la source de façon à l'amener en S': $\vec{\ud S} = \vec{SS'}$ Ceci implique une variation sur la distance [H-prépa] $\ud(SA) = \vec{u} \cdot \vec{SA}$. Cette opération s'accompagne d'une variation de $\delta$ qui, si le déplacement est suffisamment faible, vaut à l'ordre $1$ $\ud \delta = (\vec{u_2} - \vec{u}_1) \cdot \vec{SS'} =0$.

[Thibierge] Ce résultat est général au sens où il vaut pour n’importe quel interféromètre, mais il est issu d’un développement limité au premier ordre en $[SS']/[SM]$. Il y a deux possibilités pour que le contraste des interférences soit préservé quand la source
est élargie [Mauras]:
\begin{itemize}
\item Diviseurs de front d'onde $\u_1 - \u_2$ impose $\u_1 - \u_2 \perp \vec{SS'}$
\item Diviseurs d'amplitude $\u_1 = \u_2$
\end{itemize}

[Femto]  La première configuration impose une géométrie particulière à la source. Par exemple, dans l'expérience des trous d'Young, remplacer la source ponctuelle par une fente source horizontale quand les trous sont verticaux (et vice versa) est bénéfique puisque la condition est vérifiée: la luminosité du motif d'interférence est renforcée sans perte de contraste. 

La deuxième configuration ne porte pas sur la source mais sur l'interféromètre. En effet, la condition $\u_1 = \u_2$ n'est remplie que dans les interféromètres à division d'amplitude où un rayon incident unique est divisé en deux. A priori, tous les points M de l’espace ne permettent pas de vérifier ce critère: cette condition à un coût : le lieu des points M où interfèrent ces rayons est en général une surface précise ; on dit que les interférences sont localisées. Les interférences sont localisées au voisinage des points M qui le permettent.

\subsubsection{Énoncé du théorème}

[Femto] Les interféromètres à division d'amplitude donnent lieu à un phénomène d'interférences contrastées en présence d'une source étendue. Toutefois, l'extension de la source produit un phénomène de localisation des interférences. La surface de localisation est le lieu des intersections des rayons émergeant issus du même rayon incident partant de S.

\textbf{N.B} [Thibierge] Remarquer la modération du théorème : on parle de « pouvoir donner lieu ». Le critère de non-brouillage est en effet un résultat de premier ordre, mais rien ne dit que les ordres suivants sont toujours négligeables. Heureusement, c’est toujours le cas en pratique car on utilise des sources pas trop larges qui éclairent les interféromètres avec des incidences pas trop grandes.

\subsubsection{Intérêt}

Avec la division d'amplitude on peut augmenter la largeur de la source sans
réduire la cohérence, le prix à payer étant la localisation.


\section{Interféromètre de Michelson}

[Houard, p. 224]

\subsection{Présentation du dispositif}

Schéma (slide). Présenter le dispositif réel. Rôle de la séparatrice. Rôle de la compensatrice. Schéma de principe.

\subsection{Configuration en lame d'air}

[Te\&Doc, Sanz, H-prépa]

\subsection{Présentation}

Définition. Construire le schéma équivalent en suivant Tec\&Doc, p. 96.

\subsection{Franges d'interférence}

\textcolor{mycolor5}{Illustrer avec les manip' à chaque fois que c'est pertinent}

Localisation à l'infini pour une source étendue. \textbf{Interprétation qualitative de la localisation} [Tec\&Doc + TD II + H-prépa]. Franges d'égales inclinaison, différence de marche. Ordre d'interférence. Rayon des anneaux.

Calcul de la différence de marche [TD II] + adapter [Pérez, p. 321] en réflexion:
\begin{align*}
\delta &= n ([ABC] - [AD]) \\
[ABC] &= 2 AB = 2 \frac{e}{\cos \theta} \\
[AD] &= AC \sin i = n AC \sin \theta = 2n e \tan \theta \sin \theta \\
\delta &= 2 n e\left(\frac{1}{\cos \theta} - \tan \theta \sin \theta\right) = 2 n e \cos \theta
\end{align*}

\textbf{N.B.} [Piquemal] Pour l’interféromètre de Michelson en source étendue, la surface de localisation est
l’ensemble des points d’intersection des deux émergents correspondant au même
incident primitif.

\subsubsection{Application: Mesure du doublet du sodium}

[Tec\&Doc]

Faire les calculs proprement: insister sur l'incohérence (lumière non monochromatique) et comment ça mène vers le brouillage.

\paragraph{Expérience}  L'écart $\Delta e$ entre deux anticoïncidences est donné par :

\begin{equation}
      \Delta e = \frac{\bar{{\lambda}}}{2\Delta\lambda^2}
\end{equation}
avec $\Delta\lambda =|\lambda_2-\lambda_1|$ et $\bar{{\lambda}}=\frac{\lambda_1+\lambda_2}{2}$.

\subsection{Configuration en coin d'air}

\subsubsection{Présentation}

Définition. Schéma équivalent.

\subsection{Franges d'interférence}

\textcolor{mycolor5}{Les montrer expérimentalement}

Localisation au voisinage des miroirs pour une source infini. Différence de marche. Franges d'égale épaisseur.

\section{Interféromètre de Fabry-Perrot}

Si le temps, mais peu probable. Lire le Houard pour les questions. 

C'est un interféromètre à ondes multiples constitué de deux lames de verre à faces parallèles.

\section*{Conclusion}
  \addcontentsline{toc}{section}{Conclusion}

Selon la maîtrise: 
\begin{itemize}
\item Il y a d'autres interféromètre à division d'amplitude: Fabry-Perrot (ondes multiples), Mach-Zehnder et Sagnac, détecteur d'ondes gravitationnelles (Virgo ~ Michelson)
\item Applications [Houard]
	\begin{itemize}
	\item Lame d'air: métrologie, spectroscopie interférentielle
	\item Coin d'air: détection de défauts ou perturbation: contrôle polissage de miroirs, mesure de l'épaisseur d'une lame mince, indice de réfraction d'un gaz.
	\item Fabry-Perrot: cavité laser, traitement anti-reflet, filtre interférentiel, spectrométrie interférentelle en astronomie
	\item Mach-Zehnder et Sagnac: mesure de faibles vitesses angulaires
	\end{itemize}
\end{itemize}



\section*{Description de l'expérience}
  \addcontentsline{toc}{section}{Description de l'expérience}


\textcolor{mycolor5}{TP Interférences}

[Houard]


\begin{tcolorbox}[breakable, enhanced, colback=red!2!white,colframe=mycolor!85!black,title=\textbf{\textbf{Expérience}}]
\paragraph*{Matériel}
\paragraph*{Manip' introductive}
\begin{itemize}
\item Lampe à vapeur de sodium
\item Bifente (Young)
\item Fente source réglable
\item Filtre interférentiel (jaune)
\item Lentille $f'_1 = 15cm$ 
\item Lentille $f'_2 = $
\item Écran
\end{itemize}
\paragraph*{Doublet du sodium}
\begin{itemize}
\item Laser He-Ne
\item Lentille de courte focale (ex: $f' = 5mm$)
\item Porte lentille + support élévateur
\item Lentille de grande focale (selon la distance à l'écran souhaitée)
\item Michelson en lame d'air
\item Lampe à vapeur de sodium
\item Condenseur (lentille) de 70 mm 
\item Lentille convergente $f' = 15cm$
\item Pied à coulisse
\item Grande règle et/ou mètre
\item Écran
\item Chronomètre
\end{itemize}

\paragraph*{Protocole } 

\paragraph*{Manip' 1}

\begin{itemize}[label=$\triangleright$]
		\item $\oint$ 8.3.6 du Houard (p. 208)
\end{itemize}

\paragraph*{Manip' 2}

\begin{itemize}[label=$\triangleright$]
		\item $\oint$ 8.5.5.b du Houard (p. 230)
		\item Régler le Michelson au laser en suivant les indications du TP interférences. Trouver la teinte plate
		\item Mettre la lampe à sodium et se placer en lame d'air en chariotant.
		\item Lecture sur le verrier : 
		\begin{itemize}
		\item On lit en haut. Ex: 25mm
		\item On regarde les petits traits d'en bas. Si on voit le petit trait, c'est qu'on a dépassé 0.50. Ex: 25.50
		\item Enfin, on regarde où coïncide la grande ligne avec le trait de la vis pour avoir la virgule. Ex: 43 donne 25.50 + 0.43 = 25.93.
		\end{itemize}
		\item Lire sur le verrier les positions successives entre deux anticoïncidences et calculer $\Delta e$
		\item Voir sur la notice les valeurs attendues
\end{itemize}

\paragraph*{Aspect quantitatif :} Mesure du doublet du sodium : $\Delta \lambda = \frac{\lambda^2}{2 \Delta e}$ avec $\lambda = 189$ nm. J'ai trouvé en faisant la manip' avec la lampe de Montrouge $x_ = 27.72$ mm et $x_2 = 27.43$ mm.


\end{tcolorbox}



\newpage

%------------------------------------------


\chapter{Diffraction de Fraunhofer}


\paragraph*{Niveau:} L3
\paragraph*{Prérequis:} 
\begin{itemize}
\item Optique ondulatoire (amplitude, chemin optique, notation complexe, etc.)
\item Interférences
\item Transformées de Fourier
\item Trigonométrie
\end{itemize}

\paragraph*{Bibliographie:}
\begin{itemize}
\item H-prépa bleu. Optique ondulatoire 2e année: MP-PC-PSI-PT. Brébec et al.
\item Physique Spé MP. Tec\&Doc. G.S.O.M.
\item Optique ondulatoire, Pascal Legagneux-Piquemal, PC, MP, PSI, PT. Nathan
\item Optique. Sylvain Houard. de boeck.
\item Dictionnaire de physique, Taillet
\item Sylvio Rossetti : \url{https://perso.ens-lyon.fr/sylvio.rossetti/AGREG/LP/LP35_Diffraction%20de%20Fraunhofer/LP35_Diffraction_de_Fraunhofer.pdf}
\item Femto : \url{https://femto-physique.fr/optique/diffraction-de-fresnel.php}
\item En bonus: Optique, Pérez.
\item En bonus: Optique, Hecht.
\item En bonus: TD de C. Sayrin \url{http://www.lkb.upmc.fr/cqed/teaching/teachingsayrin/}.
\end{itemize}

\paragraph{Notes agrégat}
\begin{itemize}
\item 2017 : Les conditions de Fraunhofer et leurs conséquences doivent être présentées, ainsi que
le lien entre les dimensions caractéristiques d’un objet diffractant et celles de sa figure
de diffraction.
\item Jusqu’en 2013, le titre était : Diffraction de Fraunhofer. Applications.
\item 2014, 2013, 2012, 2011 : Les conditions de l’approximation de Fraunhofer doivent être
clairement énoncées. Pour autant, elles ne constituent pas le coeur de la leçon.
\end{itemize}

\section*{Introduction}
  \addcontentsline{toc}{section}{Introduction}

[H-prépa] 

\paragraph*{Expérience introductive} Fermer progressivement une fente devant un laser.

[JF] Comme le faisceau est parallèle en sortie du laser, on obtient une tache de largeur a sur l’écran (ok optique géo). Quand on ferme la fente, a diminue : la tache sur l’écran s’élargie et on voit apparaître une tache centrale et des taches secondaires : DIFFRACTION. On est plus dans la limite de l’optique géométrique : on doit considérer l’aspect ondulatoire de la lumière. Avec quels principes expliquer ce phénomène?

\section{Phénomène de diffraction}

\subsection{Définition}

Déf., critère, ordre de grandeur.

[Piquemal] Ce phénomène a lieu dans tous les domaines de la physique, pour toutes les ondes de longueur d’onde arrivant sur un objet dont l’échelle typique d de variations spatiales est du même ordre de grandeur. Si on veut explorer un cristal dont la maille est de l’ordre de quelques 0,1 nm, alors il faut disposer d’une onde de longueur d’onde C’est pourquoi on utilise des rayons X.


\subsection{Principe d'Huygens-Frensel}

[Tec\&Doc, H-prépa]


\subsubsection{Énoncé}

Surface diffractance plane. Faire le schéma avec les 3 repères. Ampltitude complexe instantanée $a(M,t) = a_0(M) \e^{i \omega t}$. On va s'intéresser à l'amplitude complexe tout court (on zappe le $ \e^{i \omega t}$).

[Clément] Chaque point M d’une surface $\Sigma$ atteinte par la lumière peut être considérée comme une source secondaire émettant une onde sphérique. L’état vibratoire de cette source secondaire est proportionnel à celui de l’onde incidente en M et à l’élément de surface $\ud \Sigma$ entourant le point M. \\
Les vibrations issues des différentes sources secondaires interfèrent entre elles.	



\subsubsection{Discussion}

[Tec\&Doc, p. 139] \\

[Sylvio] En réalité, il n’y a pas de source secondaire, c’est un résultat mathématique qui dit que c’est comme s’il y avait
des sources secondaires.

[Femto] \url{https://femto-physique.fr/optique/diffraction-de-fresnel.php}


\subsubsection{Facteur de transmission}

[H-prépa] Définir et donner des exemples.

[Houard, p. 31]


\subsubsection{Expression mathématique du principe}  [Tec\&Doc, H-prépa]

Faire le schéma avec les 3 repères.L'amplitude issue de $P \in$ pupille au point $M$:

\begin{equation}
\ud a_P(M) = K t(P) a_S(P) \frac{\e^{-i k PM}}{PM} \ud \Sigma.
\end{equation}
avec $a_S(P) = \frac{A_0}{SP} \e^{- i k_0 SP}$ l'amplitude qu'aurait en $P$ l'onde incidente en l'absence de diffraction. Finalement
\begin{equation}
a_P(M) = K A_0 \iint t(P) a_S(P) \frac{\e^{-i k PM}}{PM} \ud \Sigma = K A_0 \iint t(P) \frac{\e^{- i k SP}}{SP} \frac{\e^{-i k PM}}{PM} \ud \Sigma
\end{equation}

[TD, Hprépa]
\paragraph{En pratique} Les dimensions transverses de l'expérience (taille de la pupille, zone d'observation) sont petites devant les dimensions longitudinales (distance objet-écran $D$ et source-objet $d$): $x,y << d,D$ (petite pupille), $X,Y << D$ (petits angles). On peut On peut alors faire l’approximation $PM \simeq D$ dans la norme $\frac{a_S(P)}{PM}$ : la norme ne varie significativement que si les variations de PM sont de l’ordre de D. En revanche, pour évaluer la phase $\varphi = k PM$, il faut tenir compte des variations de PM à l’échelle de la longueur d’onde, idem pour $SP \simeq d$ dans le facteur:

\begin{equation}
a_P(M) = K' \iint t(P) \e^{-i k SP} \e^{-i k PM}\ud \Sigma
\end{equation}
avec 
\begin{align}
PM &= D \left[1 + (\frac{x-X}{D})^2 + (\frac{y-Y}{D})^2\right]^{\frac{1}{2}} \\
SP &= d \left[1 + (\frac{x_0-X}{D})^2 + (\frac{y_0-Y}{d})^2\right]^{\frac{1}{2}}
\end{align}


\paragraph{Transition} On va s'intéresser à ce que devient cette expression dans des conditions particulières dites de Fraunhoffer.

\section{Diffraction de Fraunhofer}

[Tec\&Doc, Hprépa, TD]

\subsection{Conditions}

[Tec\&Doc, p. 140] S et M sont à l'infini.

\subsection{Amplitude et intensité diffractée par une pupille plane}

Faire le schéma de [Hprépa, p. 180]
$(PM) = (OM) - \vec{OP} \cdot \vec{u}_d$. Terme de phase global on s'en fout (n'a pas d'influence sur l'éclairement). On fait pareil avec $(SP) =(SO) +  \cdot \vec{u}_i$. Finalement:
\begin{equation}
a_P(M) = K' \iint t(P) \e^{-i k (\vec{u}_d-\vec{u}_i) \cdot \vec{OP}} \ud \Sigma
\end{equation}

\textbf{On reconnait la transformée de Fourier du facteur de transmission.}

[Houard, p. 329] L'ampltitude de l'onde représente le spectre en fréquences spatiales qui décrivent via la transmittance la structure de l'objet.

[TD] On a une relation de transformée de Fourier entre t(x, y) et le profil d’intensité diffracté dans le cadre de la diffraction de Fraunhofer. On appelle plan de Fourier le plan où l’on observe cette figure de diffraction.

\subsection{Réalisation pratique}

[Houard, Tec\&Doc, Hprépa, Femto] Exemple: montage à deux lentilles: $u_{d,x} \simeq \frac{x}{f'}$, etc.

\paragraph{Montage à une lentille} (si le temps, sinon tant pis).

\paragraph{Transition} On a maintenant tous les outils pour comprendre la diffraction, on va s’intéresser aux figures de quelques objets.


\section{Quelques exemples de pupilles}

\subsection{Diffraction par une fente rectangulaire}

Calcul : $sinc$. Discussion [Hprépa, Tec\&Doc]


\subsubsection{Expérience quantitative} Mesure de la largeur d'une fente avec la caméra CCD.

\subsection{Diffraction par un diaphragme circulaire}

\url{https://web-labosims.org/animations/diffraction/diffraction.html}

Raisonner avec un cercle dans un carré [TD, Hprépa]. 

Si pas le temps, le montrer sans faire calcul. 
Tache d'Airy. Diamètre $1.22 \frac{\lambda}{R}$.

\subsection{Propriétés de la diffraction de Fraunhofer}

\textbf{Ne pas tout faire si pas beaucoup de temps}

\textcolor{mycolor5}{A chaque fois, illustrer expérimentalement et/ou animation \url{https://phet.colorado.edu/sims/html/wave-interference/latest/wave-interference_en.html}}

[Piquemal] Tous les résultats suivants découlent des propriétés de la transformée de Fourier.

\subsubsection{Déplacement de la source}

[H-prépa, p. 188] 


\subsubsection{Translation}

[Piquemal] La figure de diffraction est inchangée par translation de la pupille dans son plan. $t'(X, Y) = t(X-u, Y-v)$.

Ce résultat nous sera très utile pour toutes les pupilles formées de plusieurs motifs identiques : les deux trous d’Young, les deux fentes d’Young et les réseaux.

[Pérez, p. 273]

\subsubsection{Dilatation}

[Piquemal] Toute dilatation de la pupille dans une direction se traduit par une contraction de la figure de diffraction dans la même direction et toute contraction de la pupille dans une direction se traduit par une dilatation de la figure de diffraction dans la même direction.


\subsubsection{Rotation}

[Piquemal, Hprépa p. 191] Pupille et figure de diffraction forment un bloc pour une rotation autour de l’axe optique. 

Illustration : \url{https://phet.colorado.edu/sims/html/wave-interference/latest/wave-interference_all.html?locale=fr}

\textbf{N.B.} : Rotation de la pupille de $\theta_0$ autour de l'axe optique. Il faut utiliser les coordonnées polaires dans le plan de la pupille: $k \cdot OP = k_x r \cos \theta + k_y r \sin \theta$. $t'(r,\theta) = t(X,\theta-\theta_0)$. On trouve $a'(k_x,k_y) = a(K_x,K_y)$ avec $K_x = k_x \cos \theta_0$ et $K_y = k_y \sin \theta_0$.

La figure de diffraction tourne du même angle dans le plan de l’écran. Pupille et figure de diffraction forment un bloc pour une rotation autour de l’axe optique.




\subsubsection{Théorème de Babinet}

[Tec\&Doc, p. 150]

[Piquemal] La figure de diffraction de deux pupilles complémentaires est la même sauf en l’image géométrique de la source.

\section{Application}

Si le temps en section. Sinon en CCL.

\subsection{Pouvoir séparateur}

Limitation du pouvoir séparateur, critère de Rayleigh [Houard, p. 307. Hprépa, p. 186].

[Tec\&Doc, p. ]

\subsection{Filtrage spatial}

[Hprépa, p. 199]

\section*{Conclusion}
  \addcontentsline{toc}{section}{Conclusion}


Inconvénients: limite les appareils. Avantages: applications.

Conclure sur quelques applications (diffraction par les réseaux, holographie, analyse du speckle).

[Tec\&Doc, p. 151] Universalité (ex: acoustique, diffraction de rayon X par les réseaux). 

\textbf{N.B.} Lire les TDs de Clément pour les questions.

\section*{Description de l'expérience}
  \addcontentsline{toc}{section}{Description de l'expérience}

\textbf{Houard, p. 303}


\begin{tcolorbox}[breakable, enhanced, colback=red!2!white,colframe=mycolor!85!black,title=\textbf{\textbf{Expérience}}]
\paragraph*{Matériel}
\begin{itemize}
\item Laser
\item Fente réglable
\item Écran
\item Fente de largeur bien connue
\item Caméra CCD + densité (ex: d = 3 ou 4)
\item Porte-lentille
\item Support élévateur
\item Lentille CV de courte focale (ex: $f'_0 = 5mm$) pour avoir une source ponctuelle
\item Porte lentille + support élévateur
\item Deux autres lentilles CV (ex: $f'_1 = 10 f'_0 = 5 cm$ pour avoir des rayons à l'infini et $f'_2 = 15 cm$ pour converger vers la caméra
\item Écran
\item Scotch noir
\item Grande règle/mètre
\item Pied à coulisse
\item Trou circulaire très petit
\item Chronomètre
\end{itemize}

\paragraph*{Protocole } 

\begin{itemize}[label=$\triangleright$]
		\item Montage: source ponctuelle (laser + lentille de courte focale) $\rightarrow \mathcal{L}_1 \rightarrow $ Fente calibrée $\rightarrow \mathcal{L}_2 \rightarrow$ écran ou CCD dans le plan focale de $\mathcal{L}_2$
		\item Utiliser la caméra CCD avec le logiciel associé. Exporter les données
		\item Convertir pixel en $\mu 
		m$. Pour la caméra Mightex, largeur d'un pixel = $8 \mu m$.
		\item Importer sur Qtiplot. Centrer le pic maximal en zéro et enlever l'offset en translatant la courbe à $0$
		\item Au choix:
		\begin{itemize}
		\item Faire un ajustement en $\mathrm{sinc}^2(Ax)$ avec $A = \frac{\pi a}{\lambda f'_2}$
		\item Avec les curseurs, mesurer la distance entre les deux premières annulations de part et d'autre du pic $\Delta x = \frac{2 \lambda f'_2}{a}$
		\item Si vraiment cata, le faire en mode shlague en mesurant au pied à coulisse/règle la largeur du tiret central sur l'écran
		\end{itemize}
\end{itemize}

\paragraph*{Aspect quantitatif :} Mesure de la largeur de la fente.


\end{tcolorbox}



\newpage

%------------------------------------------


\chapter{Diffraction par des structures périodiques}


\paragraph*{Niveau:} L3
\paragraph*{Prérequis:} 
\begin{itemize}
\item Interférences
\item Diffraction de Fraunhofer (TF, diffraction par une fente, théorème de Babinet)
\end{itemize}

\paragraph*{Bibliographie:} 
\begin{itemize}
\item Optique ondulatoire, Pascal Legagneux-Piquemal, PC, MP, PSI, PT. Nathan
\item  Tec\&Doc Physique Spé MP. Gié et al.
\item H-prépa bleu. Optique ondulatoire 2e année: MP-PC-PSI-PT. Brébec et al.
\item Optique. Sylvain Houard. de boeck.
\item Optique expérimentale. Sextant.
\item Neil W. ASHCROFT. Physique des solides. EDP Sciences, 2002
\item OU: Physique des électrons dans les solides I. Henri Alloul
\item TD de C. Sayrin, Diffraction I \& II.
\item Images: \url{http://culturesciencesphysique.ens-lyon.fr/ressource/simu-diffraction.xml#p6}
\item Animation (désordre) : \url{https://phet.colorado.edu/sims/html/wave-interference/latest/wave-interference_all.html?locale=fr}
\item Animation : \url{https://femto-physique.fr/simulations/reseaux-construction-de-fresnel.php}
\end{itemize}

\paragraph{Notes Agrégats}

\begin{itemize}
\item 2017 : Il faut traiter de diffraction par des structures périodiques et pas seulement d'interférences
à N ondes.
\item 2015 : Il est important de bien mettre en évidence les différentes longueurs caractéristiques
en jeu.
\item 2014, 2013, 2012 : Cette leçon donne souvent l'occasion de présenter les travaux de Bragg ;
malheureusement, les ordres de grandeur dans différents domaines ne sont pas toujours
maîtrisés.
\item 2010, 2009 : La notion de facteur de forme peut être introduite sur un exemple simple.
L'influence du nombre d'éléments diffractants doit être discutée.
\end{itemize}

\section*{Introduction}
  \addcontentsline{toc}{section}{Introduction}

\paragraph{Vidéo CD} \url{https://www.youtube.com/watch?v=F2vpp3weFcc}

Un CD sépare la lumière. Pourquoi ? Quel est phénomène en jeu ? Diffraction sur une structure périodique ! \\

\textbf{N.B.} [Wiki] La technique du disque compact repose sur une méthode optique : un faisceau de lumière cohérente (laser) vient frapper le disque en rotation. Les irrégularités (appelées « pits », cavités dont la longueur varie entre 0,833 et 3,56 $\mu$m, et dont la largeur est de 0,6 $\mu$m) dans la surface réfléchissante de celui-ci produisent des variations binaires.

Ordinary pressed CD and DVD media are every-day examples of diffraction gratings and can be used to demonstrate the effect by reflecting sunlight off them onto a white wall. This is a side effect of their manufacture, as one surface of a CD has many small pits in the plastic, arranged in a spiral; that surface has a thin layer of metal applied to make the pits more visible. The structure of a DVD is optically similar, although it may have more than one pitted surface, and all pitted surfaces are inside the disc.


\section{Diffraction par un ensemble de structures}

[TD C. Sayrin, Tec\&Doc]

On se place dans le cadre de la diffraction de Fraunhoffer.

\subsection{Diffraction par un ensemble de structures}

[Tec\&Doc, p. 152 + TD Diffraction I]

Ensemble de $N$ structures diffractantes \textbf{identiques}. On éclaire cet objet par une onde plane monochromatique. L'amplitude diffractée est $a(M) = K A_0 \iint t(P) \frac{\e^{-i (k-k_0) \cdot OP}}{PM}$. On a $t(P) = \sum_n t_n(P)$ et $OP = OO_n + O_nP$. On injecte dans l'intégrale. $t_n$ ne dépend que de $O_nP$ car motifs identiques et $O_iP$ indép' de $i$ car structures identiques.  $a(M) = a_{1,motif}(M) \sum_n \e^{-i (k-k_0) \cdot OO_n}$.

[TD] La figure de diffraction obtenue est le produit d’un \textbf{facteur de structure}, qui ne dépend que de la répartition des structures sur l’écran diffractant, et d’un \textbf{facteur de forme}, qui ne dépend que de la forme d’une structure unique.

\paragraph{Eclairement} On prend la norme au carré. Produit de l'éclairement diffracté par un motif et de la fonction d'interférence.

\paragraph{Application} Diffraction de poudre. [TD] Cette situation est particulièrement utile quand l’on cherche, par exemple, à connaître le rayon moyen des grains d’une poudre. En répartissant de façon aléatoire les grains sur une plaque, la figure de diffraction obtenue pour l’ensemble des grains est la figure de diffraction d’un grain de diamètre moyen. En montage, on peut par exemple mesurer le diamètre de spores de lycopode par cette méthode. \\

\textbf{N.B.} Il faut décompose $OP = OO_i + O_iP$ pour que dans l'amplitude diffractée pour un seul motif, on ait bien un point de référence $O_i$ qui fasse partie de la pupille.


\paragraph{Transition} On distingue deux cas.


\subsection{Structures réparties de façon aléatoire}

[TD] lorsque le nombre de motifs devient grand, le terme de structure vaut $N$. Si on a N motifs répartis aléatoirement, on obtient la figure de diffraction d’un seul motif mais N fois plus intense qu’avec un motif unique.

\textbf{N.B.} Somme infinie d'exponentielles complexes nulle : On peut le voir en prenant la partie réelle (éclairement réel). Somme de cosinus (variable aléatoire), ça revient à prendre une moyenne dans la limite "thermo". Ça donne 0. On peut aussi le voir avec une somme de termes qui se compensent. Lire \url{https://femto-physique.fr/optique/interference-a-N-ondes.php}.


\subsection{Structures périodiques : le réseau}


Description d'un réseau.

\subsubsection{Eclairement}

[Hprépa, p. 217] Calcul de l'éclairement. $I(M) = I_0 \left(\frac{\sin \frac{N \varphi}{2}}{\frac{\varphi}{2}}\right)^2$.

On montre la simulation \url{https://phet.colorado.edu/sims/html/wave-interference/latest/wave-interference_all.html?locale=fr}. Jouer sur le désordre. 


\subsubsection{Intérêt}

\url{https://femto-physique.fr/simulations/reseaux-construction-de-fresnel.php}

[Piquemal] On constate d'une part que les franges très brillantes sont toujours situées au même
endroit (figures 1 et 2), et que les
franges brillantes
des interférences à deux ondes
deviennent de plus en plus fines
et, d'autre part, que des franges beaucoup moins brillantes apparaissent entre les franges très brillantes. Quand N
augmente, ces dernières disparaissent et ils ne restent plus que des pics fins correspondant aux interférences à deux
ondes entre deux motifs consécutifs. L'
effet des interférences à N ondes cohérentes
est essentiellement directif : l'intensité n'est maximale que dans certaines directions et quasi nulle ailleurs.

L'intérêt est évident en spectroscopie: si la source contient deux longueurs d'onde très
proches, elles seront d'autant mieux séparées que les franges brillantes correspondant à
chacune d'entre elles seront fines.


\paragraph{Transition} On va voir comment on peut utiliser cette étude préliminaire pour obtenir des informations sur la source dans un premier temps, puis sur l'objet diffractant dans un second temps.

\section{Étude spectrale de la source}


(Pour info: Deux types: par réflexion et par transmission. Aussi: d'amplitude et de phase.)


\subsubsection{Formule des réseaux}

[Tec\&Doc, p. 162. Hprépa, p. 221. TD]

Formule des réseaux (par transmission). Ordre de diffraction.


\subsubsection{Expérience} 
Détermination du spectre d'une lampe à vapeur de mercure [Houard, p. 320 et p.340].

\subsubsection{Retour sur le CD}

[Hprépa, p. 230]


\section{Étude de la structure diffractante}

[TP Sayrin, Physique du solide, lire aussi Alloul]

\url{http://www.lkb.upmc.fr/cqed/wp-content/uploads/sites/14/2020/10/optique_TD_diffraction_2_corrige.pdf}

Les atomes ou molécules d'un cristal sont ordonnés selon un arrangement régulier de
motifs en 3 dimensions, formant ainsi un cas particulier de réseau.


\subsection{Nécessité des rayons X}

[Tec\&Doc, p. 152 et p. 476].

[Piquemal] Ce phénomène a lieu dans tous les domaines de la physique, pour toutes les ondes de longueur d’onde arrivant sur un objet dont l’échelle typique d de variations spatiales est du même ordre de grandeur. Si on veut explorer un cristal dont la maille est de l’ordre de quelques 0,1 nm, alors il faut disposer d’une onde de longueur d’onde C’est pourquoi on utilise des rayons X.

On peut donc retenir comme critère : la diffraction est perceptible pour $d$ inférieur à la centaine de d étant une longueur caractéristique des variations spatiales de l'objet.

\subsection{Loi de Bragg}

[Piquemal, p.189, TD Clément, Ashcroft, p. 112]

On suppose que les plans parallèles d'atomes présents dans le cristal agissent
comme des miroirs semi-réfléchissants, chaque plan ne réfléchissant qu'une partie du rayonnement incident.

Quelle est la condition, appelée Loi de Bragg, sur la distance
$d$ entre les plans, la longueur d'onde $\lambda$ du rayonnement et son angle d'incidence $\theta$ sur
les plans considérés pour observer une amplitude diffractée non nulle ?

Faire le calcul de la différence de marche pour le réseau de Bragg: On simplifie le problème en considérant que le cristal est formé de plans équidistants, et que ces plans réfléchissent la lumière. Le modèle de Bragg est un modèle simple qui permet de comprendre ce qu'il se passe. Deux plans successifs forment alors un système équivalent à une lame d'air. 

On note I et J les points d'intersection entre les deux rayons et les plans, H et K les points projetés (pour utiliser Malus). La différence de marche: $\delta = HK + JK = 2d \sin \theta$. On a donc, pour les rayons lumineux réfléchis entre deux plans successifs, une différence de marche 
\begin{equation}
\delta = 2 d \sin \theta.
\end{equation}



\subsection{Condition de Laue [facultatif]}

[Ashcroft, TD]

\subsection{Exploitation}

La répartition des pics de diffraction permet de remonter à d. Pour le mesurer,
on peut soit faire varier la longueur d'onde $\lambda$, soit l'angle d'incidence $\theta$ (méthode du cristal tournant, méthode des poudres).

\paragraph{Méthode du cristal tournant} Cette méthode utilise des rayons X
monochromatiques, mais permet la variation de l'angle d'incidence. En pratique, la direction du faisceau de rayons X est maintenue fixe, et c'est l'orientation du cristal qui varie. Dans la méthode du cristal tournant, le cristal tourne autour d'un axe fixe, et tous les pics de Bragg qui apparaissent pendant la rotation sont enregistrés sur une pellicule.

\paragraph{Diffraction de poudre} Un diffractomètre produit des ondes à une fréquence déterminée, qui est donnée par sa source. La source sont souvent des rayons X, parce qu'ils sont le seul type d'ondes avec une fréquence correcte pour l'inter-diffraction de l'échelle atomique. Cependant, les électrons et les neutrons sont également des sources communes, avec leur fréquence déterminée par l'Hypothèse de De Broglie. Lorsque ces ondes atteignent la cible, les atomes de la poudre analysé agissent comme un réseau de diffraction, produisant des points lumineux à des angles particuliers. En mesurant l'angle où se produisent ces points lumineux, l'espacement du réseau de diffraction peut être déterminé par la loi de Bragg. 

\section*{Conclusion}
  \addcontentsline{toc}{section}{Conclusion}
  
On peut étudier d'autres structures dans la nature comme l'aile du papillon morpho. 

\begin{itemize}
\item \url{https://www.youtube.com/watch?v=M-cwYrRcCAE}
\item \url{https://en.wikipedia.org/wiki/Morpho_(genus)}
\item \url{https://web.archive.org/web/20110716081623/http://newton.ex.ac.uk/research/emag/butterflies/pdf/Vukusic_Morpho.pdf}
\end{itemize}

La diffraction n'a pas que des inconvénients. Elle présente un certain nombres d'applications [Houard]: spectroscopie en astronomie, filtrage spatial (ex: strioscopie), microscopie électronique en transmission ou à balayage (diffraction d'électrons).



\section*{Description de l'expérience}
  \addcontentsline{toc}{section}{Description de l'expérience}

[Houard p. 320/340. Sextant, p.4]

\begin{tcolorbox}[breakable, enhanced, colback=red!2!white,colframe=mycolor!85!black,title=\textbf{\textbf{Expérience}}]
\paragraph*{Matériel}
\begin{itemize}
\item Lampe à vapeur de mercure (basse pression)
\item Fente réglable
\item Diaphragme circulaire
\item Condenseur (lentille de f'=5cm ou 7cm) [pour augmenter la luminosité, facultative]
\item Réseau 3000 traits/cm
\item Lentille achromatique f'=20cm
\item Écran (lourd si possible)
\item Feuilles blanches 
\item Règle > 1m
\item Petite règle
\item Scotch
\end{itemize}

\paragraph*{Protocole } 

\begin{itemize}[label=$\triangleright$]
		\item Faire le montage du Houard. N.B.
		\begin{itemize}
		\item Distance écran-réseau D = 1m
		\item Fraunhoffer à une lentille : conjuguer par la lentille l'écran et la fente source
		\end{itemize}
		\item Prendre une feuille blanche et tracer les différentes raies afin de les mesurer à la règle
		\item Tableau récapitulatif avec les $\lambda$ mesurée et tabulées.
\end{itemize}

\paragraph*{Aspect quantitatif :} Mesure des longueurs d'ondes associées aux raies. Voir notice et Sextant (UV, violet, violet, vert, doublet jaune) pour les valeurs et Houard pour les calculs: $\lambda \simeq \frac{d y_k}{D}$ avec $d = \frac{1}{3000} cm$ le pas du réseau et $y_k$ la distance entre la raie à l'ordre 1 et la raie blanche de l'ordre 0. Hypothèses:
\begin{itemize}
\item Incidence quasi-normal $\theta_ i \simeq 0$
\item $\tan \theta = \frac{y_k}{D} \simeq \theta  \simeq \sin \theta$. Rq: on pourrait le faire exactement avec $\sin$ et $\tan$, surtout si $\frac{y_k}{D}$ est pas si petit (vérifier l'hypothèse avant). 
\end{itemize}


\end{tcolorbox}


\newpage

%------------------------------------------


\chapter{Absorption et émission de la lumière}


\paragraph*{Niveau:} 2ème année CPGE

\paragraph*{Prérequis:} 
\begin{itemize}
\item Modèle du corps noir
\end{itemize}

\paragraph*{Bibliographie:}
\begin{itemize}
\item Dictionnaire de physique. Taillet.
\item Physique atomique 1. Atomes et rayonnement. Cagnac, Tchang-Brilley et Pebay-Péroula. DUNOD
\item Tout-en-un PC. Sanz. Dunod.
\item Corps noir: Tec\&Doc MP. Gié et al.
\item Les nouveaux précis Tout-en-un PC. Bréal. Tisserand, Brendels et al.
\item Optique. Houard.
\item Physique PC. Pascal Olive. ellipses. p. 783.
\item \url{https://perso.ens-lyon.fr/sylvio.rossetti/AGREG/LP/LP37_Absorption%20et%20emmission%20de%20la%20lumi%e8re/LP37_Absorption_et_emmission_de_la_lumi_re.pdf}
\item Bonus important: Optique. Hecht. 
\end{itemize}

\paragraph{Notes agrégat}

\begin{itemize}
\item 2017 : Cette leçon ne peut se résumer à une présentation des relations d’Einstein.
\item 2015 : Cette leçon peut être traitée de façons très variées, mais il est bon que les candidats
aient réfléchi aux propriétés des diverses formes de rayonnements émis, aux dispositifs
exploitant ces propriétés et au cadre théorique permettant de les comprendre.
\item Jusqu’en 2013, le titre était : Absorption, émission spontanée ou induite du rayonnement.
Caractéristiques et applications.
\item 2014, 2013, 2012, 2011 : Trop souvent, il y a confusion entre les processus élémentaires
pour un atome et un ensemble d’atomes. De même le candidat doit préciser au cours de
sa leçon le caractère monochromatique ou non du champ de rayonnement qu’il considère
et plus généralement les caractéristiques du rayonnement stimulé.
\end{itemize}


\section*{Introduction}
  \addcontentsline{toc}{section}{Introduction}
  
Intro : J'aimerais qu'on commence cette leçon en partant de deux situations expérimentales et qu'on essaie d'analyser un petit peu ce qu'on observe. On va s'intéresser à deux sources de lumière:

\paragraph{Situation 1} On envoie la lumière du soleil sur un prisme : le faire si possible, sinon \url{https://www.youtube.com/watch?v=sT_EJKolE8Y}

\paragraph{Situation 2} Envoyer une lampe à vapeur de sodium ou de mercure sur un réseau.

\begin{itemize}
\item Dans les deux cas, on observe les longueurs d'ondes composant la lumière de la source : on parle de \textbf{spectre d'émission}.
\item Cas 1: spectre continu (source thermique). Cas 2: spectre discret (source spectrale).
\end{itemize}

En fait, la situation 1 a déjà été traitée dans le cours de thermodynamique: c'est le rayonnement d'un corps noir.  Dans cette leçon, nous allons nous intéresser non pas aux sources thermiques, mais aux sources lumineuses, celles pour lesquelles la lumière émise dépend de la nature des atomes mis en jeu et non pas de la température d’un filament.

[wiki] La théorie du rayonnement prévoit que le rayonnement émis par un corps chauffé est proportionnel à la température absolue et inversement proportionnel à la longueur d'onde portée à la puissance 4. Au cours de l'année 1900, des expériences montrent que cette théorie du rayonnement fonctionne bien pour des émissions allant de l'infrarouge au vert. Par contre, pour le bleu, le violet et, plus encore, l'ultraviolet, les résultats ne concordent pas du tout avec cette théorie, qui est mise en échec $\rightarrow$ \textbf{catastrophe ultraviolette}.

Autre problème: ce modèle n'explique pas le spectre continue qu'on observe avec la lampe à vapeur de Mercure.

C'est pour apporter une réponse à ce problème de théorie du rayonnement que le physicien allemand Max Planck propose à la fin de l'année 1900 une idée révolutionnaire qui, pour la première fois, postule que {les interactions entre la lumière et la matière se font \textbf{de manière quantifiée}}.

Dans cette leçon, on va voir comment cette quantification permet de rendre compte des  interactions lumière-matière qui se manifestent dans des phénomènes de notre vie de tous les jours.

\paragraph*{Mais de quoi parle-t-on précisément ?}  
  
\section{Interaction lumière-matière}

\subsection{Approche expérimentale}

Parmi les interactions lumière-matière, on distingue [sur slide] 

\subsubsection{Émission spontanée}

[Taillet p. 259] Déf. [Cagnac] 

\paragraph{Sources spectrales} Rayonnnement par émission spontanée. Ce sont des lampes à vapeurs atomiques, qui fournissent des spectres de raies. Ces lampes nécessitent un autotransformateur spécial. En effet, une haute tension est nécessaire pour établir une décharge entre les électrodes, les électrons accélérés excitant les atomes par collision (ceux-ci émettent alors de la lumière en se désexcitant). En TP, on utilise fréquemment des lampes à vapeur de sodium (le doublet jaune du sodium fournit également la lumière des réverbères) ou des lampes à vapeur de mercure (qui émettent plusieurs raies). Ces dernières peuvent être
\begin{itemize}
\item soit basse pression (raies fines, effet Doppler dominant, émission par unité de surface faible)
\item soit haute pression (raies élargies, effet collisionnel important mais Doppler toujours dominant, émission par unité de surface élevée). Ces sources n’atteignent leur régime de fonctionnement permanent qu’après plusieurs minutes : il faut donc les brancher en avance et éviter de les éteindre, car elles ont besoin de refroidir avant d’être rallumées. Ex: Lampes dites Philora. 
\end{itemize}
\textbf{Principe de fonctionnement:} [Olive, p. 118] On fait éclater des étincelles dans un tube contenant un gaz, provoquant une excitation des atomes. Un rayonnement correspondant à un spectre de raie est émis. Lorsque P ou T augmentent, les chocs entre atomes augmentent, ce qui induit une désexcitation plus rapide: le temps de vie de l'état et la durée du train d'onde diminuent. Les raies sont plus larges. L'effet Doppler est souvent le facteur prépondérant de l'élargissement.

\paragraph*{Expérience} Mesure raie d'émission du mercure.
  
\subsubsection{Absorption}

[Taillet p. 5] Déf. C'est le processus inverse de l'émission.

\paragraph*{Expérience} Spectre d'absorption de la rhodamine. Puis émission en luminescence.

\url{https://www.123couleurs.fr/explications/explications-lumi%C3%A8re/tl-fluophospho/}

\paragraph{Transition} Description quantitative de ces phénomène.

\subsection{Modèle d'Einstein}


\subsubsection{Hypothèses du modèle}

\begin{itemize}
\item Atomes : Résonance optique : deux niveaux d'énergie (justifier que bcp de pbs en physique peuvent se ramener à deux niveaux). Résonance (attention : Les niveaux ont une largeur, donc cettte condition n’est pas aussi stricte)
\item Lumière : $E = h \nu$. A cause de la largeur de la raie, on va considérer une densité spectrale.
\item $N_1, N_2$. Equilibre thermo et loi de Boltzmann.
\end{itemize}

Tableau avec les trois processus à remplir au fur et à mesure. Décrire schématiquement avec les deux niveaux. 

\textbf{N.B.} [Sylvio] Pertinence d’un système à deux niveau : L’hypothèse des deux niveaux est peu restrictive parce que l’interaction électromagnétique concernant les processus de transition est résonante autour d’une fréquence centrale, $\nu_0$.
Elle ne met donc en jeux que deux niveaux résonants tels que $E_{21} h \nu_0$.


\subsection{Coefficients d'Einstein}

[Pascal Olive] Essayer de faire rapidement cette partie.

\paragraph{Spontanée} On a vu la lampe spectrale. Modélisation: $A_{21}(\nu)$. Les OE associées aux photons émis ont des direction et des polarisations aléatoires. $\frac{\ud N_2}{\ud t} = - A_{21} N_2$. Unité: $A_{21}$ en $s^{-1}$.

[Wiki] $A_{21}$ gives the probability per unit time that an electron in state 2 with energy $E_2$ will decay spontaneously to state 1 with energy $E_1$, emitting a photon with an energy $E_2 - E_1 = h\nu$. Due to the energy-time uncertainty principle, the transition actually produces photons within a narrow range of frequencies called the spectral linewidth.

[Wiki] $B_{21}$ gives the probability per unit time per unit energy density of the radiation field per unit frequency that an electron in state 2 with energy $E_{2}$ will decay to state 1 with energy $E_{1}$, emitting a photon with an energy $E_2 - E_1 = h\nu$.

\paragraph{Absorption} $\frac{\ud N_1}{\ud t} = - B_{12} u_\nu(\nu_0) N_1$. Unité :  $B_{12}$ en $J^{-1} m^{3} s^{-2}$. \\

[Wiki] $B_{12}$ gives the probability per unit time per unit energy density of the radiation field per unit frequency that an electron in state 1 with energy $E_{1}$ will absorb a photon with an energy $E_2 - E_1 = h\nu$ and jump to state 2 with energy $E_{2}$.  \\

\paragraph{Transition} Et le Laser ? On en a pas parlé: ça utilise autre chose. 
 
\paragraph{Stimulée} \href{https://www.youtube.com/watch?v=7u3rRy97m9Y&t=19s}{Vidéo Absorption sodium}. Modélisation: $\frac{\ud N_2}{\ud t} = - B_{21} u_\nu(\nu_0) N_2$. Unités: $u_\nu$ en $J m^{-3} s^{-1}$ et $B_{21}$ en $J^{-1} m^{3} s^{-2}$. Un photon est émis avec la même fréquence, direction, polarisation et en phase que le photon incident.

[Taillet p. 132] 
Phénomènes d'absorption et d'émission aléatoires $\rightarrow$ approche probabiliste. Mais on considère un grand nombre d'atomes.

\textbf{Attention à bien distinguer le cas monochromatique du cas polychromatique}

[Sanz]
Remplir le tableau avec
\begin{itemize}
\item Coeff d'Einstein (Proba de transition par unité de temps) et unités
\item Évolution temporelle [Précis p. 690 ; Cagnac p. 102]
\item Régime stationnaire (à l'équilibre dynamique et non pas thermodynamique !)
\item Caractéristiques de chaque processus
\end{itemize}

\paragraph{Équation d'évolution:} On écrit l'équation totale impliquant tous les processus.

\paragraph{Transition} Quel est le lien entre ces coefficients ?

\subsection{Relation entre les coefficients}

[Précis p. 690 ; Cagnac p. 103] \textbf{On suppose maintenant l'équilibre thermodynamique.} Loi de Boltzmann pour $N_i$. $\frac{N_1}{N_2} = \e^{\beta h \nu_0}$. Dans la limite $T \rightarrow \infty$, $\frac{N_1}{N_2} = 1$ et $u_\nu \rightarrow \infty$, impliquant: $\frac{B_{21}}{B_{12}} = 1$. Si les atomes sont en équilibre avec le rayonnement il faut que la densité spectrale d’énergie coincide avec celle de Planck [Sanz, Houard p. 382, \textbf{Hecht, p. 604}]. Par identification avec la loi de planck  $u_\nu = \frac{8 \pi h \nu_0^3}{c^3} \frac{1}{\e^{\beta h \nu_0}}$, on trouve $\frac{A_{21}}{B_{12}} = \frac{8 \pi h \nu_0^3}{c^3}$.


Insister sur la nécessité de l'émission stimulée.

\paragraph{N.B. (voir si j'en parle ou pas)} En réalité, on a une largeur $\ud \nu$ autour de $\nu_0$ dans ces processus [Sylvio]:
\begin{itemize}
\item Elargissement naturel : l'inégalité de Heisenberg temps-énergie implique une largeur spectrale de l’ordre de
10$^{-5}$ nm.
\item Élargissement collisionel : les chocs entre particules modifient légèrement leurs états énergétiques. Ça dépend de la fréquence des chocs, donc de la section efficace et de la densité (donc de la pression, donc ils dominent dans les lampes haute pression). L’élargissement est de l’ordre de 10$^{-3}$ à 10$^{-2}$ nm.
\item Elargissement Doppler thermique : toutes les particules sont animées d’une vitesse, dont la répartition est isotrope. Par effet Doppler, toutes les émissions n’ont pas la même fréquence dans le référentiel du labo. Plus la température de la source est chaude, plus les vitesse sont élevées et plus l’effet est marqué. L’élargissement a le même ordre de grandeur que dans le cas de l’élargissement collisionnel, mais prédomine dans les lampes basse pression.
\end{itemize}

[JF] Discuter la largeur de raie liée à Heinseberg (cf. Sanz, p. 1068) et celle liée à l’élargissement pas effet Doppler (cf. Cagnac, p. 24). On peut évoquer les méthode de mesure de ces profil par interférométrie à division d’amplitude. Voir aussi Sanz, p. 1068).

\section{Applications}

\subsection{Application à la chimie}

[facultatif]

Bilan de puissance pour un système à deux niveaux soumis à une onde électromagnétique plane [Olive, p. 787]. Retrouver Beer-Lambert.

\paragraph{Expérience} Remonter à la concentration de la rhodamine connaissant $l$ et $\epsilon$.

\subsection{Application au laser}

Animation introductive: \url{https://www.youtube.com/watch?v=UDxdq_ogqR8}

Laser = Light Amplification by Stimulated Emission Radiation

Comparaison à une lampe spectrale : le laser est monochromatique, directif, cohérent. Il faut [Houard]:
\begin{itemize}
\item Un milieu amplificateur
\item Un cavité optique
\item Un Système de pompage
\end{itemize}

\subsection{Amplification par émission stimulée}

[Précis, p. 692]

Nécessité d'une inversion de population: la la variation de puissance $\delta P \propto [B_{21} N_2 - B_{12} N_1]$. Une amplification n'est possible que si $N_2 > N_1$. Or, à l'équilibre, $\frac{N_2}{N_1} = \e^{- \beta h \nu_0} < 0$.

Système amplificateur = milieu dans lequel a été réalisé l'inversion de population tel qu'un cristal solide, un gaz ou un semi-conducteur, qui est capable d'amplifier la lumière. Ce milieu contient des atomes, des molécules ou des ions excitables. 


Par exemple, par \textbf{pompage} (avec 3 niveaux). Le milieu amplificateur est "pompé" en lui fournissant de l'énergie, généralement sous forme de lumière (flash) ou d'électricité (décharge). Cette énergie d'excitation amène les atomes du milieu amplificateur à un état énergétique supérieur. En réalité, un pompage direct de   1 à 2 permet au maximum d'obtenir $N_1 = N_2$ : On utilise un troisième niveau.

\textbf{Ex: Laser He-Ne}: He: état 0 et 3. Pompage de 0 à 3 par décharge électrique. He cède de l'énergie par choc à Ne. Ne passent de 0 à 2 tandis que He se désexcitent sans émettre de photons: inversion de population entre 1 et 2 de Ne.  Puis émission stimulée de Ne de 2 vers 1, la désexcitation de 1 vers fonda se fait rapidement.

\subsection{Rôle de la cavité optique}
 
[Voir Sanz, Houar, Olive]

La cavité Fabry-Pérot affine le spectre en sélectionnant seulement certaines fréquences. La largeur des pics dépend du coefficient de réflexion énergétique du miroir, leur hauteur dépend de la fraction du rayonnement émis, qui correspond à des pertes énergétiques.

Une rétroaction optique positive est nécessaire pour amplifier et maintenir le processus d'émission stimulée. Cela est réalisé à l'aide d'un résonateur optique, qui est composé de deux miroirs parallèles. L'un des miroirs est partiellement réfléchissant, permettant à une partie de la lumière émise de s'échapper sous forme de faisceau laser, tandis que l'autre miroir est hautement réfléchissant, renvoyant la lumière à l'intérieur du résonateur. 

\paragraph{Analogie avec l'oscillateur à pont de Wien} [Olive] Chaîne d'action ($\mu$) = milieu amplificateur. Chaîne de retour (rétroaction: $\beta$) = cavité optique. Elle laisse passer les fréquence propre. Le bouclage se fait par réflexion sur les miroirs. Seuls les modes pour lesquels $\mu \beta > 1$ peuvent naître dans la cavité. Pour n'avoir qu'un mode, soit on réduit la taille de la cavité, soit  on diminue le coeff de réflexion d'un miroir: mais perte de puissance. En pratique, on introduit un filtre interférentiel très sélectif. Pour les lasers He-Ne, la bande de fréquence admise est du même ordre de grandeur que la largeur à mi-hauteur de la raie d'émission de longueur d'onde $\lambda_0 632.8 =$ nm.
  
\section*{Conclusion}
  \addcontentsline{toc}{section}{Conclusion}
  
[Lire Olive, p. 792 et Houard]
  
Les lasers ont de nombreuses utilisations dans divers domaines en raison de leurs propriétés uniques. 
\begin{itemize}
\item Communication : Les lasers sont utilisés dans les fibres optiques pour transmettre des signaux de communication à haute vitesse sur de longues distances. Ils sont essentiels pour les réseaux de télécommunications modernes.
\item Médecine : Les lasers sont utilisés en chirurgie pour couper, cautériser et vaporiser les tissus de manière précise et contrôlée. Ils sont également utilisés dans des traitements médicaux tels que l'élimination des tatouages, l'épilation au laser et le remodelage de la cornée.
\item Recherche scientifique : Les lasers sont utilisés dans de nombreux domaines de recherche, tels que la physique, la chimie et la biologie. Ils sont utilisés pour l'étude des propriétés des matériaux, la spectroscopie, l'imagerie, la manipulation de particules, la fusion nucléaire contrôlée et bien d'autres applications.
\item Lecture optique : Les lasers sont utilisés dans les lecteurs de codes-barres, les lecteurs de DVD et les lecteurs de disques Blu-ray pour lire l'information enregistrée de manière précise et rapide.
\item Télémétrie: mesure de la distance Terre-Lune [olive, p. 798].
\end{itemize}

On a vu une application. Une autre est celle du ralentissement ou refroidissement des atomes.

\subsection*{A savoir}
\begin{itemize}
\item {[Olive]} Une onde parfaitement sinusoïdale (de durée infinie) n'existe pas dans la vraie vie (l'émission spontanée de 2 vers 1 doit avoir un début et une fin). Le principe d'Heisenberg montrer qu'il y a une indétermination de l'énergie $\Delta E$ d'autant plus grande que la durée de vie d'un atome dans l'état excité 2 est faible ($\Delta E \tau \sim h$). Comme la durée de vie dans l'état 1 est plus grande (état plus stable), on peut négliger $\Delta E_1$ et considérer l'indétermination sur la fréquence $\Delta \nu$ liée $\Delta E_2$: $\Delta \nu \sim \frac{1}{\tau}$. Or (Fourier), la durée d'un train d'onde $\Delta t \Delta \nu \sim 1$. CCL: la durée d'un train d'onde est du même ordre de grandeur que celle de l'état excité. Ces grandeurs (durée, amplitude, phase) varie aléatoirement autour d'une valeur moyenne d'un train d'onde l'autre.
\item Le temps de cohérence est la durée d'un train d'onde $\tau_c = \Delta t$.
\item La loi de Planck donne la luminance (puissance par unité de surface qui émet dans une direction normale par unité d'angle solide par unité de fréquence) vs la longueur d'onde. Au fur et à mesure que la température diminue, le sommet de la courbe de rayonnement du corps noir se déplace à des intensités plus faibles et des longueurs d'onde plus grandes.  
\item En TP: On utilisera essentiellement deux types de sources lasers avec des puissances de l’ordre du mW. Les lasers hélium-néon, ayant pour avantage d’être très monochromatiques (longueur d’onde de 632,8 nm pour le rouge, 543 nm pour le vert), peu divergents (avec une ouverture de l’ordre de 10$^{-3}$ rad) mais très encombrants (à cause de la cavité Fabry-Pérot). les diodes lasers, beaucoup plus portables, mais avec une divergence de faisceau un peu plus grande et une longueur d’onde à calibrer. Un laser polarisé est un laser qui n’émet que sur une polarisation bien définie. Un laser non polarisé est un laser dont la polarisation fluctue au cours du temps. Par conséquent, en plaçant un polariseur après un laser non polarisé, on obtient une source polarisée mais dont l’intensité fluctue.
\item \textbf{Sources thermiques:} Rayonnement thermique émis par désexcitation par chocs thermiques. Elles ont un spectre continu dont l’intensité dépend de la longueur d’onde. Elles constituent une bonne première approximation de la lumière blanche (spectre plat dans le visible). Exemples:
\begin{itemize}
\item Lampes à incandescence ordinaires: Le filament de tungstène est porté à une température d’environ 2800 K. Il est placé sous vide, ou dans une atmosphère gazeuse inerte, pour éviter l’oxydation. La répartition spectrale est à peu près celle d’un corps noir porté à la même température (avec donc une part importante du rayonnement dans l’infrarouge).
\item Lampes à incandescence Quartz-Halogène (en particulier Quartz-Iode): Le principe est le même que pour les lampes précédentes, mais l’ajout d’un gaz halogène à l’intérieur de l’ampoule augmente son temps de vie, en limitant la vaporisation du tungstène. Cela permet donc de porter le filament à une température plus élevée (3200 K), ce qui augmente l’intensité lumineuse et décale le maximum d’émission du spectre vers le visible.
\end{itemize}
\item En raison de l'énergie thermique, les particules subissent des vibrations et des collisions aléatoires, ce qui modifie constamment leurs trajectoires et leurs vitesses. Ces variations créent des fluctuations rapides dans leur mouvement ("dipôle oscillant"), ce qui se traduit par l'émission de photons.
\item La différence clé entre la fluorescence et la phosphorescence réside dans la durée de l'émission lumineuse après l'excitation. La fluorescence émet de la lumière immédiatement et cesse rapidement après l'excitation, tandis que la phosphorescence émet de la lumière même après la fin de l'excitation, avec une durée de vie plus longue.
\item Y a aussi les LEDs. Principe: La recombinaison d'un électron et d'un trou d'électron dans un semi-conducteur conduit à l'émission d'un photon. Contrairement à l'émission spontanée, l'émission lumineuse dans une LED est basée sur l'émission stimulée déclenchée par l'application d'un courant électrique.
\item La diffusion est un phénomène passif et non radiatif. Elle ne modifie pas l'état d'excitation ou d'émission des particules impliquées, mais plutôt la direction de propagation de la lumière. La diffusion est responsable de phénomènes tels que la diffusion Rayleigh (responsable de la couleur du ciel) et la diffusion de la lumière dans les matériaux dispersifs.
\end{itemize}

\section*{Description de l'expérience}
  \addcontentsline{toc}{section}{Description de l'expérience}

\textcolor{mycolor5}{TP Spectroscopie}

\begin{tcolorbox}[breakable, enhanced, colback=red!2!white,colframe=mycolor!85!black,title=\textbf{\textbf{Expérience}}]
\paragraph*{Matériel}
\paragraph{Spectre d'émission de la lampe Hg + lumière}
\begin{itemize}
\item Prisme
\item Lampe à vapeur de mercure
\item Fente réglable
\item Diaphragme circulaire
\item Condenseur (lentille de f'=5cm ou 7cm) [pour augmenter la luminosité, facultative]
\item Réseau 3000 traits/cm
\item Lentille achromatique f'=20cm
\item Écran (lourd si possible)
\item Feuilles blanches 
\item Règle > 1m
\item Petite règle
\item Scotch
\end{itemize}

\paragraph{Spectre d'absorption de la rhodamine}
\begin{itemize}
\item Lampe quartz-iode
\item Diaphragme circulaire
\item Spectromètre USB
\item Fibre optique
\item Cuve éthanol
\item Cuve rhodamine 610+éthanol
\item Support mini-potence avec trou
\item Filtre vert.
\end{itemize}

\paragraph{Protocoles} 

\paragraph{Protocole raie Mercure} [Houard p. 320/340. Sextant, p.4]
\begin{itemize}
		\item Faire le montage du Houard. \textbf{N.B.}
		\begin{itemize}
		\item Distance écran-réseau D = 1m
		\item Fraunhoffer à une lentille : conjuguer par la lentille l'écran et la fente source
		\end{itemize}
		\item Prendre une feuille blanche et tracer les différentes raies afin de les mesurer à la règle
		\item Tableau récapitulatif avec les $\lambda$ mesurée et tabulées.
\end{itemize}

\paragraph*{Aspect quantitatif :} Mesure des longueurs d'ondes associées aux raies. Voir notice et Sextant (UV, violet, violet, vert, doublet jaune) pour les valeurs et Houard pour les calculs: $\lambda \simeq \frac{d y_k}{D}$ avec $d = \frac{1}{3000} cm$ le pas du réseau et $y_k$ la distance entre la raie à l'ordre 1 et la raie blanche de l'ordre 0. Hypothèses:
\begin{itemize}
\item Incidence quasi-normal $\theta_ i \simeq 0$
\item $\tan \theta = \frac{y_k}{D} \simeq \theta  \simeq \sin \theta$. Rq: on pourrait le faire exactement avec $\sin$ et $\tan$, surtout si $\frac{y_k}{D}$ est pas si petit (vérifier l'hypothèse avant). 
\end{itemize}




\paragraph*{Protocole Rhodamine} 

\url{https://omlc.org/spectra/PhotochemCAD/html/009.html}

\begin{itemize}[label=$\triangleright$]
\item Brancher la fibre optique au spectro usb et la mettre sur un support trou.
\item Lancer SpectraSuite. Régler le temps d'intégration si besoin (par exemple: 100ms).
\item Faire le noir (lumière éteinte) et le blanc (lumière+cuve éthanol) en appuyant sur les deux ampoules.
\item Mettre la Rhodamine. Visualiser le spectre d'absorption. Noter $\lambda_{abs} = 542$nm
\item Visualiser le spectre de transmission. C'est le complémentaire de l'absorption.
\item Mettre un filtre vert. Noter les longueurs d'onde transmises du filtre (539nm et 804nm) sans et avec la rhodamine. Constater que le vert (539nm) est fortement absorbé.
\item On met la fibre optique sur le côté de la cuve (pour s'affranchir de la lumière incidente de la lampe et avoir le signal le plus fort possible) et voir une couleur apparaître (autour de $\lambda_{em} =  610$ nm) différente de la couleur excitatrice (vert).
\item Observer que la luminescence a lieu à des énergies inférieures à celle de la lumière excitatrice.
\item La raie fine excitatrice n'est pratiquement pas visible sur le spectre précédent, car le phénomène de diffusion est ici faible devant celui de luminescence.
\item CCL: en transmission: la couleur est celle du complémentaire du vert (magenta). En luminescence: la couleur est celle de $\lambda_{em, théo} = 610$ nm correspondant au orange.
\end{itemize}

\paragraph*{Aspect quantitatif :} Mesure des raies de la lambe Hg.

Mesure de $\lambda_{abs}$ et $\lambda_  {em}$ pour la rhodamine.


\end{tcolorbox}



\newpage


%------------------------------------------


\chapter{Propriétés macroscopiques des corps ferromagnétiques}


\paragraph*{Niveau:} L3

\paragraph*{Prérequis:} 
\begin{itemize}
\item Électromagnétisme dans les milieux aimantés
\end{itemize}

\paragraph*{Bibliographie:}
\begin{itemize}
\item Electromagnétisme MP, PC, PSI. Daniel Mauras. Physique-Prépa-Chimie.
\item Physique de l'état solide. Charles Kittel. Science Sup.
\item H-prépa. Electromagnétisme 2ème année PC, PSI.
\item Bertin, Faroux, Renault. Electromagnétisme 4 : milieux diélectriques et milieux magnétiques.
\item Tec\&Doc Physique PSI-PSI*. Olivier, More \& Gié.
\item Physique PSI-PSI* Tout-en-un. Dunod.
\item Électronique II. H-Prépa 2ème année PSI-PSI*.
\item Électronique -- Conversion de Puissance. PSI-PSI*. ellipses. Taupe-niveau. Meiler, Irlinger \& Kempf.
\item Physique. PSI. Pascal Olive. 
\item Bonus: sujet agrégation 2019, problème de physique.
\end{itemize}

\paragraph{Notes agrégats}
\begin{itemize}
\item 2017 : L’introduction des milieux linéaires en début de leçon n’est pas judicieuse.
\item 2016 : Un bilan de puissance soigné est attendu.
Jusqu’en 2013, le titre était : Propriétés macroscopiques des corps ferromagnétiques. Applications.
\item 2009, 2010 L’intérêt du champ ~H doit être clairement dégagé. L’obtention expérimentale
du cycle d’hystérésis doit être analysée.
\end{itemize}

\section*{Introduction}
  \addcontentsline{toc}{section}{Introduction}
 
Parler des milieux aimantés (para, dia, ferro) [Mauras]. Donner des exemples de matériaux.
  
\section{Aimantation des matériaux ferromagnétiques}  

[Oliv0, p. 825] Dans la matière, les charges et les courants peuvent être libres ou liées. $\rho_p$ et $j_p$ (liés) ne peuvent pas être directement mesurés (contrairement aux libres). 

Milieux magnétique: Il existe un moment magnétique $\ud^3 m$ dans un volume mésoscopique $\ud \tau$. On montre que $\rho_p = 0$, $j_p = \nabla \wedge M$ où $M = \frac{\ud^3 m}{\ud \tau}$ est l'aimantation.

\textbf{N.B.} [wiki] Le moment magnétique est une grandeur vectorielle qui permet de caractériser l'intensité d'une source magnétique. Cette source peut être un courant électrique, ou bien un objet aimanté. L'aimantation est la distribution spatiale du moment magnétique. Elle  caractérise à l'échelle macroscopique la capacité d'un matériau à se comporter comme un aimant.

Équation de Maxwell-Ampère: $\nabla \wedge H = j_l$ avec $H = \frac{B}{\mu_0} - M$ l'excitation magnétique et $j_l$ courants libres. Théorème d'Ampère: $\oint H \cdot \ud OM = i_l$ (courant libre). Les sources de $H$ sont les courants libres alors que les sources de $B$ sont les courants libres et l'aimantation (courants liés).

\textbf{N.B.} Fundamentally, contributions to any system's magnetic moment may come from sources of two kinds: motion of electric charges, such as electric currents; and the intrinsic magnetism of elementary particles, such as the electron.

[wiki] Any molecule has a well-defined magnitude of magnetic moment, which may depend on the molecule's energy state. Typically, the overall magnetic moment of a molecule is a combination of the following contributions, in the order of their typical strength:

\begin{itemize}
\item magnetic moments due to its unpaired electron spins (paramagnetic contribution), if any,
\item orbital motion of its electrons, which in the ground state is often proportional to the external magnetic field (diamagnetic contribution)
\item The combined magnetic moment of its nuclear spins, which depends on the nuclear spin configuration.
\end{itemize} 

\paragraph{Transition} $M$ et $B$ sont la réponse à l'excitation $H$. On va étudier la réponse en fonction de l'excitation.

\paragraph{BFR}

\subsection{Courbe de première aimantation}

\begin{itemize}
\item Courbe.
\item Aimantation à saturation. Odg $M_{sat}, B_{sat}$.
\item Température de Curie.
\item Non linéarité
\end{itemize}

\subsection{Interprétation en domaines de Weiss}

\begin{itemize}
\item Domaines de Weiss, explique $M(H)$.
\item Irréversibilité (transition).
\end{itemize}

\textbf{N.B.} Domaines de Weiss sont à l'échelle mésoscopique (de l'ordre de quelques $\mu$m).

\section{Cycle d'hystérésis}

Dire que si on redescend, on suit pas le même chemin. Tracer le cycle.

\subsection{Étude expérimentale : transformateur}

\paragraph{H-prépa}


\begin{itemize}
\item Canalisation des ldc.
\item Présentation du dispositif.
\item Calculs théorème d'Ampère et le reste.
\item Manip': XY. $B(H)$.
\item Mesure de $B_{sat}, B_r, H_c$.
\item Comparer aux valeurs d'un ferro doux.
\end{itemize}

\subsection{Aspects énergétiques}

\paragraph{BFR, H-prépa, Tec\&Doc, Olive p. 849}

\begin{itemize}
\item Hystérésis
\item Courants de Foucault
\item Pertes cuivre
\end{itemize}

Transition : choisir le bon matériau pour limiter les pertes par hystérésis

\subsection{Classement des milieux ferromagnétiques}

[Peut être dans une section à part ou en conclusion selon le temps]

Doux vs dur. $H_c, B_r$, cycle, $\mu_r$ (grand $\mu_r \Rightarrow$ canalisation, lien entre $\mu_r$ et cycle. \\
Donner exemples. Applications.

  
\section*{Conclusion}
  \addcontentsline{toc}{section}{Conclusion}

On peut conclure sur les applications.\\
On peut aussi parler d'un autre aspect: la transition de phase ferro-para (montrer la \ref{https://www.youtube.com/watch?v=03XDF5kzrEs}{vidéo}.)

\section*{Description de l'expérience}
  \addcontentsline{toc}{section}{Description de l'expérience}



\begin{tcolorbox}[breakable, enhanced, colback=red!2!white,colframe=mycolor!85!black,title=\textbf{\textbf{Expérience}}]
\paragraph*{Matériel}
\begin{itemize}
\item Transformateur variable (vrai) $220/110V$
\item Transformateur de Leybold démontable (deux bobines + le fer).
\item Bobine $n_1 = 500$ spires et $n_2 = 250$ spires 
\item Rhéostat $R = 20-30 \Omega$ (et non AOIP !)
\item Résistance AOIP $R'$ ($\times 10^5$)
\item Capacité
\item Oscilloscope
\item Fils, câbles, etc.
\item Chronomètre
\end{itemize}

\paragraph*{Protocole } 

\begin{itemize}[label=$\triangleright$]
		\item Faire le montage du TP "Conversion de puissance"
		\item $R = 22 \Omega$, $R' = 10^5 \Omega$, $C = 5 \mu F$.
		\item $V_x$ au bornes de $R$ (donne $H$) et $V_y$ aux bornes de $R'$ (donne $B$ après intégration avec le circuit intégrateur $R'C$)
\end{itemize}

\paragraph*{Aspect quantitatif :} Mesure de $B_{sat}, B_r, H_c$. 


\end{tcolorbox}


\newpage


%------------------------------------------


\chapter{Mécanismes de la conduction électrique dans les solides}


\paragraph*{Niveau:} L3

\paragraph*{Prérequis:} 
\begin{itemize}
\item 
\end{itemize}

\paragraph*{Bibliographie:}
\begin{itemize}
\item Physique PC/PC*. Pascal Olive. ellipses (pour Drude: p.217).
\item Électromagnétisme I. Bertin, Faroux, Renault (surtout pour la théorie des bandes, p.171).
\item Physique de l'état solide. Charles Kittel. SCIENCES SUP.
\item Physique des électrons dans les solides I. Henri Alloul
\item Neil W. ASHCROFT. Physique des solides. EDP Sciences, 2002
\item \url{https://perso.ens-lyon.fr/sylvio.rossetti/AGREG/LP/LP47_M%E9canisme%20de%20condiction%20%E9lectrique%20dans%20les%20solides/LP47_M_canisme_de_condiction__lectrique_dans_les_solides.pdf}
\end{itemize}

\section*{Introduction}
  \addcontentsline{toc}{section}{Introduction}
 
 
 
\section{Modèle classique de la conduction électrique dans les métaux}

\subsection{Présentation du modèle de Drude}

Hypothèses. Sens physique.

\subsection{Origine des collisions}

\subsection{Loi d'Ohm}

Discuter ordres de grandeur.

\subsection{Limites}

Le modèle de Drude permet bien de relier
conductivité à temps de collision en revanche il ne prédit pas la bonne valeur du libre parcours moyen et la bonne évolution en température. La vitesse des électrons n'est pas donnée par $\sqrt{T}$ (qui provient du gaz parfait classique) et le libre parcours moyen n'est pas limité par les collisions
avec les ions du réseau. Le modèle de Drude ne prédit pas non plus l'existence de conducteurs et d'isolants. Il faut donc développer un modèle quantique.

\section{Modèle des électrons libres (approche semi-classique)}

\subsection{Modèle}

Un métal peut être décrit comme un gaz parfait de fermions dégénéré. Donner des ordres de grandeur de la température de Fermi et de la vitesse de Fermi.


Traiter les métaux comme un gaz de Fermions libres et introduire l'énergie de Fermi et la vitesse de Fermi. La vitesse de Fermi est 10 fois plus importante que la vitesse
thermique et les libres parcours moyens associés sont 10 fois plus grands. Une interprétation des collisions provenant des interactions avec le réseau n'est alors plus possible.


\subsection{Avantages et limites}


\section{Théorie des bandes (approche quantique)}


Qu'est ce qui limite le transport dans un métal : les collisions avec les impuretés et les défauts ainsi qu'avec les phonons. La dépendance de la résistivité avec la température est reliée aux collisions
avec les phonons (terme linéaire en T).
Montrer des courbes de variation de résistivité/conductivité en fonction de la température (p 148 Kittel pour les métaux, p 127,128 129 Alloul, p565 Ashcroft et
Mermin pour les semi-conducteurs).



\subsection{Théorie des bandes}

[Voir aussi Houard, p. 383]

\subsubsection{Introduction à la théorie et classification des matériaux}

La théorie quantique explique les
bandes d'énergie dans un solide, et la
différence entre métal, isolant et semi-conducteur. Donner des ordres de grandeurs de gap pour différents matériaux
(comme Silicium, diamant).

Discuter l'existence de conducteurs ou d'isolants selon la position du potentiel chimique.

Parler des bandes permises et interdites. Bandes de valence et de conduction.

\subsubsection{Semi-conducteurs}

Discuter ensuite le cas des semi-conducteurs intrinsèques avec une dépendance
exponentielle du nombre de porteurs avec la température. Ouvrir en conclusion sur le dopage chimique ou électrostatique et ses applications immense.

Bande interdite. \\
Porteurs intrinsèques.\\
Conductivité due aux impuretés.

\section*{Conclusion}
  \addcontentsline{toc}{section}{Conclusion}


\section*{Description de l'expérience}
  \addcontentsline{toc}{section}{Description de l'expérience}

\textcolor{mycolor5}{TP Série 0 : Mesures électriques}

\begin{tcolorbox}[breakable, enhanced, colback=red!2!white,colframe=mycolor!85!black,title=\textbf{\textbf{Expérience}}]
\paragraph*{Matériel}
\begin{itemize}
\item Bobine de cuivre pur verni dans cristallisoir
		\item Thermocouple + lecteur
		\item Alimentation stabilisée (jusqu'à $15V$)
		\item Voltmètre
		\item Ampèremètre
		\item Fils
		\item Eau + bouilloire + glace
		\item Chronomètre
\end{itemize}

\paragraph*{Protocole } 

\begin{itemize}[label=$\triangleright$]
		\item Faire le montage $4$ pointes du TP (ou directement faire la mesure avec le multimètre numérique.
		\item $R = \frac{V}{I}$.
		\item $R = \frac{\rho L}{S}$. Comparer avec la valeur tabulée $\sigma = 59.6~10^6$ S/m.
		\item Éventuellement faire plusieurs valeurs de $V$ et $I$ et faire un fit.
		\item Autre manip': mettre de l'eau, suivre la variation de la tension avec la température.
		\item Tracer $\rho(T)$. Faire un fit linéaire.
\end{itemize}

\paragraph*{Aspect quantitatif :} Mesure de la conductivité électrique du cuivre (mesure 4 points). On pourra également effectuer cette mesure à différentes températures, en plongeant le fil dans un cristallisoir contenant de l'eau. Vérifier qu'on peut approximer localement la résistivité en fonction de la température par une loi affine  (loi de Matthiessen). Le coefficient $\alpha$, appelé coefficient de température, est donné dans le Handbook.




\end{tcolorbox}



\newpage


%------------------------------------------


\chapter{Phénomènes de résonance dans différents domaines de la physique}


\paragraph*{Niveau:} L2
\paragraph*{Prérequis:} 
\begin{itemize}
\item Équations différentielles linéaires.
\item Électrocinétique (RLC)
\item Mécanique du point
\item Formalisme complexe  
\item Principe d'un laser.
\end{itemize}

\paragraph*{Bibliographie:}
\begin{itemize}
\item Dictionnaire de physique. Taillet, Villain, Febvre.
\item Mécanique I 1ère année MPSI-PCSI-PTSI. H-prépa. Brébec.
\item Les nouveaux Précis. Mécanique PCSI. Clerc. Bréal.
\item Mécanique 1ère année. Tec\&Doc. Gié et Sarmant.
\item \url{https://www.physagreg.fr/electrocinetique-5-resonances-rlc-serie.php}
\item Expériences de physique: optique, mécanique, fluides, acoustique. Bellier, Bouloy, Guéant. 4e édition.
\item Optique, Houard.
\item Physique PC-PC*. Pascal Olive. ellipses
\item Les nouveaux Précis, tout-en-un Physique PC. Tisserand et al. Bréal.
\item Bonus: Acoustique des instruments de musique. Chaigne et Kergomard.
\end{itemize}


\paragraph{Notes agrégat}
\begin{itemize}
\item 2015 : Présenter l’exemple célèbre du pont de Tacoma n’est pas pertinent, sauf s’il s’agit d’effectuer une critique d’une interprétation erronée très répandue.
\item 2010 : L’analyse du seul circuit RLC est très insuffisante pour cette leçon. Le phénomène de résonance ne se limite pas aux oscillateurs à un degré de liberté.
\end{itemize}

\section*{Introduction}
  \addcontentsline{toc}{section}{Introduction}

Verre qui se brise: \url{https://www.youtube.com/watch?v=47cPhhywvOo}

Pourquoi ? Phénomène de résonance.


\subsection*{Définition}
  \addcontentsline{toc}{subsection}{Définition}

\textcolor{mycolor5}{Dico, Taillet.}

Phénomène selon lequel l'excitation périodique d'un système à une fréquence $\omega$ proche de l'une de ses fréquences propres $\omega_0$  provoque une réponse de très forte amplitude.

Explication avec les mains: l'excitation pousse constamment le système dans la direction où son mouvement libre l'entraînait (cf. balançoire). Le phénomène de résonance est un effet d'accumulation de l'énergie en injectant celle-ci au moment où elle peut s'ajouter à l'énergie déjà accumulée, c'est-à-dire « en phase » avec cette dernière.

\url{https://www.youtube.com/watch?v=B_u3sGbpM8M}

\section{Oscillations forcées d'un oscillateur linéaire amorti}


\subsection{Exemple en mécanique: l'oscillateur harmonique}


\textcolor{mycolor5}{Manip' ou simulation qualitative: \url{https://phyanim.sciences.univ-nantes.fr/Meca/Oscillateurs/ressort_rsf.php}}

[précis, p. 171. Hprépa, p. 117] On va modéliser ça. Schéma de l'animation. Référentiel galiléen. $u_x$ vers le bas. Forçage en A ($x_A$), extrémité du ressort. Equa diff $m \ddot{x} = - \alpha \dot{x} - k (x - l_0) + mg - f_{archimede} + x_A(t)$. On trouve
Forme canonique
\begin{equation}
\ddot{x} + \frac{\omega_0}{Q} \dot{x} + \omega_0^2 x = F(t)
\end{equation}
avec $\omega_0^2 = \frac{k}{m}$ et $2\alpha = \frac{\omega_0}{Q} = \frac{h}{m}$.
On va s'intéresser à des excitations sinusoïdale $F(t) = A \cos (\omega t)$.

\textbf{N.B} Le facteur qualité $Q = \frac{1}{2 \alpha}$ où $\alpha$ est l'amortissement: plus un système est amorti, plus sont facteur qualité est faible (pas ou peu d'oscillation). Inversement, un amortissement nul (Q infini) correspond au cas de l'oscillateur harmonique. Q donne de l'ordre de grandeur du nombre de pseudo-oscillations visibles. De plus, en régime pseudo-périodique, pour $Q >> 1$, $\frac{\Delta E_m}{E_m} = \frac{2\pi}{Q}$: Q chiffre la diminution d'énergie mécanique du système par pseudo-période: plus $Q$ est grand, moins il y a de perte d'énergie par pseudo-période.

[pécis, p. 138] Avoir en tête les différents cas du régime libre $F=0$ : 
\begin{itemize}
\item $Q > \frac{1}{2}$ ($\alpha<1$): régime pseudo-périodique.
\item $Q < \frac{1}{2}$: régime apériodique: décroissance sans oscillations.
\item $Q=0$: régime critique: relaxation à $x=0$ est la plus rapide.
\end{itemize}

\subsection{Résonance d'élongation}

[Hprépa, chapitre V] Equation linéaire, $x = x_H + x_p$, avec $x_0$ solution de l'équation homogène qui tend vers 0 lorsque l'oscillateur est amorti. On s'intéresse donc uniquement à la solution particulière. Montrer \url{https://phyanim.sciences.univ-nantes.fr/Meca/Oscillateurs/ressort_rsf.php} OU BIEN \url{https://femto-physique.fr/simulations/resonance-oscillator.php}: On observe un régime transitoire, qui ne dure que peu de temps, suivi d'un régime sinusoïdal forcé. \\

On résout en complexe. Puis, on s'intéresse aux variations du module avec la pulsation.  On cherche le max en dérivant (cf. \url{https://perso.ens-lyon.fr/sylvio.rossetti/AGREG/LP/LP48_ph%e9nom%e8ne%20de%20r%e9sonance%20dans%20diff%e9rents%20domaines%20de%20la%20physique/LP48_ph_nom_ne_de_r_sonance_dans_diff_rents_domaines_de_la_physique.pdf}).

[Hprépa, Précis, \textbf{Tec\&Doc}] On trouve deux solutions. On distingue deux cas $Q < \frac{1}{\sqrt{2}}$ ($0$ unique solution), ou $Q > \frac{1}{\sqrt{2}}$ : on a résonance à $\omega_r = \omega_0 \sqrt{1 - \frac{1}{\frac{1}{2Q^2}}}$. La fréquence de résonance n'est pas stricto-sensus la fréquence propre. Montrer l'allure des courbes (slide ou Simulation). le système se comporte comme un filtre passe-bas ou passe-bande. Plus Q est grand, plus 

[JF]
\begin{itemize}
\item  A très faible fréquence d’excitation, le mobile suit la force excitatrice donc l’amplitude de l’excitateur se retrouve
exactement sur le résonateur. 
\item A très haute fréquence d’excitation, le mobile est parfaitement incapable de suivre la force excitatrice de sorte
qu’il fait quasiment du surplace. L’amplitude de la position est proche de 0.
\end{itemize}

[JF] la résonance n’a lieu que pour $Q>\frac{1}{\sqrt{2}}$, c’est-à-dire qu’il faut que le frottements ne soit
pas trop fort (c’est logique!). On comprend aussi pourquoi, à la résonance en position, l’excitateur et le résonateur
ne sont pas en phase. Avec les mains : si on commence à tirer sur le résonateur quand il est en bout de course on
n’exploite pas correctement le délai dû à l’inertie du ressort ; il faut que l’excitateur commence à tirer un peu plus
tôt pour que le ressort agisse sur le résonateur au moment exact où celui-ci arrive en bout de course. De la même
manière on peut peut-être essayé de comprend pourquoi la fréquence de résonance en tension n’est pas exactement
la fréquence propre mais c’est pas trivial...

\subsection{Résonance de vitesse}

[Tec\&Doc, Hprépa] $v = j\omega x$. La résonance en vitesse a lieu pour toute valeur de Q et $\omega_r = \omega_0$. Passe-bande.


[JF] - A très faible fréquence d'excitation, le mobile suit la force excitatrice qui va lentement donc sa vitesse tend vers
0. \\
- A très haute fréquence d'excitation, le mobile est parfaitement incapable de suivre la force excitatrice de sorte
qu'il fait quasiment du surplace. Sa vitesse tend à nouveau vers 0. \\
Ainsi, il existe forcément une fréquence d'excitation pour laquelle l'amplitude en vitesse est maximale.



\subsection{Équivalent en électrocinétique}

\subsubsection{Analogie électromécanique}

[Hprépa exo 1, Tec\&Doc] Circuit RLC série. Loi des mailles. On retrouve l'équation de la mécanique sur q avec les correspondances suivantes [Tableau slide].

\begin{itemize}
\item Élongation $\longleftrightarrow$ Charge 
\item Vitesse $\longleftrightarrow$ Courant
\end{itemize}

\subsubsection{Application: le circuit RLC série}

\textcolor{mycolor5}{TP Électronique de base - Résonance}

\url{http://mawy33.free.fr/cours%20sup/32-100%20%C3%A9l%C3%A9ctrocin%C3%A9tique%20RLC%20sinusoidal.pdf}

On étudie la réponse d'un circuit RLC série à une excitation produite par un échelon de tension délivré par un générateur BF (signal en créneaux de période suffisamment longue).

Le montage RLC série peut-être étudié en tension ou en intensité et qu'il y a donc existence de deux types de résonances aux caractéristiques différentes.

\subsubsection{Résonance en intensité}

RLC série avec étude de la tension aux bornes de la résistance.

Il y a toujours résonance en intensité (par rapport à Q) contrairement à la résonance en tension aux bornes du condensateur.

On peut mesurer aussi le facteur de qualité (bande passante): $Q = \frac{f_0}{\Delta f}$.

\subsubsection{Résonance en tension}

Jouer sur différents paramètres $R$, $C$ pour changer
\begin{itemize}
\item $f_0 = \frac{1}{2\pi \sqrt{LC}}$.
\item $f_r = f_0 \sqrt{1 - \frac{1}{2Q^2}}$.
\item $Q = \frac{1}{R} \sqrt{\frac{L}{C}}$.
\end{itemize}

\subsection{Un exemple en chimie: polarité des molécules}

[Hprépa, p. 179 (exo)] Exemple: HCl. Système à deux points A et B porteurs de charges $\pm \delta e$. Spectroscopie IR.

Montrer \url{https://www.chemtube3d.com/spectrovibhcl1-ce-final/}

\paragraph{Transition} On vient de voir le cas où le système avait pouvait entrer en résonance pour une fréquence. Un système peut avoir plusieurs, voire une infinité de fréquence de résonance: c'est le cas des cavités.

\section{Cavité résonnante}

\subsection{Définition} 

Une cavité résonnante (ou résonante), parfois appelée résonateur, est un espace creux à l'intérieur d'un solide dans lequel une onde entre en résonance.

\subsection{Exemple en acoustique}

\textcolor{mycolor5}{Manip' qualitative en acoustique: diapason avec et sans caisse de résonance: amplification}

\paragraph*{Interprétation}

Établissement d'un système d'ondes stationnaires.

Une cavité résonante en acoustique fonctionne de manière similaire à une cavité résonante en optique, mais au lieu de manipuler la lumière, elle manipule les ondes sonores. Elle est utilisée pour amplifier certaines fréquences sonores et créer des résonances. Une cavité résonante acoustique est généralement constituée d'une enceinte fermée avec des parois réfléchissantes. Ces parois peuvent être rigides ou flexibles, et elles sont souvent conçues de manière à réfléchir efficacement les ondes sonores à l'intérieur de la cavité. Lorsqu'une onde sonore est introduite dans la cavité, elle se réfléchit entre les parois de la cavité, créant ainsi des interférences constructives et destructives. Cela conduit à la formation d'ondes stationnaires à des fréquences spécifiques appelées modes de résonance. Les dimensions de la cavité, y compris sa taille et sa géométrie, déterminent les fréquences des modes de résonance. Les cavités résonantes peuvent être conçues pour amplifier certaines fréquences et supprimer d'autres. Par exemple, une cavité de résonance conçue pour amplifier les basses fréquences aura des dimensions plus grandes.

\paragraph*{Application en musique} La plupart des instruments acoustiques emploient des résonateurs, tels que les cordes et le corps d'un violon, la longueur du tube d'une flûte, et la forme d'une membrane de tambour.

Les cavités résonantes en acoustique sont utilisées dans plusieurs applications. Par exemple, les instruments à cordes, tels que les guitares et les violons, utilisent une cavité résonante pour amplifier les vibrations des cordes et produire un son plus fort. Les enceintes acoustiques utilisent également des cavités résonantes pour amplifier les fréquences sonores et améliorer la qualité sonore. En résumé, une cavité résonante en acoustique fonctionne en permettant aux ondes sonores de se réfléchir entre les parois de la cavité, créant ainsi des ondes stationnaires de résonance à des fréquences spécifiques. Cela permet d'amplifier et de contrôler les fréquences sonores pour diverses applications acoustiques.

\paragraph{Exemple de la guitare} Pour une guitare acoustique, les fréquences amplifiées dépendent principalement des caractéristiques de la caisse de résonance de l'instrument. La caisse de résonance de la guitare amplifie principalement les fréquences situées dans la plage des basses et des médiums.

Les basses fréquences amplifiées se situent généralement entre 80 Hz et 250 Hz. Ces fréquences correspondent aux notes les plus graves produites par les cordes de la guitare.

Les fréquences médiums amplifiées se situent généralement entre 250 Hz et 2000 Hz. C'est dans cette plage que se trouvent les harmoniques et les résonances caractéristiques de la guitare, donnant au son de l'instrument sa richesse et sa chaleur.


\subsection{Exemple en optique}

Expliquer la nécessité d'utiliser une cavité optique [Olive, Houard, Précis PC].

Lorsqu'une source de lumière est introduite dans la cavité, les miroirs réfléchissent la lumière en arrière et en avant entre eux. Cela crée des interférences entre les ondes réfléchies, ce qui conduit à la formation d'ondes stationnaires à des fréquences spécifiques appelées modes de résonance. Ces modes sont caractérisés par des nœuds (points où l'amplitude de l'onde est nulle) et des ventres (points où l'amplitude est maximale) fixes dans la cavité.

Dans un laser, le résonateur optique joue un rôle essentiel dans le processus d'amplification et de génération du faisceau laser cohérent. Bien que le milieu amplificateur puisse fournir l'amplification de la lumière, le résonateur optique permet d'obtenir une rétroaction positive et de maintenir la cohérence du faisceau laser. Le résonateur optique permet de sélectionner certaines longueurs d'onde spécifiques et de rejeter les autres. Cela est possible en ajustant la distance entre les miroirs réfléchissants pour créer une résonance constructive à une certaine longueur d'onde. Lorsque les photons de cette longueur d'onde particulière sont émis par émission stimulée, ils sont amplifiés à chaque passage entre les miroirs et renforcés par la rétroaction optique. Les autres longueurs d'onde qui ne satisfont pas aux conditions de résonance constructive sont rapidement atténuées. Cela permet d'obtenir une émission laser monochromatique.

\subsection{Analogie avec la corde de Melde}

[Chaigne, Sylvio] Tableau.

Une corde vibrante est un système mécanique où une corde tendue est mise en vibration, produisant des ondes sonores. Lorsque la corde est excitée, elle peut vibrer à différentes fréquences naturelles déterminées par sa longueur, sa tension et sa densité linéique. Ces fréquences naturelles sont appelées modes de résonance de la corde. Lorsque la corde vibre à l'une de ces fréquences, elle résonne et amplifie l'amplitude des vibrations. 

De manière similaire, une cavité résonante est une structure qui peut emmagasiner et amplifier des ondes sonores. Elle peut être constituée d'un volume d'air ou d'un autre milieu avec des parois réfléchissantes. Lorsqu'une onde sonore entre dans la cavité, elle peut se réfléchir entre les parois, créant des interférences constructives. Si la fréquence de l'onde correspond à l'une des fréquences naturelles de résonance de la cavité, une amplification significative de l'amplitude de l'onde se produit. Cela se traduit par une augmentation de l'énergie sonore à cette fréquence spécifique.

Dans les deux cas, que ce soit une corde vibrante ou une cavité résonante, l'idée de résonance est centrale. Lorsque le système est excité à sa fréquence de résonance, une amplification de l'amplitude des vibrations ou de l'énergie sonore se produit. Les fréquences de résonance sont déterminées par les caractéristiques du système, telles que la longueur de la corde, la tension, la densité, ou les dimensions de la cavité. L'analogie réside dans le fait que les deux systèmes ont des fréquences naturelles de résonance et peuvent amplifier les vibrations ou les ondes sonores à ces fréquences spécifiques.

\section*{Conclusion}
  \addcontentsline{toc}{section}{Conclusion}

[Pérez]

Conclure sur l'importance de l'étude des résonance: soit pour les éviter, soit pour les exploiter.

Exemple d'applications (attention: il faut maîtriser les concepts): RMN, résonance de spin électronique, résonateur confocal (laser).


\subsection*{A noter}
\begin{itemize}
\item La résonance est une notion relative à des oscillations forcées,
alors que les modes propres sont eux relatifs à des oscillations libres.
\end{itemize}

\section*{Description de l'expérience}
  \addcontentsline{toc}{section}{Description de l'expérience}



\begin{tcolorbox}[breakable, enhanced, colback=red!2!white,colframe=mycolor!85!black,title=\textbf{\textbf{Expérience}}]
\paragraph*{Matériel}
\begin{itemize}
\item GBF GX320 Metrix
\item Boîte à décades $C$ : $C = 20nF$
\item Boîte à décades $R$ : $R = 20 \Omega$
\item SELF $L$ : $L = 4.7 mH$, $r = 12 \Omega$
\item Oscilloscope
\item RLC-mètre
\item Fils et câbles
\item Chronomètre
\end{itemize}

\paragraph*{Matériel manip' qualtitative}
\begin{itemize}
\item Diapason + caisse de résonance + truc pour frapper
\end{itemize}

\paragraph*{Protocole } 

\begin{itemize}[label=$\triangleright$]
		\item Attention : il faut prendre en compte TOUTES LES RÉSISTANCES (i.e. celle de la bobine $r$ et celle du GBF $R' = 50 \Omega$)
		\item Mesurer tous les $R$, $L$ et $C$.
		\item Faire le circuit (choisir le bon ordre selon qu'on mesure $U_R$ ou $U_C$).
		\item Mettre en $X$ le sweep (fréquence) (VGC IN Sweep out) et en $Y$ $U_R$ ou $U_C$.
		\item Mode XY, persistance, haute-résolution, etc.
		\item Déterminer la conversion tension/fréquence (soit mode normal, soit avec les pas 256 = 8bits): On code $2V$ et $f_i \rightarrow f_f$ sur $\Delta t$ s/256 pas. $V = At$, $f = A' t + B$. Puis on réinjecte pour avoir $f(V)$.
\end{itemize}

\paragraph*{Aspect quantitatif :} Mesure de $Q = \frac{f_0}{\Delta f}$ (valable pour $U_R$, et pour $U_C$ sous condition que $Q$ grand (i.e. $>5$), et mesure de $f_r$/$f_0$. Comparer au valeurs théoriques. 

Manipulation qualitative: effet de changement des paramètres sur $f_0$ et sur la résonance $Q$ pour $U_C$.


\end{tcolorbox}


\newpage


%------------------------------------------


\chapter{Oscillateurs ; portraits de phase et non-linéarités}


\paragraph*{Niveau:} L2
\paragraph*{Prérequis:} 
\begin{itemize}
\item Mécanique de 1ère année.
\item Modèle du pendule simple.
\item Oscillateur harmonique.
\end{itemize}

\paragraph*{Bibliographie:}
\begin{itemize}
\item A utiliser partout : Toute la mécanique MPSI-PCSI, MP-PC-PSI. Bocquet, Faroux, Renault. J'intègre DUNOD.
\item Mécanique: fondements et applications. Pérez.
\item BUP : \url{https://uhincelin.pagesperso-orange.fr/LP49_BUP_portrait_phase_oscil.pdf}
\item Physique Sup PCSI. Tec\&Doc. Grécias, Migéon.
\item Borda: \url{https://bd2kstmm.files.wordpress.com/2020/06/lp49_oscillateurs_portraits_de_phase_1.pdf}
\item Simulation: \url{https://femto-physique.fr/simulations/simple-pendulum.php}
\end{itemize}

\section*{Introduction}
  \addcontentsline{toc}{section}{Introduction}

Mettre des images de la vie de tous les jours (horloge, cœur qui bat, balançoire, etc.) $\rightarrow$ oscillateur.

\section{Oscillateurs non-linéaires}

\subsection{Définition oscillateur}

cf. Taillet.

On distingue oscillateurs linéaires et non linéaires.

\subsection{Définition système (non) linéaire}

Un système linéaire répond à une excitation proportionnellement à l'amplitude de celle-ci. 

Un exemple d'oscillateur linéaire: l'OH: $\ddot{\theta} + \omega^2 \theta = 0$. Résolution connue (sinus). 


Un système non-linéaire ne répond pas linéairement. Il y a dès lors deux manifestations élémentaires, et parfaitement générales, de la non-linéarité d'un système :

\begin{itemize}
\item l'amplitude de la réponse n'est pas proportionnelle à celle de l'excitation ;
\item si on l'excite sinusoïdalement à une fréquence $f$, il répond éventuellement à d'autres fréquences.
\end{itemize}

\subsection{Exemple du pendule simple : mise en évidence de la non-linéarité}

- Un exemple d'oscillateur non-linéaire: le pendule simple non-amorti.

$\ddot{\theta} + \omega^2 \sin \theta = 0$.

- Dérivation du la formule de Borda [Toute la méca].

\url{https://bd2kstmm.files.wordpress.com/2020/06/lp49_oscillateurs_portraits_de_phase_1.pdf}

\subsubsection{Expérience} Formule de Borda, fit.


C'est bien un système non-linéaire. Résolution analytique compliquée : comment étudier ça ? -> un nouvel outil: le portrait de phase. 

\textbf{[Pérez] Il faut discuter les principaux effets non linéaires dans le cas des oscillateurs, à savoir, variation de la fréquence en fonction de l'amplitude et création d'harmoniques.}

\section{Portrait de phase}

\subsection{Définition}

\textcolor{mycolor5}{BUP} \\
\textcolor{mycolor5}{[Mécanique I, H-prépa]}

Introduire les différentes notions (système, plan de phase, trajectoire,  etc.)

\subsection{Propriétés}

\begin{itemize}
\item Déterminisme: 
	\begin{itemize}
	\item Deux trajectoires ne se coupent pas. 
	\item Trajectoires fermées = mouvements cycliques.
	\end{itemize}
\item Points fixes.
\item Sens de parcours
\item CI.
\item Energie
\end{itemize}

C'est un outil très riche pour l'analyse de nombreux systèmes et en particulier des oscillateurs.

\section{Application à différents oscillateurs}

[Tec\&Doc, H-prépa]

\subsection{Oscillateur non-amorti}

- Application sur le pendule non-amorti linéaire. Montrer l'acquisition.

- Application sur le pendule non-amorti non-linéaire. Montrer l'acquisition.

- Interprétation (le dessiner au tableau): sens de lecture, conditions initiales. 

- Finir l'illustration avec une animation: \url{https://femto-physique.fr/simulations/simple-pendulum.php}

- Retrouver le régime linéaire pour les petits angles.

[Toute la méca] discuter les différents cas selon l'énergie (p. 452).

\subsection{Oscillateur amorti}

- Application sur le pendule non-amorti non-linéaire. Montrer l'acquisition.

- Interprétation (le dessiner au tableau): sens de lecture, conditions initiales. 

- Finir l'illustration avec une animation: \url{https://femto-physique.fr/simulations/simple-pendulum.php}

\subsection{Oscillateur auto-entretenu}

(Si le temps).

\subsubsection{Définition}

[BUP] Un système qui, tel une horloge, évolue indéfiniment de façon périodique doit recevoir de l'énergie pour compenser les phénomènes dissipatifs inévitables qui accompagnent son fonctionnement. Un tel
système est appelé oscillateur entretenu.

\subsubsection{Oscillateur de Van Der Pol}

[Toute la méca] \\
Equa diff. Notion de cycle limite.

Compliqué à résoudre -> soit numérique soit approximation.

\subsubsection{Portrait de phase}

(numérique: Euler, Runge-Kutta, etc.)   \url{https://phyanim.sciences.univ-nantes.fr/Meca/Oscillateurs/vdp_phase.php}

Parler des bifurcations?

\section*{Conclusion}
  \addcontentsline{toc}{section}{Conclusion}

Le portrait de phase est utile pour décrire le fonctionnement d'un système sans calculs lourd.


\section*{Description de l'expérience}
  \addcontentsline{toc}{section}{Description de l'expérience}



\begin{tcolorbox}[breakable, enhanced, colback=red!2!white,colframe=mycolor!85!black,title=\textbf{\textbf{Expérience}}]
\paragraph*{Matériel}
\begin{itemize}
\item Pendule
\item Masse
\item Carte d'acquisition + câbles
\end{itemize}

\paragraph*{Protocole } 

\begin{itemize}[label=$\triangleright$]
		\item Brancher et faire l'acquisition sur Latis Pro.
\end{itemize}

\paragraph*{Aspect quantitatif :} Mesure $T(\theta)$ sur Latis-Pro (faire une modélisation en cos puis mesures automatiques). Faire l'ajustement de la formule de Borda sur Qtiplot.

Tracer les portraits de phases dans le cas non dissipatif (avec masse) et dissipatif (sans masse). 


\end{tcolorbox}


\newpage



%------------------------------------------


\chapter{Cinématique relativiste. Expérience de Michelson et Morley}


\paragraph*{Niveau:} L3
\paragraph*{Prérequis:} 
\begin{itemize}
\item Mécanique classique
\item Électromagnétisme
\end{itemize}

\paragraph*{Bibliographie:}
\begin{itemize}
\item Introduction à la relativité. D. Langlois. Vuibert
\item Introduction à la physique moderne : relativité et physique quantique. Claude Fabre, Charles Antoine, Nicolas Treps. Dunod.
\item Relativité. M. Boratav et R. Kerner. ellipses.
\item Relativité restreinte: bases et applications. Claude Semay. Dunod.
\item Relativité: fondements et applications. Pérez.
\item Cours Laurent : \url{http://supernovae.in2p3.fr/~llg/Enseignements/Agregation/Relativite/}
\end{itemize}

\paragraph*{Notes agrégat + CR}
\begin{itemize}
\item 2016 : Les notions d’événement et d’invariant sont incontournables dans cette leçon.
\item 2015 : Le jury rappelle qu’il n’est pas forcément nécessaire de mettre en oeuvre des vitesses relativistes pour être capable de détecter et de mesurer des effets relativistes.
\item 2014 : Cette leçon exige une grande rigueur dans l’exposé tant sur les notions fondamentales de relativité restreinte que sur les référentiels en jeu. Elle invite les candidats à faire preuve d’une grande pédagogie pour présenter des notions a priori non intuitives et faire ressortir les limites de l’approche classique. Un exposé clair des notions d’invariant relativiste est attendu.
\item Karim : Préciser que les transformations de Lorentz viennent du fait que c est constante.
\end{itemize}


\section*{Introduction}
  \addcontentsline{toc}{section}{Introduction}
  
[Fabre] Les lois qui régissent le mouvement des corps ont été pendant longtemps basées sur des évidences : temps universel, distance absolues. 

Aujourd'hui, on sait que le temps, l'espace et le mouvement sont en fait des concepts plus subtils. 

En effet, vers la fin du 19ème siècle, on a remis en cause ces postulats à travers des interrogations qui ont paradoxalement portés sur un autre domaine de la physique: l'électromagnétisme. Il s'en est suivi le développement d'une nouvelle théorie appelée "la relativité restreinte". Cette dernière permet entre autre de répondre à des questions comme:
\begin{itemize}
\item Le tic-tac d'une horloge ou la longueur d'une règle sont-ils les mêmes partout ?
\item La vitesse de la lumière dépend-elle du référentiel d'étude ?
\end{itemize}

Dans cette leçon, nous allons voir comment la relativité restreinte permet de répondre à ces questions. Mais avant cela, découvrons comment cette théorie est née.
  



\section{De Galilée à Einstein}

En mécanique classique, nous avons étudié les changements de référentiels galiléens, notamment la transformation de Galilée pour passer d’un référentiel galiléen à un autre. Nous avons vu en particulier la formule de composition des vitesse. 

\subsection{Mécanique classique}

[ellipses, Langlois, Semay]

\paragraph{Universalité du temps} Postulat de Newton. Possibilité de définir un temps identique $t = t'$ dans $\mathcal{R}$ et $\mathcal{R}'$.

\paragraph{Principe de la relativité galiléenne} Les lois de la physique sont invariantes par changement de référentiel galiléen (ou inertiel).

\paragraph{Transformations de Galilée}


[Semay, ellipses] $\rf$ et $\rf'$ en mouvement relatif, vitesse constante $v = v_{\rf'/\rf}$.

\begin{align*}
r' &= r - vt \\
t' &= t.
\end{align*}

Implique la \textbf{loi de composition des vitesses} : $u' = u - v$.   

En particulier : appliqué à la lumière : La mécanique classique prédit que la vitesse apparente d’un rayon lumineux doit changer selon le référentiel de l’observateur $c' = c - v$.

\subsection{Incompatibilité avec l'électromagnétisme}

[ellipses]

Cependant en 1865, Maxwell publie sa théorie de l’électromagnétisme. Le problème: les équations de Maxwell ne sont pas invariantes sous la transformation de Galilée: montrer \textbf{sur slide} un exemple d'une équation de Maxwell (ex: MF) ou de l'équation de d'Alembert dans un réf $\rf$ et ce que ça donne dans un réf $\rf'$ sans faire le calcul (\textbf{mais savoir le faire pour les questions}).

\subsubsection{Trois solutions}

\begin{enumerate}
\item Maxwell a faux
\item Rendre Maxwell et Galilée compatibles
\item La méca classique est fausse
\end{enumerate}

(1) improbable. (3) euh.. pas tout de suite quand même. (2) On tente des choses $\rightarrow$ l'éther.

\subsubsection{Éther}

Référentiel privilégié servant de support à la propagation des OE (comme l'atmosphère pour les ondes acoustiques) et dans lequel les équations de Maxwell sont valables.



\subsection{Expérience de Michelson et Morley}

[Langlois]

Michelson et Morley (1881, puis 1887, souvent répétée et améliorée depuis). Nécéssité d'un interféromètre pour mesurer précisément des différences de trajets optiques. Dire qu'il faut mesurer la différence de temps de trajet aller-retour entre les deux bras.

\subsubsection{Description}

\paragraph{Analogie avec deux nageuses} \url{https://www.youtube.com/watch?v=6_hyWb8TEEQ}

\paragraph{Calcul avec le Michelson}

Faire le schéma (lame d'air). $c$: vitesse de la lumière par rapport à l'éther. $u = 30$ km/s : vitesse de la terre par rapport à l'éther. $v$ : vitesse de la lumière par rapport à la Terre. 
\begin{equation}
\bm v_{lum/Ether} = \bm v_{lum/Therre} + \bm u.
\end{equation}
Faire le calcul de la durée du trajet aller-retour.

Animation : \url{https://www.youtube.com/watch?v=6_hyWb8TEEQ}

\paragraph{Bras $\parallel$}
En norme: aller $v = c + u$, retour $v = c - u$.
\begin{equation}
t_{\parallel} =  \frac{d}{c+u} + \frac{d}{c-u} = \frac{2d}{c} \left(1 + \frac{v^2}{c^2}\right) + o\left(\frac{v^2}{c^2}\right)
\end{equation}


\paragraph{Bras $\perp$}
En norme: Animation + faire le schéma. Pythagore donne $c^2 = v^2 + u^2$. D'où une vitesse à l'aller et au retour de $v = \sqrt{c^2 - u^2}$.
\begin{equation}
t_{\perp} =  \frac{2d}{\sqrt{c^2 - u^2}} = \frac{d}{c} \left(1 + \frac{v^2}{c^2}\right) + o\left(\frac{v^2}{c^2}\right)
\end{equation}

\paragraph{Différence de temps de parcours}
A l'ordre 2: $\Delta t = t_{\parallel} - t_{\perp} = \frac{d}{c} \frac{v^2}{c^2}$. Donne un déphasage $\varphi = \frac{2\pi}{\lambda}$ et une différence de marche $\delta = c \Delta t = \frac{d v^2}{c^2}$.

\paragraph{Ordre d'interférence}

$p = \frac{\delta}{\lambda} = \frac{d v^2}{c^2}$.

A.N.: $u = 30km/s$, $c = 3 10^8 m/s$, $d = 10m$, $\lambda = 500nm$ donne $p = 0.2$ franges $\rightarrow$ parfaitement observable !

L'appareil n'étant pas parfait, on ne peut assurer que la distance entre la lame séparatrice et le miroir soit parfaitement égale dans les deux directions, et de ce fait l'apparition de franges ne permet pas de conclure directement. En revanche, on peut faire la différence entre les franges dues à l'appareil, et celles dues au phénomène qu'on veut mettre en évidence : il suffit pour cela de faire tourner l'appareil d'un quart de tour pour intervertir les deux trajets, et d'observer si les franges se modifient.

Ils ont donc fait une rotation de 90° des bras. On devrait observer un décalage de la figure d'interférence mais ils n'ont rien vu. CCL: « s'il y a un mouvement relatif entre la Terre et l'éther luminifère, il doit être petit ». "The expected deviation of the interference fringes from the zero should have been 0.40 of a fringe – the maximum displacement was 0.02 and the average much less than 0.01 – and then not in the right place."

\textbf{N.B.} Comme on ne sait pas a priori quelle est la vitesse de la Terre par rapport à l'éther, ni même si l'on n'est pas, par hasard, dans un endroit et à un moment où sa vitesse est nulle, il faut refaire l'expérience dans plusieurs directions, et avec plusieurs mois d'écart pour profiter du fait que la vitesse de la Terre par rapport à l'éther est modifiée.

\textbf{N.B} [Pérez] Depuis, on a refait l'expérience plein de fois, de plus en plus précise, mais pareil. En octobre 2003, précision relative de $2.6 10^{-15}$.


\subsubsection{CCL}

Pas d'éther, invariance de ma vitesse de la lumière dans le vide $\rightarrow$ Solution 3: la méca classique est fausse. 

Il faut construire une nouvelle cinématique en accord avec la théorie de Maxwell: cette nouvelle construction de la cinématique est la relativité restreinte.

  
\subsection{Postulat d'Einstein}

Plusieurs physiciens théoriciens (Lorentz, Poincaré) proposent des solutions formelles ad hoc pour réconcilier mécanique classique et électromagnétisme, sans toutefois en tirer toutes les conséquences physiques.

[Langlois]

En 1905, Einstein va construire une nouvelle cinématique en s'appuyant sur deux postulats
\begin{itemize}
\item Toutes les lois de la physique (y compris EM) sont les mêmes dans tous les référentiels galiléens.
\item La vitesse de la lumière dans le vide est la même dans tous les référentiels : la vitesse de la lumière $c$ est \textbf{invariante} par changement de référentiel.
\end{itemize}

À partir de ces postulats, on trouve les nouvelles équations de changement de référentiel (formulées
par Lorentz et Poincaré), et on construit un nouvel ensemble de lois mécaniques baptisé relativité restreinte (special relativity).

\paragraph{Transition} Comme on l'a vu, les postulats sont en contradiction avec la loi de composition de vitesse de Galilée. Il faut donc changer les lois de transformation d'un référentiel galiléen vers un autre.
  
\section{Description relativité restreinte}

\subsection{Événement}

[Fabre, p.11] 

Phénomène physique bien localisé à la fois dans l'espace et le temps. Donner des exemples (émission d'un signal lumineux, collisio entre deux particules, etc.). On lui associé pour un référentiel donné 4 nombres: $t,x,y,z$. Pas nécessairement les mêmes pour une autre référentiel. 

\paragraph{Transition} Comment passer de l'un à l'autre ?

\subsection{Transformation de Lorentz spéciale}  

Transformation vérifiant ces deux principes, pour deux référentiels en translation rectiligne uniforme l’un par rapport à l’autre.

\textbf{N.B.} (ne pas forcément en parler, mais l'avoir en tête) En première approche, ces deux postulats + homogénéité de l'espace-temps + isotropie de l'espace mène vers Invariance de la norme au carré de
l’intervalle d’espace-temps $\Delta s^2 = c^2 \Delta t^2 - \Delta x^2 - \Delta y^2- \Delta z^2$. Les transformation vérifiant cette invariances sont les transformations de Lorentz.

[Langlois]

Poser le cadre (même origine, mouvement relatif de l'axe $x'$ parallèle à l'axe $x$, etc.). \textbf{Avoir une idée de comment on les dérive.} Transformations (spéciales) de Lorentz:
\begin{align*}
ct' &= \gamma(ct-\beta x) \\
x'  &= \gamma (x-\beta ct) \\
y'  &= y \\
z'  &= z
\end{align*}

Remarquer que la transformation réciproque est simplement obtenue en inversant le sens de la vitesse

\textbf{N.B.} Il y a une forme plus générale (axe quelconque).


\section{Conséquences sur la cinématique}

[Langlois]

Les transformations de Lorentz ont de nombreuses conséquences pratiques assez surprenantes et contre-intuitives. Le temps n’est plus universel : il s’écoule différemment selon le référentiel considéré. Le concept de simultanéité devient relatif. La synchronisation des horloges dans un référentiel
donné nécessite d’élaborer un protocole d’échanges de signaux lumineux (car c est constante et universelle).

\subsection{Perte de la simultanéité}


\subsection{Dilatation du temps}

Temps propre. Pour un observateur, le temps $\Delta \tau$ mesuré dans le référentiel qui lui est attaché semble toujours
s’écouler plus lentement que dans tout autre référentiel $\Delta t = \gamma \Delta \tau > \Delta \tau$.

Calculer $\frac{\tau}{\tau_p}$ à comparer avec $\gamma$.

\subsubsection{Expérience de Frisch et Smith}


[Pérez, p. 46] 

Introduire le Muon. Instabilité (désintégration) $\mu \rightarrow e^- + \nu_\mu + \bar{\nu}_e$. Loi de désintégration $N(t) = N_0 \e^{-t/\tau}$. Temps moyen de désintégration \textbf{\emph{au repos}} $\tau_p = 2.198 \pm 0.02 \mu s$. Vitesse $0.992c$. Mesure du flux (muons par heure) en haut et en bas de la montagne. $t_1$ et $t_2$ instants de détection, $\ln \frac{n_2}{n_1} = - \frac{t_2 - t_1}{\tau}$. En déduire $\tau$. Interpréter la différence avec le temps propre: il faut prendre en compte les deux référentiels (Muon et terrestre, pas pareil). \textbf{Faire les calculs dans un tableur ou sur Jupyter par exemple. Prendre en compte les incertitudes. cf. \url{http://supernovae.in2p3.fr/~llg/Enseignements/Agregation/Relativite/biblio/Frisch-et-Smith--1963--Measurement-of-the-Relativistic-Time-Dilation-Using-Mu-Mesons.pdf}}

Parler de l'expérience récente du CERN qui donne un super valeur [Pérez].


\subsection{Contraction des longueur}

Exemple théorique élémentaire (position des extrémités d’un objet, immobile dans un référentiel et en mouvement dans l’autre). Un diagramme d’espace-temps peut aider. Si vous voulez illustrer l’effet de contraction des distance expérimentalement, vous pouvez évoquer l’interprétation du résultat de l’expérience de Frisch et Smith dans le référentiel propre des muons : du point de vue d’un muon, sa vie moyenne dure
bien $2.2\mu  s$ dans son référentiel, mais la hauteur parcourue dans l’atmosphère est par contre contractée (contraction de FitzGerald-Lorentz), du même facteur $\gamma(v)$.

\subsection{Composition des vitesses}

On pourrait s’attendre à voir traitée la loi relativiste de composition des vitesses, avec une comparaison avec son équivalent classique.

\paragraph{Expérience de Fizeau}

\subsection{Intervalle d'espace-temps}

Invariance.

\section*{Bonus}

\begin{itemize}
\item Diagrammes de Minkowski
\item Intervalle d'espace-temps
\end{itemize}

\section*{Conclusion}
  \addcontentsline{toc}{section}{Conclusion}

La théorie de la relativité est une réalité, on peut mesurer ses effets expérimentalement, même pour de petites vitesses comme le montre l’expérience de Fizeau. Ces effets sont à prendre en compte (GPS). Parler aussi des horloges en mouvement [Fabre, p.51].

\newpage





%------------------------------------------


\chapter{Effet tunnel : application à la radioactivité alpha}


\paragraph*{Niveau:} PC
\paragraph*{Prérequis:} 
\begin{itemize}
\item Physique ondulatoire
\item Notion de mécanique quantique (Schrödinger indépendante
du temps, courant de probabilité, puits
de potentiel de profondeur fini afin de déjà avoir la
notion d’onde évanescente)
\item Notions de lycée sur la radioactivité
\end{itemize}

\paragraph*{Bibliographie:}
\begin{itemize}
\item Introduction à la physique quantique. Jean-Louis Basdevant. De Boeck Supérieur.
\item Physique quantique : Tome 1, Fondements. M. Le Bellac. EDP Sciences.
\item Quantique : fondements et applications. Pérez.
\item Tout-en-un PC. Sanz.
\item Les nouveaux précis Physique PC. Bréal. Tisserand, Brendals et al.
\end{itemize}

\paragraph*{Notes agrégat}
\begin{itemize}
\item 2017 : Encore une fois, il ne s’agit pas de se limiter à des calculs. L’exposé doit présenter l’analyse d’applications pertinentes.
\item 2015 : Trop de candidats pensent que l’effet tunnel est spécifique à la physique quantique.
\item 2013, 2012, 2011 : Dans le traitement de l’effet tunnel, les candidats perdent souvent trop de temps dans les calculs. Le jury invite les candidats à réfléchir à une présentation à la fois complète et concise sans oublier les commentaires physiques relatifs à la dérivation de la probabilité de transmission. Certains candidats choisissent d’aborder le cas de la
désintégration alpha mais ne détaillent malheureusement pas le lien entre la probabilité de traversée d’une barrière et la durée de demi-vie de l’élément considéré. La justification des conditions aux limites est essentielle ! Le microscope à effet tunnel peut être un bon exemple d’application s'il est analysé avec soin (hauteur de la barrière, origine de la résolution transverse, ...).
\end{itemize}

\section*{Introduction}
  \addcontentsline{toc}{section}{Introduction}
  
[Karim]
  
Les lois de la physique quantique sont très différentes de celles de la physique classique. De nombreux phénomènes contre-intuitifs : dualité onde-particule, relations d'incertitude, ubiquité d'une particule.

L'effet tunnel est une conséquence de la dualité.

\paragraph{Présentation rapide de l'effet tunnel} [Bellac, p. 32]. Deux situation: pour un puit de potentiel.\emph{En classique:}
\begin{itemize}
\item Puit: $E>0$: diffusion. $V_0<E<0$: état lié.
%\item Barrière: $E<0$: diffusion (repart en arrière). $E>V_0$ : franchit la barrière.
\end{itemize}

En méca Q: la particule peut franchir la barrière même si $E < V_0$ : c'est l'\textbf{effet tunnel}.

\textbf{N.B.} En méca classique: $E$ est continue dans lié et diffusion. En quantique : $E$ est discrète en lié et continue en diffusion. 

\textbf{N.B.} En quantique : une particule peut rebondir sur une barrière même si $E > V_0$.

\section{Position du problème}  

\textbf{Dans la suite, on se restreint à un pb à 1d.}


[Basdevant] 
  
\subsection{Méthodologie}

Hamiltonien $H = - \frac{\bar{h}^2}{2m} \Delta + V(\bm r)$. Conservation de l'énergie: on se ramène à $\psi = \phi(\bm r) \e^{-Et/\bar{h}}$ et au problème aux valeurs propres $H \phi = E \phi$.


\subsection{États liés et états de diffusion}

\subsubsection{États liés}

Potentiel confinant, fonction d'onde vérifie l'équation aux valeurs propres \emph{et} est normalisable.
On montre mathématiquement que l'ensemble $\{\phi_n, E_n\}$ est discret. Etat lié général = superposition d'états liés stationnaires. 

\subsubsection{États de diffusion}

Potentiel non confinant. Il existe des solutions de l'équation aux valeurs pour un ensemble \textbf{continu} d'énergies \textbf{SI} la fonction d'onde est non-normalisable. Ces solutions correspondent aux états de diffusion. Pas de pb physique car leurs superpositions, elles, sont normalisables: ce sont des paquets d'onde qui se déplacent dans l'espace. 

\textbf{c'est à ces que l'on va s'intéresser dans la suite.}

\subsection{Cas de la barrière de potentiel}

Pour simplifier les calculs, nous modélisons les potentiels réels par formes simplifiées pouvant comporter des discontinuités. Prix à payer: il faut imposer à $\psi$ et à $\psi'$ des conditions de continuité (conséquence directe de l'équation $- \frac{\bar{h}^2}{2m} \psi''(x) + V(x) \psi(x) = E \psi(x)$).

Faire le schéma. 3 zones : I, II et III. On choisit $E < V_0$. On méca classique, la particule rebondit. On va voir ce qui se passe pour la particule quantique.

\section{Effet tunnel}

\subsection{Résolution}

[Sanz] 
\begin{itemize}
\item Régions I et III: $\varphi'' + k^2 \varphi = 0$
\item Région II: $\varphi'' - q^2 \varphi = 0$
\end{itemize}
avec $k = \frac{\sqrt{2mE}}{\bar{h}}$ et $q = \frac{\sqrt{2m(V_0 - E)}}{\bar{h}}$.

\begin{enumerate}
\item Écrire les solutions dans les 3 régions. 6 inconnues.
\item Discuter l'interprétation physique dans chaque zone (I: incidente + réfléchie ; III: transmise, pas d'incidente de l'autre côté: $B_3 = 0$ ; II: superposition de deux ondes évanescentes). 5 inconnues.
\item 4 conditions de continuité sur $\varphi$ et $\varphi'$. 1 inconnue. On peut exprimer $A_1$ en fonction de toutes les autres: pas de quantification.
\end{enumerate}

\subsection{Coefficients de réflexion et de transmission}

[Bréal, p. 731 et p. 754 ou encore Pérez]

On écrit les équations issues des conditions de continuité où on a introduit
\begin{align*}
r &= \frac{B_1}{A_1} \\
t & = \frac{A_3}{A_1} \\
x &= \frac{A_2}{A_1} \\
y &= \frac{ B_2}{A_1}
\end{align*}
\begin{itemize}
\item Calcul formel : \url{https://www.wolframalpha.com/input?i=1+%2B+r+%3D+x+%2B+y+%2C+i*k*%281+-+r%29+%3D+q*%28x+-+y%29+%2C+x*exp%28q*a%29+%2B+y*exp%28-q*a%29+%3D+t*exp%28i*k*a%29%2C+q*%28x*exp%28q*a%29+-y*exp%28-q*a%29%29+%3D+i*k*t*exp%28i*k*a%29} \\
\item Entrer: \textbf{1 + r = x + y , i*k*(1 - r) = q*(x - y) , x*exp(q*a) + y*exp(-q*a) = t*exp(i*k*a), q*(x*exp(q*a) -y*exp(-q*a)) = i*k*t*exp(i*k*a)}.
\item Appuyer sur l'expression de $t$ dans un nouvel onglet.
\item Prendre l'expression simplifié de $t$ faisant intervenir $sh$ et $ch$ et la simplifier au tableau pour retrouver l'expression du Pérez.
\item Idem pour $r$.
\end{itemize}

On écrit les courants de probabilité associés ($j(x,t) = \frac{\bar{h}}{2mi} Re[\psi*(x,t) \frac{\partial \psi(x,t)}{\partial x}]$). On calcule $R$ et $T$ comme la norme au carré de $r$ et $t$. Faire le calcul au tableau pour $t$ et retrouver l'expression du DUNOD. Balancer $R$.

Discuter:
\begin{itemize}
\item $T \neq 0$: effet tunnel.
\item $R+T = 1$: conservation de l'énergie.
\end{itemize}

Jouer sur la simulation \url{https://phet.colorado.edu/sims/cheerpj/quantum-tunneling/latest/quantum-tunneling.html?simulation=quantum-tunneling}

\subsection{Cas de la barrière épaisse}

[Sanz]

\section{Application à la radioactivité $\alpha$}

[Sanz]

\subsection{Approche expérimentale}

Rappel de la définition. Constatation expérimentale de la radioactivité alpha: la demi-vie de la particule alpha est d’autant plus courte que l’énergie cinétique de la particule est grande. 

\paragraph{Transition} nous allons expliquer ces temps de vie grâce au phénomène de l’effet tunnel.

\subsection{Modèle de Gamow, Gurney et Condon}

1928. Gamow et, de façon concomitante et indépendante, Gurney et Condon.

Afin de décrire ces phénomènes on adopte un modèle. On suppose que la particule alpha existe à l’intérieur du noyau. Elle est soumis à $E_p$. Tracer la forme du potentiel : c'est un puit. 

\subsection{Probabilité de transmission}

[JF] On reprend l’expression du coefficient de transmission de la barrière. On approxime la barrière variant continument par plusieurs barrières rectangulaires. Le coefficient de transmission global
est le produit des coefficients de transmission. On obtient $\ln T$ [Sanz].

Eventuellement utiliser \url{http://hyperphysics.phy-astr.gsu.edu/hbase/Nuclear/alpdec.html}
  
\section*{Conclusion}
  \addcontentsline{toc}{section}{Conclusion}

Slide: autre application, la microscopie à effet tunnel [Sanz]. Montrer \url{https://www.youtube.com/watch?v=DC0U5viudt0} ou \url{https://www.youtube.com/watch?v=NEsbREz-BBU}

Si le temps, finir sur \url{https://www.slate.fr/life/71883/ibm-produit-la-premiere-video-atomique}


\newpage

%------------------------------------

\begin{comment}


\section*{Description de l'expérience}
  \addcontentsline{toc}{section}{Description de l'expérience}



\begin{tcolorbox}[breakable, enhanced, colback=red!2!white,colframe=mycolor!85!black,title=\textbf{\textbf{Expérience}}]
\paragraph*{Matériel}
\begin{itemize}
\item 
\end{itemize}

\paragraph*{Protocole } 

\begin{itemize}[label=$\triangleright$]
		\item 
\end{itemize}

\paragraph*{Aspect quantitatif :} Mesure de 
\end{equation}


\end{tcolorbox}


\end{comment}


% Appendices-----------------------------------------

%\newpage
%\appendix
%\section*{Appendix}
%\addcontentsline{toc}{section}{Appendices}


%Biblio ---------------------------------------------

%\clearpage
%\bibliographystyle{plain}
%\bibliography{/home/lydia/Bureau/Biblio/Biblio}
%\bibliography{Biblio}
%\addcontentsline{toc}{section}{Bibliography}

%---------------------------------------------------
\end{document}
