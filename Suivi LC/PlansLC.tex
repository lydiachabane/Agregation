\documentclass[11pt]{report}

%Packages--------------------------------------------

\usepackage[twoside, top=2cm,bottom=1.5cm,  hmargin=1.5cm, bindingoffset=0.2cm]{geometry}

\usepackage[utf8]{inputenc}
\usepackage[T1]{fontenc}
\usepackage[french]{babel}

%----------------------------------------
% Pour trier

\usepackage{datatool}% http://ctan.org/pkg/datatool
\newcommand{\sortitem}[1]{%
  \DTLnewrow{list}% Create a new entry
  \DTLnewdbentry{list}{description}{#1}% Add entry as description
}
\newenvironment{sortedlist}{%
  \DTLifdbexists{list}{\DTLcleardb{list}}{\DTLnewdb{list}}% Create new/discard old list
}{%
  \DTLsort{description}{list}% Sort list
  \begin{itemize}%
    \DTLforeach*{list}{\theDesc=description}{%
      \item \theDesc}% Print each item
  \end{itemize}%
}
%-------------------------------------------

\usepackage{titlesec}
\titleformat{\chapter}[display]{\normalfont\bfseries}{\Huge\thechapter}{0pt}{\Huge}

\usepackage{siunitx}
\usepackage[version=4]{mhchem}

\usepackage{amsmath}
\usepackage{cite}
\usepackage{stmaryrd}
\usepackage{mathtools}
\usepackage{amsfonts}
\usepackage{amssymb}
\usepackage{array}
\usepackage{bbold}
\usepackage{dsfont}
\usepackage{mathrsfs}  
\usepackage{calc}
\usepackage{xcolor}
\usepackage{framed}
\usepackage{bm}
\usepackage{mdframed}
\usepackage{braket}
\usepackage{ulem}
\usepackage{comment}
\usepackage[toc,page]{appendix} 
\usepackage{csquotes}
\MakeOuterQuote{"}
\usepackage{dirtytalk}
\usepackage{epigraph}
\usepackage{subcaption}
\usepackage{tabularx}
\usepackage{booktabs}



\usepackage{lipsum}% pour les faux textes


%%% Encadré %%%%%%%%%%%%%%%%%%%%%%%%%%%%%%%%%%%%
\usepackage{tcolorbox}
\tcbuselibrary{theorems}

\newtcbtheorem[number within=section]{mytheo}{Activité pédagogique}%
{colback=myblue!5,colframe=blue!45!black,fonttitle=\bfseries}{th}


\titleformat*{\section}{\color{mycolor}\bfseries\Large}
\titleformat*{\subsection}{\color{mycolor2}\bfseries\large}
\titleformat*{\subsubsection}{\color{mycolor5}\bfseries\large}


%%%%%%%%%%%%%%%%%%%%%%%%%%%%%%%%%%%%%%%%%%%%%%%%%%%%%

\newtheorem{property}{Property}
\newtheorem{corollary}{Corollary}
\newtheorem{definition}{Definition}
\newtheorem{conjecture}{Conjecture}

\definecolor{Prune}{RGB}{99,0,60} % l14-33: 
\definecolor{mycolor}{RGB}{186, 11, 122} % l14-33: 
\definecolor{mycolor2}{RGB}{4, 87, 87} %{4,139,154} 

\definecolor{mycolor3}{RGB}{23, 101, 125}
\definecolor{mycolor4}{RGB}{252, 93, 93}
\definecolor{mycolor5}{RGB}{149, 0 , 193}


\usepackage[pagebackref, colorlinks=true, citecolor=blue, linkcolor=blue,urlcolor= blue]{hyperref}

\usepackage[capitalise]{cleveref}
\usepackage{caption}
\usepackage{float}

\usepackage{longtable}


\usepackage{graphicx}
\usepackage{pdfpages}


\usepackage{wasysym}

\usepackage{caption}
\usepackage{float}

\usepackage[utf8]{inputenc}
\usepackage[T1]{fontenc}


\usepackage{graphicx}
\usepackage{pdfpages}

\numberwithin{figure}{section}
\numberwithin{equation}{section}
\numberwithin{table}{section}

\usepackage[]{tcolorbox}
\usepackage[]{enumitem}
\usepackage[]{lipsum}
\usepackage[]{multicol}

%%%%%%%%%%%%%%%%%%%%%%%%%%%%%%%%%%%%%%%%%%%%%%%%%%
% Raccourcis 
%%%%%%%%%%%%%%%%%%%%%%%%%%%%%%%%%%%%%%%%%%%%%%%%%%
\newcommand{\ud}{\mathrm{d}}
\newcommand{\e}{\mathrm{e}}

\newcommand{\R}{\mathds{R}}
\newcommand{\E}{\mathds{E}}

\newcommand{\bE}{\bm E}
\newcommand{\bB}{\bm B}
\newcommand{\bl}{\bm l}
\newcommand{\bv}{\bm v}
\newcommand{\br}{\bm r}

\newcommand{\f}{\varphi}
\newcommand{\g}{\boldsymbol{\gamma}}
\newcommand{\ut}{\tau}
\newcommand{\s}{\sigma}
\newcommand{\arcsinh}{\mathrm{arcsinh}}
\newcommand{\va}{A}
\renewcommand{\t}{\tau}
\newcommand{\w}{\boldsymbol{w}}
\newcommand{\ur}{{\mathrm{r}}}

\newcommand{\sm}{\mathrm{s}}
\newcommand{\um}{\mathrm{u}}
\newcommand{\Sr}{S_{\mathrm{r}}}


\renewcommand{\L}{\mathscr{L}}
\renewcommand{\H}{\mathscr{H}}


\newcommand{\A}{\boldsymbol{A}}
\newcommand{\C}{\boldsymbol{C}^n}
\newcommand{\Cd}{C_n}
\newcommand{\Ct}{\boldsymbol{C}^\tt}
\newcommand{\Ctt}{C^\tt}
\newcommand{\V}{\boldsymbol{V}^\tt}
\newcommand{\Vt}{V^\tt}
\newcommand{\bpi}{\boldsymbol{\pi}}
\renewcommand{\a}{\boldsymbol{a}}
\newcommand{\p}{\boldsymbol{p}}
\renewcommand{\P}{\boldsymbol{P}}
\renewcommand{\S}{\boldsymbol{F}}
\newcommand{\bmu}{\boldsymbol{\mu}}
\newcommand{\om}{\boldsymbol{\omega}}
\newcommand{\kk}{\hat{k}}
\newcommand{\bkappa}{\boldsymbol{\kappa}}
\newcommand{\K}{\hat{K}}
\newcommand{\uu}{\boldsymbol{u}}
\renewcommand{\r}{\mathsf{r}}
\newcommand{\er}{\epsilon \r}
\renewcommand{\ll}{\boldsymbol{l}}
\renewcommand{\f}{\boldsymbol{f}}
\newcommand{\ffp}{\boldsymbol{\f'}}
\newcommand{\h}{\boldsymbol{h}}
\newcommand{\hp}{\boldsymbol{h'}}
\newcommand{\M}{\boldsymbol{M}}
\newcommand{\N}{\boldsymbol{N}}
\newcommand{\x}{\boldsymbol{x}}
\newcommand{\y}{\boldsymbol{y}}
\newcommand{\bpsi}{\boldsymbol{\psi}}
\newcommand{\X}{\boldsymbol{X}}
\newcommand{\Y}{\boldsymbol{Y}}
\newcommand{\1}{\boldsymbol{1}}
\newcommand{\un}{\mathcal{N}}
\newcommand{\bg}{\boldsymbol{g}}
\renewcommand{\u}{\boldsymbol{u}}
\renewcommand{\bf}{\boldsymbol{\phi}}
\newcommand{\z}{\boldsymbol{z}}
\newcommand{\Z}{\boldsymbol{Z}}
\newcommand{\brho}{\boldsymbol{\rho}}
\newcommand{\n}{\boldsymbol{n}}
\newcommand{\m}{\boldsymbol{m}}
\renewcommand{\j}{\boldsymbol{j}}
\newcommand{\bDelta}{\bm{\Delta}}
\renewcommand{\d}{\bm{\mathfrak{D}}}
\newcommand{\df}{\mathfrak{D}}
\newcommand{\U}{\bm{U}}
\newcommand{\B}{\bm{B}}
\newcommand{\km}{W}
\newcommand{\kmb}{\mathcal{W}}
\newcommand{\D}{\mathcal{D}}
\newcommand{\tg}{\tilde{G}}
\newcommand{\bA}{\bar{\A}}
\newcommand{\btg}{\bm{\tilde{G}}}
\newcommand{\tips}{\mathrm{TiPS}}
\newcommand{\vp}{\chi}
\renewcommand{\tt}{\mathcal{T}}

\renewcommand{\thesection}{\arabic{section}} 



%----------------------------------------------------------


\selectlanguage{french}


\title{\textbf{Plans des leçons de chimie}}
\date{Concours externe spécial de l'agrégation de physique-chimie, option physique -- Session 2023}
\author{\color{mycolor2}\bfseries \textit{Lydia Chabane}}



%1erepage------------------------------------------

\begin{document}
\maketitle

%Sommaire-------------------------------------------

\begingroup
\hypersetup{linkcolor=black}
\tableofcontents
\endgroup

%Corps-----------------------------------------------



\chapter{Infos générales}

\paragraph{Biblio}
\begin{itemize}
\item Techniques expérimentales en chimie - Classes prépas et concours 3e éd. - Travaux pratiques: Travaux pratiques.
\item L'épreuve orale du capes de chimie - Se préparer efficacement aux montages et à l'épreuve sur dossier. Florence Porteu-de Buchère.
\item Chimie tout-en-un PCSI. Bruno Fosset, Jean-Bernard Baudin, Frédéric Lahitète. Dunod
\item Chimie - MP-MP* PT-PT*.  Pascal Frajman, Alain Demolliens, Corinne Gauthier, Agnès Auberlet-Debove. Classe Prépa, Nathan
\item Chimie PC PC* - 2ème année Tout-en-un. Alain Demolliens, Pascal Frajman, Corinne Gauthier, Jean-Marc Urbain
\item Chimie Générale - PC-PC*. Pascal Frajman, Alain Demolliens, Corinne Gauthier, Agnès Auberlet-Debove. Classe Prépa
\item Chimie organique et polymères PC. Pascal Frajman, Jean-Marc Urbain. Classe Prépa, Nathan
\end{itemize}

\paragraph{Sites internet}
\begin{itemize}
\item \url{https://je-plante-mon-agreg.com/Documents/Chimie_exp%C3%A9rimentale-Listes_TP_chimie.pdf}
\item \url{https://nc.agregation-physique.org/index.php/s/xYaj6os3ntnMsQz?dir=undefined&path=%2F%2FLyon&openfile=47260}
\item \url{https://nc.agregation-physique.org/index.php/s/xYaj6os3ntnMsQz?dir=undefined&path=%2F%2FMarseille&openfile=77726}
\item \url{http://leonardvinci.e-monsite.com/pages/term-stl-np/}
\item 1ST2S : \url{http://olical.free.fr/1ST2S/indexST2S.htm}
\item 1ère Spé: \url{http://olical.free.fr/EXO1S2/INDEXEXO.htm}
\item \url{http://olical.free.fr/}
\end{itemize}

\paragraph{Logiciels}
\begin{itemize}
\item Avogadro : Construire et visualiser des molécules en 3D.
\item ChimGéné : simulations diverses : cristallo, spectro, cinétique, dosages (pH, Redox, etc.), courbes (i-E, E-pH, etc.).
\item Gum-MC : propagateur d'incertitudes
\item Regressi : exploitation de données expérimentales (ex: pH-métrie).
\item Synchronie : pour potentiostat courbe i-E.
\item Dozzzaqueux : Simulation de dosages.
\end{itemize}

\paragraph{Programme}

Les sujets 2023 des leçons de chimie seront choisis par rapport aux programmes en vigueur dans les
différentes classes à la rentrée 2022 :
\begin{itemize}
\item les classes du lycée (filière générale et séries technologiques STI2D, STL et ST2S) (BO spécial
n°1 du 22 janvier 2019 et BO spécial n°8 du 25 juillet 2019)
\item les classes préparatoires aux grandes écoles (CPGE) : classes de première année MPSI, PTSI,
MP2I (BO spécial n°1 du 11 février 2021)
\item les classes de première et seconde année TSI seront ceux du BO n°30 du 29 juillet 2021
\item les classes de seconde année MP, PSI, PT et MPI (BO n°31 du 26 août 2021). 
\end{itemize}

\paragraph*{Introduction pédagogique}

\begin{itemize}
\item Contenu de la leçon : exposer le contenu et les partis pris et les justifier.
\item Contexte pédagogique et disciplinaire : présenter l'avant et l'après, et la place de la leçon. Souligner l'importance des notions abordées et des modes originaux d'évaluation.
\item Difficutés et remédiations (logiciels, code couleur, etc.).
\end{itemize}

\chapter*{Chimie expérimentale}

\section*{Incertitudes}

\section*{Électrodes}


\section*{pH-mètre}

[Lire A.S.]

Le pH-mètre est généralement constitué d'un boîtier électronique permettant l'affichage de la valeur numérique du pH et d'une sonde de pH constituée d'une électrode de verre permettant la mesure et d'une électrode de référence. Son fonctionnement est basé sur le rapport qui existe entre la concentration en ions \ce{H_3O^+} (définition du pH) et la différence de potentiel électrochimique qui s'établit dans le pH-mètre une fois plongé dans la solution étudiée: $\Delta E = a(pH_{mes} -pH_{ref})+ b$.

Celui-ci est constitué de deux électrodes, l'une standard dont le potentiel est constant et connu (appelée électrode de référence), l'autre à potentiel variable (fonction du pH, appelée électrode de verre). Ces deux électrodes peuvent être combinées ou séparées.

L'appareil est étalonné au moyen de deux solutions tampon (pH 4, 7 et 10 disponibles) pour déterminer $a$ et $b$. On peut aussi (après avoir réalisé cet étalonnage) déterminer la valeur du pH par simple corrélation, la différence de potentiel évoluant proportionnellement à la valeur du pH selon la formule suivante.



\part{ST2S}


\chapter{Chimie et alimentation}

\paragraph{Niveau} Tle ST2S

\paragraph{Parties associées dans le BO}
\begin{itemize}
\item Comment la dégradation des aliments peut-elle être ralentie ?
\item Comment la qualité chimique des aliments est-elle repérée ? 	
\item Quelles sont les doses de vitamines et d’oligoéléments nécessaires à l’être humain ?
\item Comment les additifs alimentaires influencent-ils les choix de consommation ?
\end{itemize}

\subsection{Bibliographie}
\begin{itemize}
\item Manuel de Chimie-Tle ST2S. Nathan.
\item Science des aliments, Jeantet Romain, Tec\&Doc Lavoisier
\end{itemize}

\paragraph{Notes LC} On peut faire le choix de se focaliser sur le contrôle qualité des aliments ou bien d’aborder les réactions possibles qui
peuvent mener à l’oxydation et la dégradation des aliments tout en mentionnant les différentes techniques utilisées
pour conserver au
mieux ces aliments.
Il est également possible d’établir un plan de leçon qui permet de combiner ces
deux grands axes.

\paragraph{Expériences possibles}
\begin{itemize}
\item Mettre en œuvre un dosage par titrage pour déterminer la
teneur en vitamine C d’un aliment ou d’un médicament. 
\item À partir d’exemples de la vie quotidienne (brunissement d’un
fruit, rancissement du beurre, caillage d’un lait, etc.), mettre en
œuvre un protocole expérimental permettant d’identifier
quelques facteurs favorisant la dégradation
alimentaire (dioxygène de l’air, température, lumière,
microorganismes, etc.) et de comparer leur influence.
\item Mettre en œuvre un protocole expérimental pour déterminer la
fraîcheur d’un lait conformément aux normes de santé
publique.
\item Mettre en œuvre un protocole expérimental pour identifier et
doser par étalonnage un colorant alimentaire.
\item Pratiquer une démarche expérimentale mettant en évidence la
solubilité des vitamines.
\end{itemize}

\paragraph{A savoir}
\begin{itemize}
\item Vitamine : Molécule polaire, possibilité de former des liaisons hydrogène, donc bonne solubilité (connaître la structure). 
\item Vitamine C: le corps ne peut pas la synthétiser (cf. en bas).
\item Vitamince C: L'organisme ne pouvant la stocker, il en élimine ainsi l'excès. La tolérance intestinale désigne la quantité de vitamine C qui peut être absorbée par l'intestin dans un temps donné. Lorsque cette quantité est atteinte, la vitamine C non absorbée est éliminée dans les selles. Durant son trajet, elle attire de l'eau dans l'intestin ce qui produit une diarrhée passagère. Ceci est une des raisons pour lesquelles on ne peut pas s'intoxiquer avec de la vitamine C.
\item Brunissement des fruits: Oxydation des phénols. Couleur marron: Mélanine (cf en bas).
\item Différence entre colorants et pigments ? En biologie, on parle de pigments pour les molécules qui donnent naturellement une couleur aux organismes qu'on étudie, et de colorants pour les produits qu'on ajoute aux préparations à étudier.
\item On utilise parfois des colorants. Pourquoi sont-ils colorés ? Présence de
liaisons conjuguées. Pour absorber de la lumière, une molécule doit être un système chimique conjugué. Plus une molécule contient de doubles liaisons conjuguées et plus la longueur d’onde des radiations absorbées augmente.
\item Méthodes pour extraire des molécules d’origine naturelle ? Hydrodistillation, extraction par un solvant, chromatographie sur colonne, filtration, etc.
\item Comment distinguer une molécule synthétisée d’une molécule naturelle ? Par spectrométrie de masse en prenant en compte les rapports isotopiques.
\end{itemize}

Certains aliments se dégradent et brunissent en raison de leur o
xydation par l’air
qui est
favorisée par
plusieurs
facteurs que sont
la température ou la lumière. Le phénol
contenu dans les fruits par exemple, la PPO
(polyphenoloxydase , il s’agit d’une enzyme) et le dioxygène entrent en contact ce qui produit de la quinone qui est ensuite oxydée.
Il se forme alors de la mélanine responsable de la couleur brune (équation : phénol + dioxygène $\xrightarrow[]{en présence de PPO}$ quinone + eau). Pour ralentir cette oxydation, il est possible de faire des :
\begin{itemize}
\item emballage sous vide (protection du \ce{O_2}), opaques (protection contre la lumière)
\item ajout d’antioxydants (pour éviter l’oxydation des aliments, ex : acide ascorbique dans le citron).
\item ajout de conservateurs alimentaires (inhibition de la prolifération des microorganismes, ex: acide benzoïque «
E210»)
\item destruction des microorganismes par pasteurisation, stérilisation par la chaleur, surgélation, irradiation
\end{itemize}

La vitamine C est soluble dans l’eau car
il s’agit d’une molécule polaire qui est capable d’établir des liaisons hydrogène
avec l’eau, elle est éliminée de notre organisme par l’urine. Il s’agit également du cas des vitamines du groupe B. en
revanche, les autres vitamines (A, D, E, K) sont liposolubles et sont stockées dans les tissus graisseux et le foie.

\chapter{Molécules d'intérêt biologique}

\paragraph*{Niveau:} 1ère ST2S

\paragraph{Partie du BO}
\begin{itemize}
\item Quelle est la structure des molécules d’intérêt biologique ?
\item Comment les transformations biochimiques des aliments produisent-elles de l’énergie ?
\item Comment les transformations biochimiques des aliments produisent-elles de l’énergie ?
\end{itemize}

\paragraph*{Élément imposé:} Réaliser un protocole permettant de différencier aldéhyde et cétone.

\paragraph{Expériences possibles}
\begin{itemize}
\item \textbf{Test au réactif de Schiff:} réaction de mise en évidence des aldéhydes. En présence de groupements aldéhydes, la fuchsine décolorée, reprend une coloration violacée. La réaction doit se faire à froid et en milieu non-basique. Or, l’expérience montre que la solution reste incolore. Cela peut s’expliquer par le fait qu’en solution, le glucose se trouve sous forme cyclique. La fonction aldéhyde n’étant pas
disponible sous forme libre, le test de Schiff va donc présenter un résultat négatif. Cette expérience peut permettre d’introduire l’existence des formes cycliques des glucides.
\item \textbf{Test à la liqueur de Fehling:}  réaction chimique qui sert couramment à caractériser des aldéhydes et certaines espèces chimiques réductrices (par exemple les sucres réducteurs) par leur oxydation par des ions cuivre(II). Si le produit comporte une fonction aldéhyde alors la liqueur de Fehling, initialement bleue, conduit à un dépôt de couleur rouge brique à chaud.
\item \textbf{Test au miroir d'argent} La réaction de Tollens est une réaction caractéristique des aldéhydes. Ceci du fait des propriétés d'oxydabilité des aldéhydes, que n'ont pas les cétones. La solution de Tollens est un complexe de nitrate d'argent en solution ammoniacale. Au cours de la réaction l'ion argent I oxyde l'aldéhyde pour donner un ion carboxylate. Un dépôt d’argent se forme alors sur les parois évoquant un miroir. N.B. Un autre test avec la 2,4-DNPH est par contre positif sur les aldéhydes et les cétones. Il est caractéristique de la fonction carbonyle.
\item \textbf{Test du biuret:}  réaction mettant en évidence les liaisons peptidiques. En présence de protéine/peptide comportant plus de 2 liaisons peptidiques, la couleur
de la solution de \ce{Cu^{2+}} initialement bleue et en milieu basique prend une coloration violette. Ce test a été réalisé sur la caséine préalablement extraite du lait. Il est possible de rendre cette expérience quantitative en réalisant un dosage spectrophotométrique à l’aide d’une courbe d’étalonnage. 
\item \textbf{Hydrolyse de la caséine:} Dans cette expérience, on réalise l’hydrolyse de la caséine qui est une protéine du lait par de l’acide
chlorhydrique. On procède ensuite à une filtration en présence de charbon actif. 
\textbf{Propriétés chimiques de la vitamine C:} Cette expérience permet de mettre en évidence les propriétés réductrices de la vitamines C. Pour ce faire, quelques gouttes de vitamine C sont introduites dans un tube à essai contenant préalablement une solution de permanganate de potassium. La solution initialement violette devient alors incolore en présence de vitamine C. 
\end{itemize} 

\paragraph*{Bibliographie:}
\begin{itemize}
\item Manuels de 1ère ST2S (Nathan 2012 et 2019, Hachette 2012)
\item Chimie du petit-déjeuner. M. Terrin et J. Fournier
\item Un livre de biochimie (ex: Toute la biochimie. DUNOD)
\item \url{https://www.digischool.fr/cours/la-structure-des-molecules-d-interet-biologique}
\item \url{http://olical.free.fr/1ST2S/indexST2S.htm}
\item \url{https://www.digischool.fr/cours/la-structure-des-molecules-d-interet-biologique}
\item \url{http://je-plante-mon-agreg.com/Documents/LC_14-2022-02-17_LC_14_CR.pdf}
\item \url{http://je-plante-mon-agreg.com/Documents/LC_14-2022-02-17_LC_14_CR%20Prof.pdf}
\end{itemize}

\section*{Introduction}
\addcontentsline{toc}{section}{Introduction}

Une biomolécule est une molécule présente naturellement dans un organisme vivant et qui participe à son métabolisme et à son entretien, par exemple les glucides, les lipides, les protéines, et les acides nucléiques.

\paragraph*{Fil conducteur} Étiquette compote (ou autre).

\paragraph{Notes LC} Pour introduire les définitions de glucides et de lipides de manière plus pédagogique, il est possible
de montrer des exemples de structures de glucides/lipides, de préciser où ces molécules peuvent
être retrouvées dans notre alimentation et d’entourer les groupements fonctionnels que les élèves connaissent déjà pour enfin en déduire une définition plus générale.

Il faut bien être vigilant au vocabulaire, dans les glucides, on trouve des fonctions alcools ou des
groupements hydroxyles (OH) qui ne doivent pas être confondus avec l’anion hydroxyde (\ce{HO^-}). 

Dans cette leçon, de nombreux tests caractéristiques sont attendus pour justement mettre en
évidence la présence des différentes fonctions chimiques. Lors de ces tests, il est indispensable de
faire des tubes témoins qui serviront de contrôles positifs et négatifs. L’un des tubes doit comporter
une molécule qui ne réagira pas avec le réactif (exemple : de l’eau) et l’autre tube doit comporter
une molécule pour laquelle le test sera forcément positif (cette molécule ne sera pas nécessairement une molécule d’intérêt biologique).



\section{Notions à aborder}

Dans cette leçon, on se propose d’aborder d’un point de vue moléculaire les molécules d’intérêt biologique vues en classe de première ST2S. On peut alors citer les 4 grandes classes de molécules :
les glucides, les lipides, les peptides/protéines et les vitamines. 

\subsection{Glucides}

\textit{On montre quelques glucides et on identifie les fonctions communes puis on arrive à une définition générale.}

classe de composés organiques contenant un groupe carbonyle (aldéhyde ou cétone) et au moins deux groupes hydroxyle (-OH).

\paragraph{Cas des glucides simples (les oses)}  monomère de glucide, molécules simples, non hydrolysables, formant des cristaux incolores. Les oses possèdent au moins trois atomes de carbone. Exemple: le glucose, le galactose ou le fructose. On distingue les \textbf{aldoses} (glucides possédant un groupe aldéhyde sur le premier carbone. Exemple: glucose: l'isomère D, également appelé « dextrose », est très répandu dans le milieu naturel, tandis que l'isomère L y est très rare. Le D-glucose est stocké chez les plantes sous forme d'amidon et, chez les animaux, sous forme de glycogène, qui peuvent être hydrolysés à tout moment pour redonner des molécules de D-glucose prêtes à être dégradées en fournissant de l'énergie dès que la cellule en a besoin. Autre exemple: galactose. Le D-galactose est présent dans le lait, Le miel en contient environ 3\%), et mes \textbf{cétoses} (les glucides possédant un groupe cétone sur le deuxième carbone: fructose que l'on trouve en abondance dans les fruits et le miel).


\subsection{Lipides}

Les lipides constituent la matière grasse des êtres vivants. Ce sont des molécules hydrophobes ou amphiphiles — molécules hydrophobes possédant un domaine hydrophile — très diversifiées, comprenant entre autres les graisses, les cires, les stérols, les vitamines liposolubles, les mono-, di- et triglycérides, ou encore les phospholipides.

Les lipides peuvent être classés selon la structure de leur squelette carboné (atomes de carbone chaînés, cycliques, présence d'insaturations, etc). Toutefois, du fait de leur diversité et de la difficulté à adopter une définition universelle, il n'existe pas de classification unique des lipides. La classification des lipides actuellement généralement acceptée établit huit classes, fondées en partie sur les définitions de l'IUPAC, dont les acides gras, les glycérides et les stérols.

\subsubsection{Acide gras}

Donner sur slide des exemples: 
\begin{itemize}
\item Acide oléique:  acide gras mono-insaturé à 18 atomes de carbone. C'est le plus abondant des acides gras dans la nature. Il est le plus abondant dans le tissu adipeux humain et le plasma. Son nom vient de l'huile d'olive dont il constitue 55 \% à 80 \%, mais il est abondant dans toutes les huiles animales ou végétales, par exemple dans l'huile de pépins de raisin ou le beurre de karité. C'est un excellent aliment énergétique. Il est utilisé pour la fabrication des savonnettes.
\item Acide stéarique: acide gras saturé.  Il est le plus répandu des acides gras saturés après l'acide palmitique. Il sert industriellement à faire des huiles, des bougies et des savons.
\item Acide palmitique: constitue l'un des acides gras saturés les plus courants chez les animaux et les plantes. Comme son nom l'indique, on en trouve dans l'huile de palme, mais aussi dans toutes les graisses et huiles animales (beurre, fromage, lait et viande) ou végétales. Industriellement on utilise l'acide palmitique pour la fabrication des margarines, des savons durs. 
\item Acide arachidique:  acide gras saturé à chaîne longue. Il est présent dans l'huile d'arachide et d'autres huiles végétales, et les huiles de poisson.
\end{itemize}

Acides carboxyliques comportant une longue chaîne carbonée. Si les acides gras les plus courants dans les structures biologiques ont une chaîne aliphatique linéaire, il existe cependant, chez les bactéries, les algues et certaines plantes, ainsi que chez les animaux en petite quantité, des acides gras à chaîne hydrocarbonée ramifiée ou cyclique.

[biochimie] Les acides gras les plus répandus (Tableau 4.1) sont des chaînes hydrocarbonées non ramifiées,
le plus souvent à nombre pair de carbones, de longueur variable, de 12 à 24 carbones,
se terminant par un groupe carboxyle: \ce{R-COOH}.

On distingue les acides gras saturés (que des liaisons simples) et insaturés (une ou plusieurs liaisons doubles, ou triple). Discuter les deux sur l'acide linoléique ou stéarique.

les acides gras permettent aux êtres vivants de stocker de l'énergie sous forme de triglycérides.

\subsubsection{Triglycéride}

\url{https://www.youtube.com/watch?v=evg5yY_9p3k&list=PLGUmnhaRCVylPD9dyPzalEaT7ES1K9rwa&index=5}

Ces lipides servent avant tout à stocker de l'énergie métabolique et constituent l'essentiel de la graisse animale. Les triglycérides sont les plus abondants des glycérides et se retrouvent dans tout le monde vivant. Ils constituent une forme de réserve. Ils sont source d’ATP par la libération des acides gras. Ils sont principalement stockés dans le tissu adipeux (adipocytes).

Analyser la structure d'exemples (Oléine, Linoléine, Stéarine, Trimyristine). En déduire la forme générale.

Sont des glycérides dans lesquels les trois groupes hydroxyle du glycérol sont estérifiés par des acides gras. 

Comme ce sont des esters, ils sont produits par estérification : trialcool (glycérol) + 3 acides carboxyliques (acides gras) $\rightarrow$ triglycéride + 3\ce{H_2O}.


\subsection{Protéines}

Les protéines sont des macromolécules biologiques présentes dans toutes les cellules vivantes. Ce sont des polymères, formées d'une ou de plusieurs chaînes polypeptidiques (chaîne d'acides aminés reliés par des liaisons peptidiques). Chacune de ces chaînes est constituée de l'enchaînement d'acides aminés liés entre eux par des liaisons peptidiques.

On parle de polypeptide lorsque la chaîne contient entre 10 et 100 acides aminés. Au-dessus de 100 (ou encore 50, définition arbitraire) acides aminés on parle généralement de protéine.

\subsubsection{Acide $\alpha$-aminé}

Donner des exemples. En déduire la forme générale.

Un acide $\alpha$-aminé est un composé organique portant sur un même atome de carbone, une fonction amine primaire \ce{-NH_2} et un groupe acide carboxylique \ce{-COOH}. Le "alpha", signifie que le groupe \ce{NH_2} est porté sur le carbone adjacent à \ce{COOH}.

En biochimie, les acides $\alpha$-aminés jouent un rôle crucial dans la structure, le métabolisme et la physiologie des cellules de tous les êtres vivants connus, en tant que constituants des peptides et des protéines. Ils constituent à ce titre l'essentiel de la masse du corps humain après l'eau.

\subsubsection{Liaison peptidique}

Une liaison peptidique est une liaison covalente qui s'établit entre la fonction carboxyle portée par le carbone $\alpha$ d'un acide aminé et la fonction amine portée par le carbone $\alpha$ de l'acide aminé suivant dans la chaîne peptidique.


\subsection{Vitamines}

Une vitamine est une substance organique, nécessaire en quantité mineure (moins de 100 mg/jour - voir tableau ci-dessous) au métabolisme d'un organisme vivant, qui ne peut être synthétisée en quantité suffisante par cet organisme. Chaque organisme a des besoins spécifiques : une molécule peut être une vitamine pour une espèce et ne pas l'être pour une autre. C'est par exemple le cas de la vitamine C indispensable aux primates mais pas à la plupart des autres mammifères.

Généralement, on sépare les vitamines en deux groupes : les vitamines hydrosolubles (solubles dans l'eau) et les vitamines liposolubles (solubles dans les graisses).

\section{A savoir}
\begin{itemize}
\item Il existe souvent une association erronée entre les termes de « glucides complexes » et ceux de « glucides (ou même sucres) lents » ainsi qu'entre les termes de « glucides simples » et ceux de « glucides (ou sucres) rapides ». De même, les mots « sucres » et « glucides » sont souvent utilisés en tant que synonymes dans la communication des groupes agroalimentaires français et par une partie du monde médical, alors qu'ils ne le sont jamais dans les définitions scientifiques.
\item Les glucides sont habituellement répartis entre oses (monosaccharides tels le glucose, le galactose ou le fructose) et osides, qui sont des polymères d'oses (oligosaccharides et polysaccharides). Les disaccharides (diholosides), tels le saccharose, le lactose ou le maltose font partie de cette dernière catégorie. Mais seuls les monosaccharides et les disaccharides ont un pouvoir sucrant. Les polysaccharides, comme l'amidon, sont insipides.
\item Les sucres complexes on peut les hydrolyser et pas les simples.
\item Les acides gras seul sont rares dans l’organisme, les plus abondants sont
les acides palmitiques et stéariques.
\end{itemize}

\part{STL - SPCL}

\chapter{Réactivité des alcools}

\paragraph*{Niveau:} 1ère STL - SPCL

\paragraph{Biblio}
\begin{itemize}
\item Chimie PCSI/PC, J’intègre tout en un, Dunod : chapitres acti vations de fonctions chimiques (alcool et carbonyle)
\item Chimie organique, Rabasso
\item \url{https://lycee.stephanegaubert.fr/premierestl/data/1STL_C2D13_cours.pdf}
\item \url{https://lycee.stephanegaubert.fr/sequence.php?classe=1STL&seq=13}
\item \url{https://leonardvinci.e-monsite.com/medias/files/01.reactions-en-chimie-organique.pdf}
\end{itemize}

\paragraph{Expériences possibles}
\begin{itemize}
\item Tests qualitatifs de la présence d’une fonction alcool–oxydation des alcools. Le Maréchal-orga-minéral pp.50-58
\item Test de Lucas: permet de préciser l'appartenance d'un alcool à l'une des trois classes d'alcool. Florilège de chimie pratique Daumarie (pp 64 - 65)
\item Si l'élément l'impose: Synthèse de l’acide acétylsalicylique. Le Maréchal orga - minéral (pp 151 - 155)
\end{itemize}

\paragraph{Notes LC}

Dans cette leçon, il est important de montrer l’importance des alcools dans différents domaines de la chimie ( santé, énergie, environnement ). Ainsi, il est primordial pour un chimiste de comprendre comment cette fonction réagit. 

Dans une première partie, on peut présenter les différentes réactivités des alcools:
\begin{itemize}
\item acido - basique (pka, comparaison avec les acides carboxyliques, et explication par la mésomérie)
\item nucléophile (activation nucléophile par déprotonation, i.e. réaction chimique au cours de laquelle un proton \ce{H^+} est retiré d'une molécule, qui va former sa base conjuguée)
\item électrophile (activation électrophile par protonation, i.e. réaction chimique au cours de laquelle un proton est ajouté à un atome, une molécule ou un ion).
\end{itemize}
Dans une seconde partie, on utilise la réactivité des alcools présentée en première partie pour aborder les
différentes réactions dans lesquelles le groupe hydroxyle est engagé. 

\section*{Introduction}

Les alcools sont utilisés dans l'industrie chimique comme :
\begin{itemize}
\item solvants : l'éthanol, peu toxique, est utilisé dans les parfums et les médicaments ;
\item combustibles : le méthanol et l'éthanol peuvent remplacer l'essence et le fioul car leur combustion ne produit pas de fumées toxiques ;
\item réactifs : les polyuréthanes, les esters ou les alcènes peuvent être synthétisés à partir des alcools ;
\item antigels : la basse température de solidification de certains alcools comme le méthanol et l'éthylène glycol en font de bons antigels.
\end{itemize}

\section{Caractérisation des alcools}

Les alcools sont une famille de molécule qui présente un groupement hydroxyle -OH.

\subsection{Classes d'alcool}

Donner trois exemples concrets et les comparer pour arriver à trois classes.

\paragraph{Primaire} le carbone comportant le groupement hydroxyle est lié à deux atomes d’hydrogène.

\paragraph{Secondaire} le carbone comportant le groupement hydroxyle est lié à un atome d’hydrogène.

\paragraph{Tertiaire} le carbone comportant le groupement hydroxyle n'est lié à aucun atome d'hydrogène.

\subsection{Acidité}

Identifier l’atome d’hydrogène labile dans les alcools. Donner des exemples. comparer leurs acidités en raisonnant sur la stabilisation des bases conjuguées par mésomérie. hydrogène labile, pka des alcools, comparaison avec les acides carboxyliques.

\paragraph{Due à la liaison O-H}

La polarisation forte de la liaison O-H donne la possibilité d'une rupture ionique : les alcools constituent donc des acides faibles, et même très faibles (pKa compris en général entre 16 et 18, 10 pour les phénols, dans l'eau) par libération d'un cation \ce{H^+} du groupe hydroxyle. Ils sont donc bien plus faibles que l'eau (à l'exception du méthanol) et ne manifestent leur caractère acide que dans des solutions non aqueuses, en réagissant par exemple avec la base NaNH2 dans une solution d'ammoniaque. On appelle la base conjuguée d'un alcool un ion alcoolate (ou alkoxyde).

\paragraph{Due aux doublets libres de l'oxygène} 

L'un des doublets libres de l'oxygène est capable de capturer un proton : l'alcool est donc une base de Brönsted, indifférente (pKA(ROH2+/ROH) d'environ -2), son acide conjugué, l'ion alkyloxonium, étant un acide fort, ne pouvant être présent qu'en très petite quantité (sauf en présence d'une concentration importante en acide fort).

\subsection{Nucléophilie} L’alcool
n’est pas un bon nucléophile (notion cinétique) car peu chargé. On acti
ve sa nucléophilie par déprotonation pour former sa base conjuguée qui est un meilleure nucléophile.

\subsection{Electrophilie}

Exaltation de l’électrophilie des alcools par protonation.

\section{Réactivité des alcools}

\url{https://www.lachimie.fr/organique/reactions/alcools.php}

\subsection{Oxydation !pas au programme 1ère! - ne pas en parler}

\url{https://spcl.ac-montpellier.fr/moodle/pluginfile.php/11631/mod_resource/content/3/Chapitre%206%20-%20Synthese%20organique%20-%20Activite%202.pdf}
\url{https://bupdoc.udppc.asso.fr/consultation/article-bup.php?ID_fiche=11333}

Un alcool peut être oxydée en aldéhydes, cétones ou acides carboxyliques selon sa classe:
\begin{itemize}
\item Primaire s'oxyde en Aldéhyde
\item Primaire s'oxyde en Cétone
\item Tertiaire ne s'oxyde pas
\end{itemize}

\subsection{Déshydratation (élimination)}


Les alcools peuvent subir une réaction d'élimination d'eau (réaction de déshydratation) à haute température en milieu acide et produire des alcènes. Cette réaction peut être inversée pour synthétiser des alcools à partir d'alcènes et d'eau (réaction d'hydratation des alcènes), mais reste peu fiable car elle produit des mélanges d'alcools.

\subsection{Substitution}

Substitution du groupe hydroxyle par un atome d’halogène : ROH + HX = RX + HOH.

Réaction de substitution entre un acide et un alcool. \url{https://leonardvinci.e-monsite.com/medias/files/01.reactions-en-chimie-organique.pdf}.


Le groupement OH est un mauvais nucléofuge (mauvais groupe partant). La méthode la plus courante est donc de le protoner en milieu acide qui permettra le départ d'une molécule d'eau. Les substitutions conduisant à la formation d'un carbocation sont favorisées dans le cas des alcools tertiaires (SN1). Si le carbocation n'est pas assez stable, les subtitutions SN2 seront donc privilégiées. \url{https://www.lachimie.fr/organique/reactions/alcools.php}

\paragraph{Expérience} Test de Lucas

\subsection{Estérification}

Estérification en milieu acide, avec un acide carboxylique.

\section{Synthèse à partir d’un alcool (aspirine)}

\url{https://marchettibenjamin.files.wordpress.com/2021/05/tp1-correction.pdf}

L’acide acétylsalicylique (AAS), plus connu sous le nom commercial d’aspirine, est un anti-inflammatoire non stéroïdien.

La synthèse consiste en l'estérification de la fonction hydroxyle de l'acide salicylique avec l'anhydride acétique, en milieu acide.

\section*{A savoir}
\begin{itemize}
\item Les phénols, sont parfois considérés comme des alcools particuliers dont le groupement hydroxyle est lié à un carbone d’un cycle benzénique. Leur réactivité étant tellement différente de celle des autres alcools (ici le carbone portant le groupement -OH n'est pas tétraédrique), les phénols sont généralement classés en dehors de la famille des alcools.
\item L'hydrogénation est une réaction chimique qui consiste en l'addition d'une molécule de dihydrogène (H2) à un autre composé.
\end{itemize}

\chapter{Techniques spectroscopiques}

\paragraph*{Niveau:} Tle STL - SPCL

\paragraph*{Bibliographie:}
\begin{itemize}
\item IR et RMN traité dans : Chimie tout-en-un PC ou chimie organique PC. Classe Prépa, Nathan. Frejman et al.
\item Chimie tout-en-un PCSI. Fosset, Baudin et al.
\item Animation \url{https://www.chemtube3d.com/}
\item Physique-Chimie 1ere S, Nathan 2011
\item Cachau. Des expériences de la famille Red-Ox. de boeck
\end{itemize}

\paragraph{Définition} La spectroscopie est l'étude de l'interaction lumière-matière.

\section{Spectre électromagnétique}

$E = h \nu$. $\lambda = \frac{c}{\nu}$.

Domaine UV-visible : Beer-Lamber. Domaine moyen IR ($10^13$-$10^14$ Hz : spectro IR). Domaine ondes radio de hautes fréquences ($60-900$GHz : RMN).

\section{Spectro UV-Visible}

[AS]

Les photons de l'UV-visible permettent d'observer des transitions électroniques.


\subsection{Spectre}

Absorbance $A = \log \frac{I_0}{I_t}$ vs $\lambda$. 

\subsection{Beer-Lambert}

$A(\lambda) = \epsilon(\lambda) \ell c$.

\subsubsection*{Limites}


\subsection{Spectrophotomètre (mono-faisceau)}

Source poly-$\lambda$ - prisme - diaphragme (pour sélectionner une longueur d'onde) - échantillon - photodétecteur.

\subsection{Applications}

Caractériser des molécules, dosage, suivis cinétiques.


\section{Spectro IR}

\url{https://www.chemtube3d.com/spectrovibwater1-ce-final/#models}

Fréquence des photons de l'ordre de grandeur des fréquences de vibration d'élongation ou de déformation. \\
Lorsqu'un photon est absorbé, l'amplitude de vibration augmente, trahissant la présence de tel ou tel enchaînement d'atome. Par exemple \ce{C=C} et \ce{C=O} ont des fréquences d'élongation différentes.

\subsection{Allure du spectre}

Expérience : spectrophotomètre à double faisceau. On trace le pourcentage de transmission $T = \frac{I}{I_0}$ vs $\sigma = \frac{1}{\lambda}$. Spectre = bandes qui pointent vers le bas (signe d'une absorption), caractérisées par
\begin{itemize}
\item $\nu$ maximum d'absorption
\item Intensité relative
\item Largeur
\end{itemize}

\subsection{Table infrarouge}

Deux zones : emprunte digitale (<1500 cm$^{-1}$ et la région [1500-4000]cm$^{-1}$ caractéristique des groupes fonctionnels.

La spectro IR permet donc d'identifier la présence ou l'absence de certains groupes fonctionnels.

Un spectre dans son entier permet l'identification sans ambiguité d'un composé organique (sanf deux énantiomères). Permet aussi le contrôle de la pureté d'un composé.

\section{Interprétation par un modèle simple - Loi de Hooke}

\subsection{Molécule diatomique}

Modélisation par un oscillateur harmonique. Classique: donne $f = \frac{1}{2\pi} \frac{k}{\mu}$. Quantique: donne la quantification. \\
A chaque niveau vibrationnel est associé plusieurs niveaux d'énergie rotationnelle, ce qui explique la largeur de la bande d'absorption.

\subsection{Molécule polyatomique}

Couplage entre les différents OH. Il y a vibration d'élongation et de déformations. Dans de nombreux cas, les couplages sont suffisamment faible pour faire vibrer chaque type de liaison presque indépendamment.

\subsubsection{Règle de sélection} seules les vibrations produisant une variation du moment dipolaire provoquent une absorption en IR.

On admet que Hooke est toujours valable. Plus une liaison est forte ($\equiv$ > \ce{=} > \ce{-}), plus k est grand, plus $\nu$ est grand. \textbf{N.B.} liaison double non-conjuguée > liaison double conjuguée > liaison simple.

\section{Spectro RMN}

\subsection{Spin nucléaire}

Les protons et les neutrons ont aussi un spin. Si leur spin ne se compense pas, les noyaux atomiques possèdent un moment de spin. Quantification de la norme et d'une composante $\vec{I}^2 = \bar{h}^2 I(I+1)$ et $I_z = \hat{h} m_I$. $I$ entier ou demi-entier positif et $m_I$ compris entre $-I$ et $-I$. Moment magnétique de spin $\vec{\mu} = \gamma \vec{I}$.

\subsection{Interaction avec $B$}

La RMN permet de détecter les noyaux avec un spin nucléaire non nul. En l'absence de $B$, les $2I+1$ orientation de $\vec{I}$ ont la même énergie. Si on applique $B$, on lève la dégénérescence (effet Zeeman) $E = -\vec{\mu} \cdot \vec{I}$. On se limite au cas du proton: la différence d'énergie entre les deux états après levée de dégénérescence est très faible.

\subsection{Condition de résonance}

On observe une transition entre les deux états si on applique une radiation de fréquence $h \nu_0 = \Delta E$.

Dans une molécule organique, les protons ne résonnent pas tous de la même manière car le champ magnétique ressenti par chaque proton diffère (écrantage dû au champ $B'$ généré par la circulation des électrons autour du noyau): $\nu_{res} = \nu_0 (1-\sigma)$, $\sigma$ constante d'écrantage. Ecrantage faible: la résonance de l'ensemble des protons ont des fréquences réparties sur une gamme très réduite au regard de la fréquence envoyée (400MHz).

\textbf{N.B.} l'impulsion magnétique envoyée contient toutes les fréquences permettant la résonance de l'ensemble des noyaux des protons de l'échantillon mais pas des autres noyaux. La relaxation vers l'équilibre est accompagnée de l'émission d'un signal détecté par un récepteur et traité par un calculateur : la TF de ce signal temporel permet d'obtenir le spectre de la molécule.


\subsection{Protons équivalents}

S'ils ont la même fréquence de résonance. Échangeables par une opération de symétrie.

\subsubsection{Déplacement chimique}

La fréquence de résonance $\nu_{res}$ d'un proton est caractérisée par son environnement (écrantage) mais aussi par l'intensité de $B_0$ appliqué, donc de l'appareil. Pour que tous les appareils donnent le même spectre, on définit pour chaque proton le déplacement chimique 
\begin{equation}
\delta = \frac{\nu - \nu_{ref}}{\nu_{ref}},
\end{equation}
avec $\nu_{ref}$ la fréquence de résonance d'un proton dans un composé de référence (le tétraméthylsilane (TMS): inerte, insoluble dans la plupart des solvants orga, donne une seule et force résonance à une fréquence relativement faible). S'exprime en ppm (revient à multiplier par $10^6$). $\delta$ ne dépend pas de $B_0$ donc de l'appareil.

\textbf{N.B.} $\delta \approx 10^6 (\sigma_{ref} - \sigma)$. Une augmentation de l'écrantage réduit $\delta$. Plus $\delta$ est élevé (resp. faible), plus un proton est déblindé (resp. blindé).

\subsection{Table des déplacements chimiques}

\begin{itemize}
\item La présence d'atome électronégatif ou groupes attracteurs réduit la densité électronique, donc réduit l'écrantage, donc engendre un déblindage, donc donc augmente le déplacement.
\item Les protons portés par une liaison \ce{C=C} et surtout par un cycle benzénique sont très déblindés (grand déplacement).
\item La présence d'une liaison hydrogène élargit la résonance.
\end{itemize}

\subsection{Spectre}

Un signal de résonance est caractérisé par
\begin{itemize}
\item Position (déplacement chimique)
\item Intensité : l'aire est proportionnelle au nombre de H équivalents. 
\item Aspect (nombre de pics) : dû au phénomène de couplage.
\end{itemize}

\subsection{Couplage}

Chaque noyau disposant d'un moment magnétique influe sur la valeur du champ magnétique ressenti par les autres noyaux.

Le couplage diminue avec la distance. Souvent, on voit que les couplages entre protons séparés par 3 liaisons.

Le couplage d'un proton porté par un hétéroatome avec ses voisins n'est pas toujours observée.


\section{Élément imposé:} Déterminer la concentration d'une espèce à
l'aide d'une droite d'étalonnage établie par spectrophotométrie.

\subsection{Expérience}

\paragraph{Manips possibles} p.38 \url{https://je-plante-mon-agreg.com/Documents/Chimie_exp%C3%A9rimentale-Listes_TP_chimie.pdf}
\begin{itemize}
\item [30] Dosage du bleu brillant dans le curaçao. Physique-Chimie 1ere S, Nathan 2011, p. 118
\item [31] Dosage des ions permanganate contenus dans une solution de Dakin. Cachau, Redox , 3 F.11 (p. 395). 
\end{itemize}

\textbf{Attention:} Il faut que la concentration du produit à doser soit bien dans la gamme de celle de la courbe d'étalonnage (faire une échelle de teinte avant).


\section*{Pour info}

\begin{itemize}
\item Rappel: nombre d'instaturation \ce{C_x H_y O_z X_v N_w}: $N = \frac{2x+2 -y + w - v}{2}$.
\item L'existence d'une liaison hydrogène entre une molécule AH et une molécule (ou ion) B modifie la longueur de liaison A-H. En supposant la liaison A-H modélisée par un ressort de constante de raideur (constante de force) k, la liaison hydrogène modifie aussi la constante de force associée à cette liaison. En conséquence la vibration relative à la liaison A-H n'est pas la même suivant que la molécule AH établit ou non des liaisons hydrogène. Ce point est particulièrement visible en spectroscopie infrarouge, où la bande d'absorption d'un groupement caractéristique peut évoluer suivant que le groupement participe ou non à une liaison hydrogène. Dans le cas de la liaison O-H d'une fonction alcool, on observe en phase gaz une bande d'absorption (pic fin) pour un nombre d'onde = 3600 cm$^{-1}$. En phase gaz, la distance entre molécules est trop grande pour que celles-ci puissent interagir les unes avec les autres. Le groupement OH est dit libre. En phase condensée, on observe pour le même groupement une bande d'absorption (bande large) pour un nombre d'onde voisin de = 3300 cm$^{-1}$. Cette grande modification du spectre met en évidence la présence de liaisons hydrogène entre molécules d'un même échantillon en phase condensée. On parle de groupement OH lié ou associé. Les liaisons hydrogène établies entre groupements OH affaiblissent la liaison covalente O-H. 
\end{itemize}


\newpage




%------------------------------------------





%------------------------------------------



% Appendices-----------------------------------------

%\newpage
%\appendix
%\section*{Appendix}
%\addcontentsline{toc}{section}{Appendices}


%Biblio ---------------------------------------------

%\clearpage
%\bibliographystyle{plain}
%\bibliography{/home/lydia/Bureau/Biblio/Biblio}
%\bibliography{Biblio}
%\addcontentsline{toc}{section}{Bibliography}

%---------------------------------------------------
\end{document}
